
\section{Resolu\cao\ usando s\'eries de Fourier }

Resolvemos a equa\cao\ da onda
\begin{eqnarray}
\uxx - \frac{1}{c^{2}}\utt =0
\end{eqnarray}
aplicando o m\'etodo de separa\cao\ de vari\'aveis e em seguida
usando desenvolvimento em s\'eries de Fourier.

\subsection {S\'eries de Fourier}

Seja $f(x)$ uma fun\cao\ peri\'odica de per\ih odo $2\pi$, integr\'avel e absolutamente integr\'avel. Considere uma expans\ao\ de $f(x)$ como abaixo:
\begin{eqnarray}
f(x) = \frac{a_{0}}{2} + \sum_{n=1}^{\infty} (a_{n}\cos (nx) +
b_{n}\sin (nx)).
\end{eqnarray}

Suponha que a s\'erie convirja uniformemente no intervalo $-\pi
\leq x \leq \pi$. Multiplicando a equa\cao\ por $\cos (mx)$, com $m$
inteiro positivo, obtemos:
\[
f(x)\cos (mx) = \frac{a_{0}}{2}\cos (mx) + \sum_{n=1}^{\infty}(a_{n}\cos (nx)\cos (mx) + b_{n}\sin (nx)\cos (mx)).
\]

Como a s\'erie continua sendo uniformemente convergente, podemos
integrar termo a termo:
\begin{eqnarray*}
\int_{-\pi}^{\pi} f(x)\cos (mx)dx &=& \int_{-\pi}^{\pi} \frac{a_{0}}{2}cos (mx)dx +
\sum_{n=1}^{\infty}\int_{-\pi}^{\pi} a_{n}\cos (nx)\cos (mx)dx \\
&+& \sum_{n=1}^{\infty}\int_{-\pi}^{\pi} b_{n}\sin (nx)\cos
(mx)dx.
\end{eqnarray*}

Para resolver a equa\cao\ acima usamos as propriedades de ortogonalidade
de senos e cossenos. Observe que as somas na equa\cao\ acima come\c cam
em $n=1$, portanto os valores de $m=n=0$ n\ao\ nos interessam:
\begin{enumerate}
\item $\int_{-\pi}^{\pi} \sin (nx)\cos (mx)dx = 0$ \\
{\bf Demonstra\cao:}
\begin{enumerate}
\item $m \neq n$:
\begin{eqnarray*}
\int_{-\pi}^{\pi} \sin (nx)\cos (mx)dx &=&
    \int_{-\pi}^{\pi} \frac{1}{2} \sin (n+m)xdx
+ \int_{-\pi}^{\pi} \frac{1}{2}\sin (n-m)xdx \\
&=& \left[-\frac{1}{2}\frac{\cos(n+m)x}{n+m} 
    - \frac{1}{2}\frac{\cos(n-m)x}{n-m}\right]_{-\pi}^{\pi} = 0.
\end{eqnarray*}
\item $m = n$: 
\begin{eqnarray*}
\int_{-\pi}^{\pi} \sin (nx)\cos (nx)dx = \int_{-\pi}^{\pi} \frac{1}{2} \sin (2n)xdx 
= \left[-\frac{1}{2}\frac{\cos (2n)x}{2n}\right]_{-\pi}^{\pi} = 0.
\end{eqnarray*}
\end{enumerate}
\item $ \int_{-\pi}^{\pi} \cos (nx)\cos (mx)dx =
\left\{
\begin{array}{ll}  
 0, & m \neq n \\ 
 \pi, & m = n
\end{array}
\right.$ \\
{\bf Demonstra\cao:}
\begin{enumerate}
\item $m \neq n$:
\begin{eqnarray*} 
\int_{-\pi}^{\pi} \cos (nx)\cos (mx)dx &=& \int_{-\pi}^{\pi} \frac{1}{2}\cos (n+m)xdx
+ \int_{-\pi}^{\pi} \frac{1}{2}\cos (n-m)xdx \\
&=& \left[\frac{1}{2}\frac{\sin (n+m)x}{n+m} + \frac{1}{2}\frac{\sin (n-m)x}{n-m}\right]_{-\pi}^{\pi} = 0. 
\end{eqnarray*}
\item $m = n$: 
\begin{eqnarray*}
\int_{-\pi}^{\pi} \cos (nx)\cos (mx)dx &=& \int_{-\pi}^{\pi} \frac{1}{2}\cos (n+m)xdx
+ \int_{-\pi}^{\pi} \frac{1}{2}\cos (n-m)xdx \\
&=& \left[\frac{1}{2}\frac{\sin (n+m)}{n+m}x + \int_{-\pi}^{\pi} \frac{1}{2}dx\right]_{-\pi}^{\pi} = 0 + \pi = \pi.
\end{eqnarray*}
\end{enumerate}
\item $ \int_{-\pi}^{\pi} \sin (nx)\sin (mx)dx = 
\left\{
\begin{array}{ll}
0, &   m \neq n \\
\pi, &  m = n
\end{array}
\right.$  \\
{\bf Demonstra\cao:} An\'aloga \`a anterior.
\end{enumerate}

Assim todos os termos da soma do lado direito da equa\cao\ se
anulam exceto para o caso $n=m$. Ent\ao\ temos:
\[\int_{-\pi}^{\pi} f(x)\cos (mx)dx = a_{m}\pi,\]
e conseguimos assim determinar os coeficientes $a_{m}$.

Para determinar os $b_{m}$ basta multiplicar (2) por $\sin (mx)$ e
repetir os c\'alculos. Ao final temos:
\[
\int_{-\pi}^{\pi} f(x)\sin (mx)dx = b_{m}\pi.
\]
Para obter $a_{0}$ integramos (2):
\[
\int_{-\pi}^{\pi} f(x)dx =
\int_{-\pi}^{\pi} \frac{a_{0}}{2}dx +
   \sum_{n=1}^{\infty}\int_{-\pi}^{\pi} a_{n}\cos (nx)dx +
   \sum_{n=1}^{\infty}\int_{-\pi}^{\pi} b_{n}\sin (nx)dx =
\left[ \frac{a_{0}}{2}x\right]_{-\pi}^{\pi} = a_{0}\pi.
 \]
Portanto temos:
\begin{eqnarray*}
a_{n} &=& \frac{1}{\pi}\int_{-\pi}^{\pi} f(x)\cos (nx)dx \\
b_{n} &=& \frac{1}{\pi}\int_{-\pi}^{\pi} f(x)\sin (nx)dx \\
a_{0} &=& \frac{1}{\pi}\int_{-\pi}^{\pi} f(x)dx.
\end{eqnarray*}
Mas, se ao inv\'es de representar a fun\cao\ no intervalo $(-\pi,\pi)$
desejarmos representar uma $f(x)$ de per\ih odo $2l$ no intervalo $(-l,l)$,
fazemos:
\begin{eqnarray*}
a_{n} &=& \frac{1}{l}\int_{-l}^{l} f(x)\cos \frac{n\pi x}{l}dx, \\
b_{n} &=& \frac{1}{l}\int_{-l}^{l} f(x)\sin \frac{n\pi x}{l}dx, \\
a_{0} &=& \frac{1}{l}\int_{-l}^{l} f(x)dx. 
\end{eqnarray*}
No caso especial em que $f(x)$ \'e uma fun\cao\ par ou \ih mpar temos:
\begin{enumerate}
\item $f(x)$ \'e uma fun\cao\ par:
\begin{eqnarray*}
b_{n} &=& 0, \\
a_{n} &=& \frac{2}{l}\int_{0}^{l} f(x)\cos \frac{n\pi x}{l}dx.
\end{eqnarray*}
\item $f(x)$ \'e uma fun\cao\ \ih mpar:
\begin{eqnarray*}
a_{n} &=& 0, \\
b_{n} &=& \frac{2}{l}\int_{0}^{l} f(x)\sin \frac{n\pi x}{l}dx.
\end{eqnarray*}
\end{enumerate}

\subsection {Resolu\cao\ da equa\cao\ da onda ac\'ustica}

Resolvemos abaixo a equa\cao\ da onda de uma corda finita (deduzida na
Se\cao\ \ref{dedonda}) com condi\coes\ iniciais e de contorno 
como segue
\begin{eqnarray*}
\uxx - \frac{1}{c^{2}}\utt = 0,  \hspace{1cm} 0 < x < l, \; t > 0 \\
\left\{
\begin{array}{ll}
u(x,0) = f(x), & 0\leq x\leq l, \; t\leq 0 \\
\frac{\partial u(x,0)}{\partial t} = g(x) & \\
u(0,t) = 0 & \\
u(l,t) = 0. &
\end{array}
\right.
\end{eqnarray*}

Procuramos solu\coes\ da forma
\[u(x,t) = X(x)T(t),\]
onde $X$ \'e fun\cao\ somente de $x$ e $T$ \'e fun\cao\ somente de
$t$. Assim $X$ deve satisfazer:
\[X(0) = 0\]
\[X(l) = 0.\]
Depois tentamos escolher $T$ de maneira a satisfazer as
condi\coes\ iniciais.

Substituindo $u(x,t)$ na EDP temos:
\begin{eqnarray}
\utt & = & \frac{d^{2}T}{dt^{2}}X(x)\nonumber \\ 
\uxx & = & \frac{d^{2}X}{dx^{2}}T(x)\nonumber \\ 
\frac{1}{X}\frac{d^{2}X}{dx^{2}} & = &
   \frac{1}{c^{2}T}\frac{d^{2}T}{dt^{2}} \; .
\end{eqnarray}
Para que esta igualdade seja satisfeita para qualquer $x$ e $t$, ambos
os lados de (3) devem ser constantes, portanto:
\[
\frac{1}{X}\frac{d^{2}X}{dx^{2}} = \lambda =
\frac{1}{c^{2}T}\frac{d^{2}T}{dt^{2}},
\]
onde $\lambda$ \'e chamada de constante de separa\cao.

Reescrevendo para $X$ temos:
\[
\frac{d^{2}X}{dx^{2}} = \lambda X,
\]
que nos fornece uma simples EDO. Esta EDO
possui solu\coes\ do tipo
\[
X(x) = e^{px},
\]
onde $p = \pm\sqrt{\lambda}$.

Veja que n\ao\ estamos interessados na 
solu\cao\ trivial que \'e aquela que satizfaz somente o caso $f(x) = g(x) = 0$. Analisemos
ent\ao\ as seguinte situa\coes:
\begin{enumerate}
\item $\lambda > 0$ 
\[X(x) = C_{1}e^{\sqrt{\lambda}x} + C_{2}e^{-\sqrt{\lambda}x},\]
aplicando as condi\coes\ de contorno temos:
\begin{eqnarray*}
X(0) = C_{1} + C_{2} = 0 \\
X(l) = C_{1}e^{\sqrt{\lambda}l} + C_{2}e^{-\sqrt{\lambda}l} = 0 \\
\Rightarrow\ C_{1} = C_{2} = 0
\end{eqnarray*}
\item $\lambda = 0$ 
\[X(x) = C_{1}x + C_{2},\]
aplicando as condi\coes\ de contorno temos:
\begin{eqnarray*}
X(0) = C_{2} = 0 \\
X(l) = C_{1}l + C_{2} = 0 \\
\Rightarrow C_{1} = C_{2} = 0
\end{eqnarray*}
\item $\lambda < 0$ 
\[X(x) = C_{1}\cos (\sigma x) + C_{2}\sin (\sigma x),\] onde 
\[\lambda = -\sigma^{2}.\]
Aplicando as condi\coes\ de contorno: \\
\begin{eqnarray*}
X(0) = C_{1} = 0 \\ 
X(l) = C_{1}\cos (\sigma l) + C_{2}\sin (\sigma l) = 0 \\
\Rightarrow C_{1} = 0, C_{2}\sin (\sigma l) = 0
\end{eqnarray*}
como n\ao\ queremos a solu\cao\ trivial (que \'e obtida no caso em que $C_{2} = 0$), queremos 
\[C_{2} \neq 0, \Rightarrow \sin (\sigma l) = 0 \]
\[\Rightarrow (\sigma l) = n\pi, n = \pm1, \pm2, .... \Rightarrow \lambda =
\frac{-n^{2}\pi^{2}}{l^{2}}\] (autovalores) 
\[\Rightarrow X_{n}(x) = C_{n}\sin (\frac{n\pi x}{l})\] (autofun\coes).
\end{enumerate}
N\ao\ \'e necess\'ario considerar os valores de $\sigma < 0$ pois
estes levam a autofun\coes\ com somente o sinal trocado.

Agora precisamos de um $T_{n}(t)$ que corresponda a cada
$X_{n}(x)$ satisfazendo
\[\frac{1}{X_{n}}\frac{d^{2}X_{n}}{dx^{2}} = \lambda = \frac{1}{c^{2}T_{n}}\frac{d^{2}T_{n}}{dx^{2}},\]
portanto,
\[T_{n}(t) = K_{n}\cos (\frac{n\pi ct}{l}) + Q_{n}\sin (\frac{n\pi ct}{l}).\]

Encontramos ent\ao\ um n\'umero infinito de fun\coes\ candidatas \`a
solu\cao\ 
\[u_{n}(x,t) = A_{n}\cos (\frac{n\pi ct}{l})\sin (\frac{n\pi x}{l}) +
B_{n}\sin (\frac{n\pi ct}{l})\sin (\frac{n\pi x}{l}),\]
onde cada uma dessas fun\coes\ satisfaz as condi\coes\ de
fronteira. 

Utilizando o princ\ih pio da superposi\cao, pela EDP e as
condi\coes\ de contorno serem homog\^ eneas lineares, sabendo que
a combina\cao\ linear de uma solu\cao\ $u_{n}$ com uma solu\cao\
$u_{m}$ tamb\'em \'e solu\cao\, e que por indu\cao\ vemos que uma
combina\cao\ linear de um n\'umero finito de solu\coes\ tamb\'em
\'e solu\cao, podemos supor que uma s\'erie do tipo
\[y(x,t) = \sum_{n=1}^{\infty}(A_{n}\cos \frac{n\pi ct}{l})\sin (\frac{n\pi
cx}{l}) + \sum_{n=1}^{\infty}(B_{n}\sin \frac{n\pi ct}{l})\sin (\frac{n\pi
cx}{l}),\]
tamb\'em \'e solu\cao\ desde que a s\'erie convirja.

Agora devemos escolher dentre estas, ajustando
corretamente $A_{n}$ e $B_{n}$, de maneira que se satisfa\c ca as
condi\coes\ iniciais. Vemos que $f(x)$ e $g(x)$ devem ser dos tipos:
\begin{eqnarray*}
u(x,0) = f(x) = \sum_{n=1}^{\infty}A_{n}\sin \frac{n\pi x}{l} \\
\frac{\partial u}{\partial t}(x,0) = g(x) = \sum_{n=1}^{\infty}B_{n}\frac{n\pi c}{l}\sin \frac{n\pi x}{l} 
\end{eqnarray*}

Vemos que a equa\cao\ acima satisfaz as condi\coes\ de contorno e que
satisfar\'a \`a EDP se for 2 vezes diferenci\'avel e se $A_{n}$
e $B_{n}$ forem ajustados. Note que $y(x,t)$ \'e uma s\'erie de senos de
Fourier em $x$. Logo $f(x)$ e $g(x)$ agora s\ao\ fun\coes\ do
tipo:
\begin{eqnarray*}
u(x,0) = g(x) = \sum_{n=1}^{\infty}B_{n}\frac{n\pi c}{l}\sin (\frac{n\pi x}{l}) \\
\frac{\partial u}{\partial t} = f(x) = \sum_{n=1}^{\infty}A_{n}\sin (\frac{n\pi x}{l}).
\end{eqnarray*}

Para que $f(x)$ e $g(x)$ possam ser desenvolvidas em s\'erie de senos de Fourier essas fun\coes\ devem ser:
\begin{enumerate}
\item fun\coes\ \ih mpares
\item fun\coes\ peri\'odicas de per\ih odo $2l$
\item fun\coes\ integr\'aveis e absolutamente integr\'aveis
\end{enumerate}
Ent\ao\,para $f(x)$ por exemplo, constru\ih mos uma $f^{'}(x)$ tal que
\begin{enumerate}
\item $f^{'}(x) = f(x)$, $0 \leq x \leq l$
\item $f^{'}(x) = f^{'}(x + 2l)$
\item $f^{'}(-x) = -f^{'}(x)$
\end{enumerate}

Em seguida expandimos $f^{'}(x)$ em s\'erie de senos de Fourier no
intervalo $0 \leq x \leq l$. E para $g(x)$ procedemos
de maneira semelhante.

Ent\ao\, se $f$ e $g$ puderem ser desenvolvidas em s\'erie de senos de
Fourier podemos encontrar $A_{n}$ e $B_{n}$ e ent\ao\ teremos uma
express\ao\ para $y(x,t)$.

%\begin{thebibliography}{99}
%\bibitem{Butkov} Butkov, E., 1988, {\em F\ih sica Matem\'atica},
%Guanabara Koogan, Rio de Janeiro, RJ.
%\bibitem{Djairo} Figueiredo, D. G., 1977, {\em An\'alise de Fourier
%e equa\coes\ diferenciais parciais}, Projeto Euclides - IMPA, RJ.
%\end{thebibliography}

