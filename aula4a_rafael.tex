
\section[Meio homog\^eneo e isotr\'opico]{Separa\c{c}\~ao da equa\cao\
elastodin\^amica para um meio homog\^eneo e isotr\'opico}

Consideramos a equa\c{c}\~ao elastodin\^amica para meios
homog\^eneos e isotr\'opicos na nota\c{c}\~ao de componentes.
Em tal meio, as derivadas de $\lambda$ e $\mu$ s\ao\ zero, fornecendo, a
partir da equa\cao\ (\ref{eqelaiso}), a seguinte express\~ao
\begin{equation}
(\lambda+\mu)u_{k,ki}+\mu u_{i,kk}+f_{i}=\rho u_{i,tt}.
\label{eqelahom}
\end{equation}
Agora, aplicamos o divergente
na equa\c{c}\~ao acima:
\begin{equation}
(\lambda+\mu)u_{k,kii}+\mu u_{i,kki}+f_{i,i}=\rho u_{i,itt},
\label{div}
\end{equation}
Usando a nota\c{c}\~ao $\theta=u_{k,k}$, a equa\c{c}\~ao
(\ref{div}) pode ser reescrita como
\begin{equation}
(\lambda+\mu)\theta_{,ii}+\mu \theta_{,kk}+f_{i,i}=\rho \theta_{,tt} %\Rightarrow 
\end{equation}
ou
\begin{equation}
\frac{\lambda+2\mu}{\rho}
\theta_{,ii}+\rho^{-1}f_{i,i}=\theta_{,tt}.
\label{eq:div-fim}
\end{equation}
Aplicando o rotacional
na equa\c{c}\~ao (\ref{eqelahom}), temos
\begin{equation}
(\lambda+\mu)\epsilon_{jli}u_{k,kil}+\epsilon_{jli}\mu
u_{i,kkl}+\epsilon_{jli}f_{i,l}=\rho\epsilon_{jli} u_{i,ltt}.
\label{rot}
\end{equation}
Usando a nota\c{c}\~ao $\Omega_{j}=\epsilon_{jli}u_{i,l}$ e
notando que $\epsilon_{jli}u_{k,kil}=0$, a equa\c{c}\~ao
(\ref{rot}) pode ser escrita como
\begin{equation}
\frac{\mu}{\rho}\Omega_{,kk}+\rho^{-1}\epsilon_{jli}f_{i,l}=\Omega_{j,tt}.
\label{eq:rot-fim}
\end{equation}
Ent\~ao, vemos que as equa\coes\ \ref{eq:div-fim} e \ref{eq:rot-fim} tem uma
estrutura semelhante. Essas equa\c{c}\~oes podem se escritas na forma vetorial,
respectivamente, como 
\begin{equation}
\alpha^{2}\Delta\theta+\rho^{-1}\mbox{div}\vec{f}=\frac{\partial^{2} \theta }{\partial t^{2}} ,
\end{equation}
onde $\alpha^{2}=\frac{\lambda+2\mu}{\rho}$, e
\begin{equation}
\beta^{2}\Delta\vec{\Omega}+\rho^{-1}\mbox{rot}\vec{f}=\frac{\partial^{2} \vec{\Omega} }{\partial t^{2}} ,
\end{equation}
onde $\beta^{2}=\frac{\mu}{\rho}$.
As equa\c{c}\~oes acima s\~ao a equa\c{c}\~ao escalar da onda para
$\theta$ e a equa\c{c}\~ao vetorial da onda para $\vec{\Omega}$.

Consideramos o caso unidimensional no qual todas a quantidades
dependendem somente da coordenada espacial $x$. Omitindo o efeito
das for\c{c}as de corpo, isto \'e, $f_{i}=0$. Ent\~ao as
equa\c{c}\~oes acima se reduzem a forma
\begin{equation}
\alpha^{2}\frac{\partial^{2} \theta }{\partial x^{2}}-\frac{\partial^{2} \theta }{\partial t^{2}}=0 ,
\end{equation}
onde $\theta=\frac{\partial u_1}{\partial x}$, e
\begin{equation}
\beta^{2}\frac{\partial^{2} \vec{\Omega} }{\partial x^{2}}-\frac{\partial^{2} \vec{\Omega} }{\partial t^{2}}=0 ,
\end{equation}
onde $\vec{\Omega}=(0,\frac{-\partial u_{3}}{\partial
x},\frac{\partial u_{2}}{\partial x})$.
A solu\c{c}\~ao para a equa\c{c}\~ao escalar tem a seguinte forma
(solu\c{c}\~ao de D'Alambert),
\begin{equation}
\theta=\theta(at+bx) ,
\end{equation}
onde $a$ e $b$ s\~ao constantes a serem determinadas.
Inserindo a solu\c{c}\~ao acima na equa\c{c}\~ao escalar reduzida da
onda, temos
\begin{equation}
\alpha^{2}\theta''b^{2}-a^{2}\theta''=0 ,
\end{equation}
de onde $b^{2}=a^{2}/\alpha^{2}$.
Sem perda de generalidade podemos colocar $a=1$, obtendo assim a
solu\c{c}\~ao mais geral para a equa\c{c}\~ao da onda
\begin{equation}
\theta=\theta(t\pm \frac{x}{\alpha}) .
\end{equation}
Analogamente, para a equa\c{c}\~ao vetorial da onda temos,
\begin{equation}
\vec{\Omega}=\vec{\Omega}(t\pm \frac{x}{\beta}) .
\end{equation}
Cada uma dessas solu\c{c}\~oes descrevem duas ondas se propagando
em dire\c{c}\~oes opostas ao longo do eixo $x$ com velocidades
$\alpha$ e $\beta$.
Deste modo, as equa\c{c}\~oes descrevem ondas caracterizadas por suas
velocidades $\alpha$ e $\beta$. Essas velocidade s\~ao chamadas
velocidades de propaga\c{c}\~ao. Em um meio homog\^eneo e isotr\'opico,
n\'os temos duas ondas independentes. Como $\lambda$ e $\mu$ s\ao\
par\^ametros el\'asticos positivos, \'e facil reconhecer que $\alpha >
\beta$ Por\'em, as ondas n\~ao s\'o diferem por suas velocidades, mas
tamb\'em por suas propriedades. A onda mais r\'apida, que se propaga com
velocidade $\alpha$, \'e uma onda longitudinal, i.e., a sua
polariza\cao\ \'e na dire\cao\ de propaga\cao. Ela \'e descrita pela
dilata\c{c}\~ao $\theta$, a qual caracteriza mudan\c{c}as de volume. Por
essa caracter\istica, tamb\'em pode ser chamada de onda compressional.
Com frequ\^encia, ela \'e chamada na s\ismica\ de onda P, onde P vem de
prim\'aria. A onda mais lenta, que se propaga com velocidade $\beta$,
\'e uma onda transversal, i.e., a sua polariza\cao\ \'e perpendicular
\`a dire\cao\ de propaga\cao. Como ela n\ao\ gera varia\coes\ no volume,
tamb\'em pode ser chamada de onda cisalhante. Com frequ\^encia, ela \'e
chamada na s\ismica\ de onda S, onde S vem de secund\'aria, pois chegam
depois a onda P.

Aplicando as opera\c{c}\~oes de div e rot na equa\c{c}\~ao
elastodin\^amica para um s\'olido homog\^eneo e isotr\'opico,
n\'os encontramos que a equa\c{c}\~ao elastodin\^amica descreve
duas ondas separadas. Se aplicarmos os mesmos operadores para a
equa\c{c}\~ao elastodin\^amica para meios n\~ao-homog\^eneos e
isotr\'opicos, n\'os n\~ao temos sucesso na separa\c{c}\~ao em
duas ondas. Em meios n\~ao-homog\^eneos, ambas as ondas s\~ao
acopladas. Isto significa que as ondas em meios n\~ao-homog\^eneos
e isotr\'opicos n\~ao s\~ao puramente longitudinais, nem puramente
transversais. Por causa disso, em meios n\~ao-homog\^eneos, as
ondas P e ondas S s\~ao acopladas.

Da nossa defini\c{c}\~ao formal de fluido como um meio onde $\mu=0$,
segue imediatamente que, em fluidos, s\'o uma onda pode se propagar,
a onda P. A velocidade da onda S em fluidos \'e $\beta=0$.

As equa\c{c}\~oes da onda que n\'os encontramos n\~ao s\~ao a
solu\c{c}\~ao final para nosso problema. N\'os devemos resolve-las
para $\theta$ e $\vec{\Omega}$ e ent\~ao determinar o
vetor de deslocamento $\vec{u}$ de
\begin{equation}
\theta=\mbox{div}\vec{u}, \mbox{  }\mbox{  }\mbox{  }\mbox{  }\mbox{  }\vec{\Omega}=\mbox{rot}\vec{u} .
\end{equation}
Esse procedimento complicado pode ser evitado pelo uso do teorema
de Lam\'e. A principal id\'eia \'e expressar o vetor deslocamento
satisfazendo a equa\c{c}\~ao elastodin\^amica como dois termos
separados correspondendo as ondas P e S. De acordo com o teorema,
o vetor deslocamento pode ser espresso em termos dos pontenciais
de Helmholtz $\varphi$ e $\vec{\psi}$,
\begin{equation}
\vec{u}=\mbox{grad}\varphi+\mbox{rot}\vec{\psi}, \mbox{  }\mbox{  }\mbox{  }\mbox{  }\mbox{  } \mbox{div}\vec{\psi}=0 ,
\end{equation}
com $\varphi$ e $\vec{\psi}$ satisfazendo a equa\c{c}\~ao escalar
e vetorial da onda. Isso vale se a for\c{c}as sobre o corpo
$\vec{f}$  tamb\'em podem ser expressas em termos de potenciais de Helmholtz
\begin{equation}
\vec{f}=\mbox{grad}\Phi+\mbox{rot}\vec{\Psi}, \mbox{  }\mbox{  }\mbox{  }\mbox{  }\mbox{  } \mbox{div}\vec{\Psi}=0 ,
\end{equation}
e se certas condi\c{c}\~oes s\~ao satisfeitas (ver \textit{Aki \& Richards (1980)}).

Abaixo, mostramos que a condi\c{c}\~ao suficiente para a equa\c{c}\~ao
elastodin\^amica  valer \'e que $\varphi$ e $\vec{\psi}$
satisfa\c{c}am as correspondentes equa\c{c}\~oes da onda. Para tal,
inserimos $\vec{u}$ expresso em termos dos potenciais de Helmholtz
na equa\c{c}\~ao elastidin\^amica
\begin{equation}
(\lambda+\mu)u_{k,ki}+\mu u_{i,kk}+f_{i}=\rho u_{i,tt} 
\end{equation}
e obtemos
\begin{equation}
(\lambda+\mu)(\varphi_{,kki}+\epsilon_{klm}\psi_{m,lki})+\mu
(\varphi_{,ikk}+\epsilon_{ilm}\psi_{m,lkk})+(\Phi_{,i}+
\epsilon_{ilm}\Psi_{m,l})=\rho
(\varphi_{,itt}+\epsilon_{ilm}\psi_{m,ltt}) ,
\end{equation}
que pode ser reescrita como
\begin{equation}
[(\lambda+2\mu)\varphi_{,kk}+\Phi-\rho\varphi_{,tt}]_{,i}+
\epsilon_{ilm}[\mu\psi_{m,kk}+\Psi_{m}-\rho\psi_{m,tt}]_{,l}=0 .
\end{equation}
\'E suficiente, mas n\~ao necess\'ario que
\begin{eqnarray}
(\lambda+2\mu)\varphi_{,kk}+\Phi-\rho\varphi_{,tt} & = & 0, \\
\mu\psi_{m,kk}+\Psi_{m}-\rho\psi_{m,tt} & = & 0,
\end{eqnarray}
que na forma vetorial nos d\'a
\begin{eqnarray}
\alpha^{2}\Delta\varphi+\rho^{-1}\Phi & = &\frac{\partial^{2} \varphi
}{\partial t^{2}}, \\
\beta^{2}\Delta\vec{\psi}+\rho^{-1}\vec{\Psi} & = &\frac{\partial^{2}
\vec{\psi} }{\partial t^{2}}.
\end{eqnarray}
Essas s\~ao as equa\c{c}\~oes para os
potenciais de Helmholtz $\varphi$ (ondas P) e $\vec{\psi}$ (ondas S).
Logo, o vetor deslocamento pode simplesmente ser determinado pela
equa\c{c}\~ao
\begin{equation}
\vec{u}=\mbox{grad}\varphi+\mbox{rot}\vec{\psi}.
\end{equation}


%\begin{thebibliography}{bib}
%
%\bibitem{pse}{I. P\v{s}en\v{c}\'\i k - \emph{Introduction to seismic
%methods}, Lecture Notes, PPPG/UFBa, 1994.}
%\bibitem{aki}{Aki and Richards - \emph{Qualitative Seismology}, second
%edition, University Science Books, 2002.}
%
%\end{thebibliography}
%\end{document}
