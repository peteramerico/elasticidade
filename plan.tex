\section {Ondas Planas}
\label{plan}

Uma onda plana \'e uma fun\c{c}\~ao que satisfaz a equa\c{c}\~ao
da onda e que em um certo plano assume um valor constante,
acontecendo o mesmo para todos os planos paralelos \cite{moura}. Assim, para
determinarmos uma representa\c{c}\~ao matem\'atica desta onda
plana, vamos impor que ela precisa ter um valor constante ao longo
de um plano e que ela deve propagar satisfazendo a equa�\~ao da
onda ac\'ustica (\ref{eoadc}).

A equa\c{c}\~ao de um plano no espa\c{c}o de tr\^es dimens\~oes
\'e dada por:
\begin{equation}
   ax + by + lz = d \;,
   \label{eq_plano}
\end{equation}
sendo $a$, $b$ e $l$ as componentes do vetor normal unit\'ario
$\vec{n}$ e $d$ a dist\^ancia do plano \`a origem. Reescrevendo a
equa\c{c}\~ao (\ref{eq_plano}) usando o produto escalar, obtemos
\begin{equation}
   \vec{n} \cdot \vec{x} = d \;.
   \label{eq_plano_esc}
\end{equation}
Como a onda plana tem os mesmos valores ao longo de planos,
ent\~ao ela \'e uma fun\c{c}\~ao que depende destes planos, ou
seja,
\begin{equation}
   \Phi(\vec{x},t) = f[(\vec{n} \cdot \vec{x} - d)/d_0] \;,
   \label{eq_ondapl_1}
\end{equation}
onde $d_0$ \'e usada apenas para normalizar o argumento da
fun\c{c}\~ao. Este argumento normalizado \'e chamado de fase,
denotado por $\varphi$. Precisamos analisar a depend\^encia
temporal da propaga\c{c}\~ao do plano, ou seja, $d=d(t)$. Antes
disso, vamos interpretar fisicamente a constante $d_0$.

Vamos tomar como exemplo, a fun\c{c}\~ao $f$ sendo representada
pela fun\c{c}\~ao seno. Logo,
\begin{equation}
   \Phi(\vec{x},t) = \sin[(\vec{n} \cdot \vec{x} - d)/d_0] \;.
   \label{eq_ondapl_seno}
\end{equation}
A dist\^ancia entre dois planos onde $\Phi$ tem o mesmo valor
chama-se de comprimento de onda, denotado por $\lambda$. Sendo
$d_1$ e $d_2$ a dist\^ancia destes dois planos em rela\c{c}\~ao
\`a origem e usando que a fun\c{c}\~ao seno \'e peri\'odica com
per\'{\i}odo $2\pi$, escrevemos,
\begin{equation}
   \Phi_1(\vec{x},t) = \sin[(\vec{n} \cdot \vec{x} - d_1)/d_0] = \sin{\varphi_1}\;,
   \label{eq_ondapl_d1}
\end{equation}
\begin{equation}
   \Phi_2(\vec{x},t) = \sin[(\vec{n} \cdot \vec{x} - d_2)/d_0] = \sin{(\varphi_1+2\pi)}\;.
   \label{eq_ondapl_d2}
\end{equation}
Como $d_1-d_2=\lambda$ e pelas equa\c{c}\~oes
(\ref{eq_ondapl_d1}) e (\ref{eq_ondapl_d2}) temos
$d_1-d_2=2 \pi d_0$, ent\~ao encontramos a rela\c{c}\~ao entre
$d_0$ e $\lambda$:
\begin{equation}
   d_0=\frac{\lambda}{2\pi} \;\;\; \mathrm{ou} \;\;\;
   \frac{1}{d_0}=\frac{2\pi}{\lambda}=k \;,
   \label{rel_d0_lam}
\end{equation}
que \'e a defini\c{c}\~ao do n\'umero de onda, $k$. O n\'umero de
onda \'e igual ao n\'umero de onda por comprimento $2\pi$. No caso
da fun\c{c}\~ao seno, cujo comprimento de onda vale $2\pi$, o
n\'umero de onda $k$ vale 1. Reescrevendo a equa\c{c}\~ao
(\ref{eq_ondapl_seno}), temos,
\begin{equation}
   \Phi(\vec{x},t) = \sin[k(\vec{n} \cdot \vec{x} - d)]\;.
   \label{eq_ondapl_seno2}
\end{equation}

Esta forma de representa\c{c}\~ao \'e v\'alida para outras ondas.
Assim, a equa\c{c}\~ao (\ref{eq_ondapl_1}) pode ser escrita
como,
\begin{equation}
   \Phi(\vec{x},t) = f[k(\vec{n} \cdot \vec{x} - d)] \;.
   \label{eq_ondapl_2}
\end{equation}
Expandindo o termo no argumento da \'ultima equa\c{c}\~ao e
definindo o vetor do n\'umero de onda como sendo
$k\vec{n}=\vec{k}$, a equa\c{c}\~ao (\ref{eq_ondapl_2}) fica
\begin{equation}
   \Phi(\vec{x},t) = f(\vec{k} \cdot \vec{x} - kd) \;.
   \label{eq_ondapl_3}
\end{equation}
Para que a onda plana se propague com velocidade constante \'e
necess\'ario que a constante de fase $kd$ que est\'a no argumento
varie proporcionalmente ao tempo, ou seja, $kd=\omega t$, onde
$\omega$ \'e uma constante. Portanto, a equa\c{c}\~ao
(\ref{eq_ondapl_3}) que descreve a onda plana passa a ser
escrita como
\begin{equation}
   \Phi(\vec{x},t) = f(\vec{k} \cdot \vec{x} - \omega t) \;.
   \label{eq_ondapl_4}
\end{equation}

Observemos que, se tomarmos dois tempos distintos $t_1$ e $t_2$ de
tal modo que a onda plana tenha a mesma configura\c{c}\~ao e tenha
sofrido uma mudan\c{c}a de fase $2\pi$, ou seja, tenha passado um
per\'{\i}odo $T$ entre eles, podemos escrever (analogamente ao que
foi feito nas equa\c{c}\~oes (\ref{eq_ondapl_d1}) e
(\ref{eq_ondapl_d2}))
\begin{equation}
   \vec{k} \cdot \vec{x} - \omega t_1=\vec{k} \cdot \vec{x} - \omega t_2 + 2\pi \;,
   \label{omega1}
\end{equation}
de onde obtemos a frequ\^encia angular
\begin{equation}
   \omega = 2\pi / T \;.
   \label{omega2}
\end{equation}

Precisamos mostrar que a fun\c{c}\~ao da onda plana
(\ref{eq_ondapl_3}) satisfaz a equa\c{c}\~ao da onda ac\'ustica
(\ref{eoadc}). Vamos
denotar o argumento da fun\c{c}\~ao $\Phi$ por $\eta$. Logo,
\begin{equation}
   \eta = \vec{k} \cdot \vec{x} - \omega t =
   k_{x}x+k_{y}y+k_{z}z-wt \;,
   \label{eta}
\end{equation}
onde $k_x=ka$, $k_y=kb$ e $k_z=kl$. Agora, derivando a
fun\c{c}\~ao $\Phi$ e aplicando a regra da cadeia obtemos,
\begin{equation}
   \frac{\partial \Phi}{\partial x} = \frac{\partial f}{\partial \eta} \frac{\partial \eta}{\partial
   x}= \frac{\partial f}{\partial \eta} k_x\;,
   \label{dPhi/dx}
\end{equation}
\begin{equation}
   \frac{{\partial}^2 \Phi}{\partial {x}^2} = \frac{\partial}{\partial \eta}
   \left( \frac{\partial f}{\partial \eta} k_x\ \right) \frac{\partial \eta}{\partial   x}= \frac{\partial^2 f}{\partial \eta^2}
   k_x^2\;.
   \label{dPhi 2/dx 2}
\end{equation}
Analogamente, temos
$$\frac{{\partial}^2 \Phi}{\partial {y}^2}=\frac{\partial^2 f}{\partial \eta^2}
   k_y^2\;, \;\;\;\;\; \frac{{\partial}^2 \Phi}{\partial {z}^2}=\frac{\partial^2 f}{\partial \eta^2}
   k_z^2\;, \;\;\;\;\; \frac{{\partial}^2 \Phi}{\partial {t}^2}=\frac{\partial^2 f}{\partial \eta^2}
   \omega^2\;. $$
Substituindo estas derivadas na equa\c{c}\~ao da onda ac\'ustica
(\ref{eoadc}) obtemos
\begin{equation}
   \frac{{\partial}^2 f}{\partial {\eta}^2}(k_x^2+k_y^2+k_z^2- \omega^2/ c^2)=0\;.
   \label{eq_ondapl_eta}
\end{equation}
Logo, a fun\c{c}\~ao (\ref{eq_ondapl_4}) satisfaz a
equa\c{c}\~ao da onda desde que
\begin{equation}
   k_x^2+k_y^2+k_z^2 = \omega^2/ c^2 \;.
   \label{eq_ondapl_cond}
\end{equation}

Como o vetor normal $\vec{n}$ \'e unit\'ario, temos que
$k_x^2+k_y^2+k_z^2 = k^2\|n\|^2 = k^2$. Sendo assim, o n\'umero de
onda e a frequ\^encia angular devem satisfazer a rela\c{c}\~ao
\begin{equation}
   c = \omega/ k \;.
   \label{k_w_c}
\end{equation}
Portanto, a fun\c{c}\~ao dada em (\ref{eq_ondapl_4}) se propaga
como uma onda plana, j\'a que sua fase \'e constante ao longo de
um plano e ela satisfaz a equa\c{c}\~ao da onda, desde que sua
velocidade de propaga\c{c}\~ao seja dada por (\ref{k_w_c}). N\'os
chegamos neste resultado pela constru\c{c}\~ao da solu\c{c}\~ao,
mas podemos mostrar agora que a velocidade da fase da onda plana
\'e dada por $c=\omega / k \;$.

Tendo um plano conhecido de fase constante num instante $t_0$,
sabemos que um plano correspondente e paralelo ao original ser\'a
localizado num instante $t_1$. Ent\~ao, queremos saber onde se
encontra o plano no qual no instante $t_1$ a fase assume o mesmo
valor $\varphi$. Ou seja, queremos posi\c{c}\~oes $\vec{x}_0$ e
$\vec{x}_1$ tal que
\begin{equation}
   \vec{k} \cdot \vec{x}_0 - \omega t_0 = \vec{k} \cdot \vec{x}_1 - \omega t_1 = \varphi
   \;,
   \label{fase1}
\end{equation}
ou ainda,
\begin{equation}
   \vec{k} \cdot \Delta \vec{x}= \omega \Delta t
   \;,
   \label{fase2}
\end{equation}
onde $\Delta \vec{x}=\vec{x}_1-\vec{x}_0$ e $\Delta t=t_1 - t_0$.
Supondo que o vetor $\Delta \vec{x}$ esteja na mesma dire\c{c}\~ao
do vetor $\vec{k}$, que \'e a dire\c{c}\~ao de propaga\c{c}\~ao,
podemos escrever $\vec{k} \cdot \Delta \vec{x} = \|\vec{k}\|
\|\Delta \vec{x}\|= k \Delta x$. Substituindo esta rela\c{c}\~ao
na equa\c{c}\~ao (\ref{fase2}), obtemos
\begin{equation}
   c = \Delta x / \Delta t = \omega/k \;.
   \label{fase3}
\end{equation}
