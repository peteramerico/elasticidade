\subsection{Autovalores e autovetores de um tensor sim\'etrico de
ordem dois}

Assim como podemos usar um vetor $u_i$ para definir um plano
normal a este, ou seja, $u_i x_i = \pm 1$, podemos usar um tensor
sim\'etrico $T_{ij}$ de ordem dois para definir a quadr\'atica
(superf\'{\i}cie de segunda ordem com seu centro na origem do
sistema de coordenadas)
\begin{equation}
   T_{ij}x_i x_j = \pm 1 \;.
   \label{T_quadratica}
\end{equation}
Vamos mostrar que a equa\c{c}\~ao (\ref{T_quadratica}) \'e
invariante com respeito \`a mudan\c{c}a de coordenadas
$x_i^{'}=\alpha_{ij}x_j$. Ent\~ao,
\begin{equation}
   T_{ij}^{'} x_i^{'} x_j^{'} = \alpha_{im}\alpha_{jn}\alpha_{ik}\alpha_{jl} T_{mn} x_k x_l = \delta_{mk} \delta_{nl} T_{mn} x_k x_l = T_{mn} x_m x_n = T_{ij} x_i x_j \;.
   \label{T_q_invariante}
\end{equation}

Para uma escolha especial de coordenadas $x_i^{'}$, a
equa\c{c}\~ao das quadr\'aticas pode ser escrita como
\begin{equation}
   T_{11}^{'} x_1^{'2}+T_{22}^{'} x_2^{'2}+T_{33}^{'} x_3^{'2}= \pm 1 \;.
   \label{T_q_especial}
\end{equation}
Aqui, as coordenadas $x_i^{'}$ coincidiram com os { \it eixos
principais} das quadr\'aticas. Isto implica que a normal \`a
quadr\'atica, no ponto onde o eixo $x_i^{'}$ intercepta a
quadr\'atica, deve ser paralelo ao eixo $x_i^{'}$. Vamos denotar
os eixos principais pelos vetores unit\'arios $g_i^{(j)}$, isto
\'e, $\vec{g}^{(1)}={\vec{i}_1}^{\;'}$,
$\vec{g}^{(2)}=\vec{i}_2^{\;'}$, $\vec{g}^{(3)}=\vec{i}_3^{\;'}$ e
vamos tentar determin\'a-los. Como a normal da quadr\'atica \'e
paralela a $g_i^{(k)}$ nas dire\c{c}\~oes dos eixos principais,
temos que $g_i^{(k)}$ \'e paralelo ao gradiente da quadr\'atica em
$x_j(g_m^{(k)})$, ou seja,
\begin{equation}
   g_i^{(k)} \sim 2T_{ij} x_j (g_m^{(k)}) \;,
   \label{auto_1}
\end{equation}
onde $x_j(g_m^{(k)})$ denota as coordenadas do ponto de
interse\c{c}\~ao entre a quadr\'atica com o eixo $x_k^{'}$.
Denotando por $a$ o comprimento deste eixo principal, obtemos que
$x_j(g_m^{(k)}) = a g_j^{(k)}$, ou ainda,
\begin{equation}
   (T_{ij}-\lambda^{(k)} \delta_{ij}) g_j^{(k)}=0 \;.
   \label{auto_2}
\end{equation}
Esta equa\c{c}\~ao pode ser usada para determinar $g_j^{(k)}$ e
assim encontrar o eixo principal $x_k^{'}$. Isto representa um
problema de autovalor, onde $\lambda^{(k)}$ s\~ao os autovalores e
$g_i^{(k)}$ s\~ao os autovetores correspondentes. Este sistema
tensorial envolve um sistema de tr\^es equa\c{c}\~oes lineares
para $g_j^{(k)}$. A condi\c{c}\~ao para termos solu\c{c}\~ao nos
leva a
\begin{equation}
   \det(T_{ij}-\lambda^{(k)} \delta_{ij})=0 \;,
   \label{auto_3}
\end{equation}
conhecida como {\it equa\c{c}\~ao caracter\'{\i}stica}. Em nosso
caso, esta equa\c{c}\~ao \'e c\'ubica com tr\^es ra\'{\i}zes
$\lambda^{(k)}$, e a cada existe um autovetor $g_i^{(k)}$
correspondente.

Um tensor $T_{ij}$ sim\'etrico e real satisfaz a rela\c{c}\~ao
\begin{equation}
   T_{ij}=T_{ji}=T_{ji}^{*} \;,
   \label{T_sim}
\end{equation}
onde o s\'{\i}mbolo $*$ denota o complexo conjugado. Vamos mostrar
que o tensor $T_{ij}$ possui tr\^es autovalores reais.
Multiplicando a equa\c{c}\~ao de autovalores
\begin{equation}
   T_{ij} g_j^{(k)} = \lambda^{(k)} g_i^{(k)} \;
   \label{auto_4}
\end{equation}
por $g_j^{(k)*}$ e a equa\c{c}\~ao complexa conjugada
\begin{equation}
   T_{ij}^{*} g_j^{(k)*} = \lambda^{(k)*} g_i^{(k)*} \;
   \label{auto_5}
\end{equation}
por $g_j^{(k)}$, obtemos
\begin{equation}
   T_{ij} g_j^{(k)} g_i^{(k)*}= \lambda^{(k)} g_i^{(k)}
   g_i^{(k)*} \;, \;\;\; T_{ij}^{*} g_j^{(k)*} g_i^{(k)}= \lambda^{(k)*} g_i^{(k)*}
   g_i^{(k)}\;.
   \label{auto_6}
\end{equation}
Como $T_{ij}$ \'e real e sim\'etrico, os lados esquerdos destas
rela\c{c}\~oes s\~ao iguais. Logo,
\begin{equation}
   (\lambda^{(k)}-\lambda^{(k)*}) g_i^{(k)} g_i^{(k)*}= 0 \;.
   \label{auto_7}
\end{equation}
Mas como $g_i^{(k)} g_i^{(k)*} \neq 0$, esta equa\c{c}\~ao implica
que $\lambda^{(k)}=\lambda^{(k)*}$, e sendo assim $\lambda^{(k)}$
\'e real.

Agora vamos mostrar que a equa\c{c}\~ao (\ref{auto_4}) deve ser
satisfeita por um autovetor real. Considerando novamente o
complexo conjugado desta equa\c{c}\~ao e usando que $T_{ij}$ e
$\lambda$ s\~ao reais, podemos escrever
\begin{equation}
   T_{ij} g_j^{(k)*} = \lambda^{(k)} g_i^{(k)*} \;.
   \label{auto_4_conjugado}
\end{equation}
Somando as equa\c{c}\~oes (\ref{auto_4}) e
(\ref{auto_4_conjugado}) obtemos
\begin{equation}
   T_{ij} (g_j^{(k)}+g_j^{(k)*}) = \lambda^{(k)} (g_j^{(k)}+g_i^{(k)*})
   \;,
   \label{auto_4_conjugado2}
\end{equation}
o que conclui nossa desmonstra\c{c}\~ao, j\'a que
$g_j^{(k)}+g_j^{(k)*}$ \'e um vetor real.

Dizemos que o tensor $T_{ij}$ \'e definido positivo se $T_{ij}x_i
x_j >0$ para qualquer $x_i$. Uma importante propriedade deste tipo
de tensor \'e que seus autovalores s\~ao positivos. Para vermos
isso, vamos multiplicar a equa\c{c}\~ao (\ref{auto_4}) por
$g_i^{(k)}$. Logo,
\begin{equation}
   T_{ij} g_i^{(k)}g_j^{(k)} = \lambda^{(k)} g_i^{(k)}g_i^{(k)} \;.
   \label{auto_4_lambda}
\end{equation}
Como o lado esquerdo \'e positivo ($T_{ij}$ \'e definido positivo)
e $g_i^{(k)}g_i^{(k)}>0$, pois este \'e o quadrado da dist\^ancia
entre $g_i^{(k)}$ e a origem, conclu\'{\i}mos que $\lambda^{(k)}$
deve ser positivo.

Para finalizar, vamos mostrar que autovetores associados a
diferentes autovalores, s\~ao mutuamente ortogonais. Tratando dois
autovalores distintos $\lambda{(k)}$ e $\lambda{(l)}$, temos
\begin{equation}
   T_{ij} g_j^{(k)}- \lambda^{(k)} g_j^{(k)}=0 \;, \;\;\; T_{ij} g_j^{(l)}- \lambda^{(l)}
   g_j^{(l)}=0 \;.
   \label{auto_4_orto}
\end{equation}
Multiplicando a primeira equa\c{c}\~ao por $g_i^{(l)}$ e a segunda
por $g_i^{(k)}$ e em seguida subtraindo estas, obtemos
\begin{equation}
   (\lambda^{(k)}-\lambda^{(l)})g_i^{(k)}g_i^{(l)}=0 \;.
   \label{auto_4_orto1}
\end{equation}
Como $\lambda^{(k)} \neq \lambda^{(l)}$, ent\~ao $g_i^{(k)}$ e
$g_i^{(l)}$ s\~ao ortogonais. Se todos os tr\^es autovalores s\~ao
diferentes, ent\~ao eles determinam tr\^es autovetores mutuamente
ortogonais.

Neste caso, a superf\'{\i}cie correspondente ao tensor positivo
definido \'e elipsoidal com seus tr\^es eixos principais tendo
comprimentos diferentes. No caso de apenas dois autovalores serem
iguais, dizemos que trata-se do caso degenerado e n\~ao podemos
determinar de maneira \'unica os autovetores associados a esses
autovalores repetidos. A superf\'{\i}cie correspondente \'e um
elipsoidal rotacional. E para o caso dos tr\^es autovalores serem
iguais, quaisquer tr\^es vetores mutuamente ortogonais servem como
autovetores e sua superf\'{\i}cie correspondente \'e uma esfera.

Todas estas considera\c{c}\~oes podem ser extendidas a um tensor
$T_{ij}$ complexo. A rela\c{c}\~ao $g_i^{(1)}g_i^{(2)}=0$ deve
ent\~ao ser interpretada na forma
\begin{equation}
   \mathrm{\it Re}(g_i^{(1)}g_i^{(2)})=0\;, \;\;\; \mathrm{\it Im}(g_i^{(1)}g_i^{(2)})=0 \;.
   \label{auto_imag_real}
\end{equation}

Vamos agora encontrar os elementos $T_{ij}^{'}$ do tensor $T_{ij}$
no sistema de coordenadas cujos eixos s\~ao paralelos aos eixos
principais do tensor. Neste sistema de coordenadas $x_i^{'}$, os
autovetores $g_i^{(k)}$ podem ser escritos como
\begin{equation}
   g_i^{(1)}\equiv(1,0,0)\;, \;\; g_i^{(2)}\equiv(0,1,0)\;, \;\; g_i^{(3)}\equiv(0,0,1) \;.
   \label{auto_vet_x}
\end{equation}
Usando estes na equa\c{c}\~ao de autovalores
\begin{equation}
   T_{ij}^{'} g_j^{(k)} = \lambda^{(k)} g_i^{(k)} \;,
   \label{auto_esp_x}
\end{equation}
obtemos
\begin{equation} T_{ij}^{'} = \left(
\begin{array}{ccc}
   \lambda^{(1)} & 0 & 0 \\
   0 & \lambda^{(2)} & 0 \\
   0 & 0 & \lambda^{(3)}
   \label{auto_tens_diag}
\end{array} \right) \;.
\end{equation}
E assim, a equa\c{c}\~ao da superf\'{\i}cie do tensor
(equa\c{c}\~ao (\ref{T_q_especial})) toma a forma
\begin{equation}
\lambda^{(1)} x_1^{'2}+\lambda^{(2)} x_2^{'2}+\lambda^{(3)} x_3^{'2}= \pm 1 \;.
\label{T_q_especial_x}
\end{equation}
Podemos ver que $(\lambda^{(i)})^{-1/2}$ representa metade do
comprimento do {\it i}-\'esimo eixo principal do tensor $T_{ij}$.

   Vimos anteriormente que a equa\c{c}\~ao $T_{ij}x_i x_j = \pm 1$
\'e invariante com respeito \`a transforma\c{c}\~ao de coordenadas
e assim a superf\'i{\i}cie do tensor \'e invariante. Isto nos diz
que a orienta\c{c}\~ao do eixo principal e os autovalores tamb\'em
s\~ao invariantes. Por outro lado, isto implica que a
equa\c{c}\~ao caracter\'{\i}stica \'e tamb\'em invariante, ou
seja,
\begin{equation} \left|
\begin{array}{ccc}
   T_{11}-\lambda & T_{13} & T_{13} \\
   T_{12} & T_{22}-\lambda & T_{23} \\
   T_{13} & T_{23} & T_{33}-\lambda
\end{array} \right|=\left|
\begin{array}{ccc}
   \lambda^{(1)}-\lambda & 0 & 0 \\
   0 & \lambda^{(2)}-\lambda & 0 \\
   0 & 0 & \lambda^{(3)}-\lambda
\end{array} \right|=0\;.
   \label{auto_caract_inv}
\end{equation}
Ent\~ao, temos duas equa\c{c}\~oes caractar\'{\i}sticas, que nos
fornecem dois polin\^omios em $\lambda$. Pela igualdade destes
temos que seus coeficientes devem ser iguais. Assim, obtemos as
invariantes do tensor $T_{ij}$,
\begin{equation} \left\{
\begin{array}{l}
T_{ii}=\lambda^{(1)}+\lambda^{(2)}+\lambda^{(3)}\;, \\
\frac{1}{2}\varepsilon_{ijk}\varepsilon_{imn}T_{jm}T_{kn}=\lambda^{(1)}\lambda^{(2)}+\lambda^{(2)}\lambda^{(3)}+\lambda^{(1)}\lambda^{(3)}\;, \\
\frac{1}{6}\varepsilon_{ijk}\varepsilon_{lmn}T_{il}T_{jm}T_{kn}=\lambda^{(1)}\lambda^{(2)}\lambda^{(3)}\;.
\end{array} \right.
\end{equation}


\subsection{Diferencia\c{c}\~ao do Tensor}

Vamos ver a diferencia\c{c}\~ao dos tensores com respeito \`as
coordenadas temporal e espaciais. Pela pr\'opria defini\c{c}\~ao
de tensor, considerando um tensor de ordem $k$ temos que
\begin{equation}
T^{'}_{m_1,m_2,...,m_k}=\alpha_{m_1 j_1}\alpha_{m_2
j_2}...\alpha_{m_k j_k}T_{j_1,j_2,...,j_k} \;.
\label{tensor_mudanca}
\end{equation}
Tendo em mente que os cossenos $\alpha_{ij}$ n\~ao dependem do
tempo, podemos escrever a derivada temporal na forma
\begin{equation}
\frac{\partial}{\partial t}(T^{'}_{m_1,m_2,...,m_k})=\alpha_{m_1
j_1}\alpha_{m_2 j_2}...\alpha_{m_k j_k} \frac{\partial}{\partial
t}(T_{j_1,j_2,...,j_k}) \;.
\end{equation}
Ent\~ao, a derivada no tempo do tensor continua sendo um tensor, e
al\'em disso, de mesma ordem.

Tomemos o mesmo tensor e fa\c{c}amos a diferencia\c{c}\~ao com
respeito \`as coordenadas espaciais. Neste caso, os cossenos
$\alpha_{ij}$ tamb\'em n\~ao dependem da posi\c{c}\~ao. Da
rela\c{c}\~ao $x_i^{'}=\alpha_{ij}(x_j-x_{oj})$ obtemos as
derivadas parciais $\partial x_i^{'} / \partial x_j = \alpha_{ij}$
e $\partial x_j / \partial x_i^{'} = \alpha_{ji}$. Ent\~ao,
derivando (\ref{tensor_mudanca}) com respeito a uma coordenada
$x_l^{'}$, obtemos
\begin{equation}
\frac{\partial}{\partial
x_l^{'}}(T^{'}_{m_1,...,m_k})=\alpha_{li}\frac{\partial}{\partial
x_i}(\alpha_{m_1 j_1}...\alpha_{m_k
j_k}T_{j_1,...,j_k})=\alpha_{li}\alpha_{m_1 j_1}...\alpha_{m_k
j_k} \frac{\partial}{\partial x_i}(T_{j_1,...,j_k}) \;.
\end{equation}
De onde vemos que a diferencia\c{c}\~ao, com respeito a uma
coordenada espacial de um tensor, aumenta a ordem do tensor
resultante em uma unidade.
