\subsection{Diagonaliza\c{c}\~ao de um tensor sim\'etrico de segunda ordem}

Problemas que envolvem tensores de segunda ordem sim\'etricos (e.g. tensor de
deforma\c{c}\~ao e tens\~ao), tornam-se mais simples se estudados em um sistema
rotacionado onde apenas suas componentes diagonais s\~ao diferentes de zero.
Tal opera\c{c}\~ao de rota\c{c}\~ao \'e denominada diagonaliza\c{c}\~ao do
tensor, em refer\^encia \`a diagonaliza\c{c}\~ao de matrizes.

Portanto, queremos uma transforma\c{c}\~ao de coordenadas de $T_{ij}$, tal que
\begin{equation}
  T'_{ij}=\left\{\begin{aligned}
    &T'_{ij}&,\, \text{se } i=j& \\
     &0&,\, \text{se } i\neq j&
\end{aligned}\right.
\, .
  \label{eq:tdiag}
\end{equation}

Utilizando a equa\c{c}\~ao de rota\c{c}\~ao de coordenadas
\begin{equation}
  T'_{ij} = a_{ik}a_{jl}T_{kl} \,,
\end{equation}
nesta equa\c{c}\~ao, os tensores de rota\c{c}\~ao $a_{ik}$, $a_{jl}$ e o tensor
$T'_{ij}$ tem de ser determinados. Para isso, come\c{c}amos mutiplicando ambos os lados por $a_{jm}$

\begin{align}
  a_{jm}T'_{ij} &= a_{jm}a_{ik}a_{jl}T_{kl} \, ,\\
  a_{jm}T'_{ij} &= a_{ik}\delta_{lm}T_{kl} \, ,\\
  a_{jm}T'_{ij} &=a_{ik}T_{km} \,. \label{eq:eigvect1}
\end{align}

Como fizemos uma escolha particular sobre $T'_{ij}$ (equa\c{c}\~ao
\ref{eq:tdiag}) a \'ultima equa\c{c}\~ao acima se torna
\begin{align}
  a_{1m}T'_{11}  &=  a_{1k}T_{km}\, , \label{eq:eigvect2start}\\
  a_{2m}T'_{22}  &=  a_{2k}T_{km} \, ,\\
  a_{3m}T'_{33}  &=  a_{3k}T_{km} \,.\label{eq:eigvect2end}
\end{align}

Definindo
\begin{align}
  a_{1m} = u_{m}; \quad  &a_{1k} = u_{k} \, , \\
  a_{2m} = v_{m}; \quad  &a_{2k} = v_{k} \, , \\
  a_{3m} = w_{m}; \quad  &a_{3k} = w_{k} \, .
\end{align}

Substituindo as novas defini\c{c}\~oes nas equa\c{c}\~oes 
\ref{eq:eigvect2start} a \ref{eq:eigvect2end},
obtemos
\begin{align}
  u_{m}T'_{11}  &=  u_{k}T_{km}\, , \\
  v_{m}T'_{22}  &=  v_{k}T_{km} \, ,\\
  w_{m}T'_{33}  &=  w_{k}T_{km} \,.
\end{align}
Perceba que, as tr\^es equa\c{c}\~oes acima possuem a mesma forma, i.e.
\begin{equation}
  z_{k}T_{km} = \lambda z_{m};\, m=1,2,3 \, ,
  \label{eq:autovetores}
\end{equation}
onde $z$ pode ser $u$, $v$ ou $w$ e $\lambda$; $T'_{11}$, $T'_{22}$ ou $T'_{33}$.
Veja que podemos escrever $z_m=\delta_{km}z_k$, portanto, passando o lado direito para
esquerda e igualando a zero, obtemos,
\begin{equation}
  (T_{km}-\lambda\delta_{km})z_k=0;\, m=1,2,3\, .
\end{equation}

A equa\c{c}\~ao acima representa um sistema linear homog\^eneo de tr\^es
equa\c{c}\~oes em tr\^es inc\'ognitas ($z_{1}$, $z_{2}$, $z_{3}$). Como
procuramos uma solu\c{c}\~ao n\~ao trivial ($z_{k}=0$), temos que o determinante
do sistema tem de ser igual a zero:
\begin{equation}
  \det(T_{km}-\lambda\delta_{km}) = 0 \, .
\end{equation}

Escrevendo explicitamente a equa\c{c}\~ao acima, obtemos
\begin{equation}
  \det\left(
  \begin{bmatrix}
    T_{11}-\lambda & T_{12} & T_{13} \\
    T_{21} & T_{22}-\lambda & T_{23} \\
    T_{31} & T_{32} & T_{33} -\lambda
  \end{bmatrix}\right)
  = -\lambda^3 + A\lambda^2-B\lambda+C = 0
  \, ,
\end{equation}
onde,
\begin{equation}
  A =T_{ii}\, , \quad C=\det(T_{km}), \,
\end{equation}
e
\begin{equation}
  B = T_{11}T_{22} - T^2_{12} + T_{22}T_{33} -T^2_{23} + T_{11}T_{33} -
  T^2_{13}\, .
\end{equation}

Observe que temos um polin\^omio c\'ubico em $\lambda$, e sua solu\c{c}\~ao
possui tr\^es ra\'izes: $\lambda^{(1)}$, $\lambda^{(2)}$ e $\lambda^{(3)}$ que
s\~ao denominados autovalores. Onde, $u_k$, $v_k$ e $w_k$ s\~ao seus respectivos
autovetores, de acordo com a equa\c{c}\~ao \ref{eq:autovetores}.

Vale destacar a rela\c{c}\~ao entre os autovetores $u_{k}$, $v_{k}$, $w_{k}$ e os coeficientes 
da matriz de rota\c{c}\~ao $a_{ij}$
\begin{equation}
  \pmb{A} = \begin{bmatrix}
    u_k \\
    v_k \\
    w_k
  \end{bmatrix} = \begin{bmatrix}
    a_{11} & a_{12} &a_{13} \\
    a_{21} & a_{22} &a_{23} \\
    a_{31} & a_{32} &a_{33} 
  \end{bmatrix}
  \, .
\end{equation}
E que podemos obter o tensor $T'_{ij}$ por meio da opera\c{c}\~ao
\begin{align}
  \lambda z_m = z_kT_{km}  \, ,\\
  \lambda z_m {z_m}^T = z_kT_{km}{z_m}^T \, , \\
  \lambda = z_kT_{km}{z_m}^T \, ,
\end{align}
aqui usamos que a inversa da matriz de rota\c{c}\~ao \'e dada pela sua
transposta.

