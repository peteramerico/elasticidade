\section{Reflex\ao/transmiss\ao\ de ondas planas}

Consideremos uma interface plana $\Sigma$ com a origem de um eixo de
coordenadas cartesianas em um ponto arbitr\'ario de $\Sigma$. A escolha
dos eixos pode ser tamb\'em arbitr\'aria. Se $n_i$ for o vetor normal a
$\Sigma$, a equa\cao\ da interface pode ser escrita como $n_i x_i = 0$.
A interface divide o espa\co\ em dois semi-espa\co s 1 e 2 que assumimos
serem de welded contact. No semi-espa\co\ 1 consideramos uma onda plana
propagando em dire\cao\ a interface que chamamos de onda incidente. A
orienta\cao\ do vetor normal \'e tal que aponta para o semi-espa\co\
onde a onda inicidente se propaga. A fim de satisfazer as condi\coes\ de
contorno vistas anteriormente para $\Sigma$ introduzimos dois tipos de
onda: a onda refletida que se propaga no semi-espa\co\ 1 e a onda
transmitida que se propaga no semi-espa\co\ 2.

\subsection{Meio Ac\'ustico}

Consideremos dois semi-espa\co s de fluidos separados por uma interface
plana $\Sigma$. Denotamos a velocidade de fase e a densidade no
semi-espa\co\ 1 por $c_1$ e $\rho_1$ e no semi-espa\co\ 2 por $c_2$ e
$\rho_2$. Usamos o \ih ndice $r$ nos par\^ametros que correspondem \`a
onda refletida e o \ih ndice $t$ nos que correspondem \`a onda
transmitida. Ent\~ao temos
\begin{eqnarray}
p(x_m,t) &=& PF(t - p_k x_k),\\
p^r(x_m,t) &=& P^r F^r(t - p_k^r x_k - \varphi^r),\\
p^t(x_m,t) &=& P^t F^t(t - p_k^t x_k - \varphi^t),\\
v_i(x_m,t) &=& \rho_1^{-1}Pp_i F(t - p_k x_k),\\
v_i^r(x_m,t) &=& \rho_1^{-1}P^r p_i^r F^r(t - p_k^r x_k - \varphi^r),\\
v_i^t(x_m,t) &=& \rho_2^{-1}P^t p_i^t F^t(t - p_k^t x_k - \varphi^t),
\end{eqnarray}
Aqui $\varphi^r$ e $\varphi^t$ representam os poss\ih veis atrasos no
tempo da onda refletida e transmitida com rela\cao\ \`a onda incidente.
As duas condi\coes\ de contorno podem ser escritas como
\begin{eqnarray}
PF(t - p_k x_k) + P^r F^r(t - p_k^r x_k - \varphi^r) = P^t F^t(t - p_k^t x_k - \varphi^t),\\
\rho_1^{-1}Pp_i F(t - p_k x_k) + \rho_1^{-1}P^r p_i^r F^r(t - p_k^r x_k - \varphi^r) = \rho_2^{-1}P^t p_i^t F^t(t - p_k^t x_k - \varphi^t).
\end{eqnarray}

\subsubsection{Transforma\cao\ do vetor de vagarosidade atrav\'es da interface}

Pelo fato do nosso modelo ser invariante com respeito \`a transla\cao\
paralela da interface $\Sigma$, as condi\coes\ acima precisam ser as
mesmas em qualquer ponto de $\Sigma$ em qualquer momento, ou seja, n\ao\
podem depender de $x_k$ e $t$. Ent\ao\ em qualquer ponto de $\Sigma$ e
qualquer $t$ temos
\begin{eqnarray}
F(t - p_k x_k) = F^r(t - p_k^r x_k - \varphi^r) = F^t(t - p_k^t x_k - \varphi^t),\\
p_k x_k = p_k^r x_k + \varphi^r = p_k^t x_k + \varphi^t.
\end{eqnarray}
Se considerarmos dois pontos $x_{1i}$ e $x_{2i}$ de $\Sigma$ obtemos das
rela\coes\ \`acima
\begin{eqnarray}
p_k(x_{1k} - x_{2k}) = p_k^r(x_{1k} - x_{2k}) = p_k^t(x_{1k} - x_{2k})
\end{eqnarray}

Conclu\ih mos ent\ao\ que as componentes tangenciais do vetor de
vagarosidade das ondas incidente, refletida e transmitida s\ao\ iguais.
Ent\ao\
\begin{eqnarray}
p_k x_k = p_k^r x_k = p_k^t x_k \; \Rightarrow \; \varphi^r = \varphi^t = 0.
\end{eqnarray}
Usamos agora esta rela\cao\ para escrever $p_k^r$ e $p_k^t$ em fun\cao\
de $p_k$. Para tal, observamos que o vetor de vagarosidade pode ent\ao\
ser escrito como
\begin{eqnarray}
p_k = a_k + (p_m n_m)n_k, \; p_k^r = a_k^r + (p_m^r n_m)n_k, \; p_k^t = a_k^t + (p_m^t n_m)n_k,\\
\end{eqnarray}
onde
\begin{eqnarray}
a_k = a_k^r = a_k^t,
\end{eqnarray}
e $a_k,a_k^r,a_k^t$ s\ao\ componentes tangenciais vetoriais dos vetores
de vagarosidade $p_k,p_k^r,p_k^t$. Podemos ainda reescrever a
express\ao\ acima como
\begin{eqnarray}
p_k - (p_m n_m)n_k = p_k^r - (p_m^r n_m)n_k = p_k^t - (p_m^t n_m)n_k.
\end{eqnarray}
Para determinar $p_k^r$ e $p_k^t$ usamos a equa\cao\ eikonal
\begin{eqnarray}
p_i p_i = c^{-2},
\end{eqnarray}
que precisa ser satisfeita para ambos os lados da interface para as 3
ondas. Temos ent\ao\
\begin{eqnarray}
p_i p_i &=& a_i a_i + (p_m n_m)^2 = c_1^{-2},\\
p_i^r p_i^r &=& a_i a_i + (p_m^r n_m)^2 = c_1^{-2},\\
p_i^t p_i^t &=& a_i a_i + (p_m^t n_m)^2 = c_2^{-2}.
\end{eqnarray}
Da \'ultima equa\cao\ obtemos
\begin{eqnarray}
p_m^t n_m = -(c_2^{-2} - a_i a_i)^{1/2} = - [c_2^{-2} - c_1^{-2} + (p_m n_m)^2]^{1/2}.
\end{eqnarray}
Observemos que o sinal negativo se deve ao fato de que $n_i$ aponta
contra $p_i$ e $p_i^t$. Usamos agora a igualdade dos componentes
tangenciais do vetor de vagarosidade juntamente com a equa\cao\ acima e
obtemos
\begin{eqnarray}
p_k^t &=& p_k - {(p_m n_m)n_k + [c_2^{-2} - c_1^{-2} + (p_m n_m)^2]^{1/2}}n_k,\\
p_m^r n_m &=& +(c_1^{-2} - a_i a_i)^{1/2} = [(p_m n_m)^2]^{1/2} = - p_m n_m,\\
p_k^r &=& p_k -2(p_m n_m)n_k.
\end{eqnarray}

Podemos notar que a onda transmitida possui vetor de vagarosidade com
valores reais somente se
\begin{eqnarray}
c_2^{-2} - c_1^{-2} + (p_m n_m)^2 \geq 0.
\end{eqnarray}

Usamos agora o \ih ndice $g$ para os par\^ametros das ondas geradas
(refletida ou transmitida). Chamamos $\theta$ (\^angulo de incid\^encia)
o \^angulo agudo entre a normal $n_i$ e o vetor de vagarosidade $p_i$ da
onda incidente, e $\theta_g$ (\^angulo de reflex\ao/transmiss\ao) o
\^angulo agudo entre $n_i$ e $p_i^g$ das ondas geradas. Note que
$\theta_r = \theta$. Temos ent\ao\
\begin{eqnarray}
c_1 ^{-2}(1 - \cos^2 i) = c_g^{-2}(1 - \cos^2 \theta_g),
\end{eqnarray}
e como $\theta$ e $\theta_g$ s\ao\ \^angulos agudos temos
\begin{eqnarray}
\frac{\sin \theta}{c_1} = \frac{\sin \theta_g}{c_g}.
\end{eqnarray}
Esta igualdade \'e conhecida como Lei de Snell.

Usamos esses \^angulos para reescrever a condi\cao\ para que $p_k^t$
possua velores reais,
\begin{eqnarray}
c_2^{-2} - c_1^{-2} + c_1^{-2}\cos^2 \theta = c_2^{-2}\cos^2 \theta_t \geq 0\\
\end{eqnarray}
ou
\begin{eqnarray}
\frac{c_2}{c_1}\sin \theta = \sin \theta_t \leq 1.\;\; \mbox{(Lei de Snell)}
\end{eqnarray}
Esta desigualdade garante a exist\^encia de valores reais de $p_k^t$ e
do \^angulo real de transmiss\ao\ $\theta_t$. Para $\sin \theta_t = 1$,
que pode ser verdadeira somente se $c_1 \leq c_2$, temos
\begin{eqnarray}
\sin \theta^{*} = \frac{c_1}{c_2},
\end{eqnarray}
onde o \^angulo de incid\^encia $\theta^{*}$ \'e chamado de \^angulo
cr\ih tico. Os \^angulos de transmiss\ao\ correspondentes s\ao\
$\theta_t = 1/2\pi$, isto \'e, as ondas trasmitidas se propagam
paralelamente \`a $\Sigma$. Se $\theta < \theta^{*}$ temos o que
chamamos de incid\^encia subcr\ih tica e se $\theta > \theta^{*}$ temos
incid\^encia supercr\ih tica. Para
\begin{eqnarray}
c_2^{-2} - c_1^{-2} + c_1^{-2}\cos^2 \theta = c_2^{-2}\cos^2 \theta_t < 0,\\
\sin \theta > \frac{c_1}{c_2},
\end{eqnarray}
$p_k^t$ e $\theta_t$ assumem valores complexos, que ocorre no caso de
incid\^encia sobrecr\ih tica.

Podemos ent\ao\ escrever
$p_k^t$ como
\begin{eqnarray}
p_k^t = p_k^{tR} + ip_k^{tI},\; 
\end{eqnarray}
onde
\begin{eqnarray}
p_k^{tR} &=& p_k - (p_m n_m)n_k,\\
p_k^{tI} &=& \pm - [c_1^{-2} - c_2^{-2} - (p_m n_m)^2]^{1/2}n_k.
\end{eqnarray}

J\'a vimos que este vetor pertence a uma onda inomog\^enea. O sinal na
frente da raiz quadrada precisa ser escolhido de forma que a amplitude
da onda plana inomog\^enea transmitida decres\ca\ com o aumento da
dist\^ancia \`a interface. Por simplicidade escolhemos o sinal da onda
plana ac\'ustica inomog\^enea harm\^onica no tempo de press\ao. A onda
plana ac\'ustica inomog\^enea transmitida de press\ao\ tem a forma
\begin{eqnarray}
p^t(x_{m},t) = P^t\mbox{exp}(-wp_{m}^{tI}x_{m})\mbox{exp}[-iw(t-p_{m}^{tR}x_{m})].
\end{eqnarray} 
Esta express\ao\ descreve uma onda inomog\^enea aceit\'avel fisicamente
se
\begin{eqnarray}
p_{m}^{tI}x_{m} > 0
\end{eqnarray}
no semiespa\co\ 2, isto \'e, se
\begin{eqnarray}
\pm [c_1^{-2} - c_2^{-2} - (p_m n_m)^2]^{1/2}n_k x_k > 0. 
\end{eqnarray}
Como no semiespa\co\ 2,
\begin{eqnarray}
n_k x_k < 0,
\end{eqnarray}
$p_{m}^{tI}x_{m}$ \'e positivo se escolhemos o sinal negativo na frente
da raiz quadrada. Das express\oes\ para $p_{m}^{tR}$ vemos que o vetor
$p_{m}^{tR}$ \'e paralelo a $\Sigma$. Para a velocidade de fase desta
onda temos
\begin{eqnarray}
c_R^{-2} = p_i^{tR}p_i^{tR} = c_2^{-2} + p_i^{tI}p_i^{tI} = c_2^{-2} + c_1^{-2} - c_2^{-2} - c_1^{-2}\cos^2 \theta = \frac{\sin^2 \theta}{c_1^2}.
\end{eqnarray}
Portanto
\begin{eqnarray}
c_R = \frac{c_1}{\sin \theta}.
\end{eqnarray}

A velocidade de fase $c_R$ de uma onda inomog\^enea transmitida
corresponde \`a aparente velocidade da frente de fase da onda incidente
que se move ao longo de $\Sigma$. Esta velocidade diminiu com o
cresciemento de $\theta$, se $\theta = \theta^{*}$, $c_R = c_2$ e se
$\theta = 1/2\pi$ ent\ao\ $c_R = c_1$.

O vetor $p_i^{tI}$ \'e perpendicular \`a $\Sigma$, ou seja, a dire\cao\
de maior descrescimento da amplitude \'e na dire\cao\ oposta a $n_i$.
Para onda incidentes sobrecr\ih ticas, o decaimento aumenta com o
crescimento do \^angulo de incid\^encia. Para $i=i_{*}$, $p_i^{tI} = 0$,
isto \'e, n\ao\ h\'a decaimento e estamos trabalhando com ondas
transmitidas homog\^eneas propagando ao longo de $\Sigma$ com velocidade
de fase $c_R = c_2$. Para $i = 1/2\pi$, $p_i^{tI} = -(c_1^{-2} -
c_2^{-2})^{1/2}n_k$.

