
\section{Ondas Planas Inomog\^eneas}

At\'e este momento assumimos que o vetor de vagarosidade $p_{i}$ possuia
valores reais. Mas as equa\coes\ de movimento podem ser satisfeitas
tamb\'em para ondas planas com valores complexos no vetor $p_{i}$. Para
tais vetores podemos escrever
\begin{eqnarray}
p_{l} = p_{l}^{R}+ ip_{l}^{I}.
\end{eqnarray}

Consideremos uma onda plana ac\'ustica harm\^onica no tempo com vetor de
vagarosidade como acima, ent\ao\ temos
\begin{eqnarray}
p(x_{m},t) &=& P\mbox{exp}(-\omega p_{m}^{I}x_{m})\mbox{exp}[-i\omega (t-p_{m}^{R}x_{m})],\\
v_{i}(x_{m},t) &=& \rho^{-1}Pp_i\mbox{exp}(-\omega p_{m}^{I}x_m)\mbox{exp}[-i\omega (t-p_{m}^{R}x_{m})].
\end{eqnarray}

A amplitude da onda ac\'ustica com vetor de vagarosidade de valores
complexos n\ao\ \'e constante e decai exponencialmente na dire\cao\ de
$p_{m}^{I}$. A velocidade de decaimento depende do tamanho de
$p_{m}^{I}$ e da freq\"u\^encia circular $\omega$. Ent\ao\, juntamente com
o plano de fase constante $p_{m}^{R}x_m = const$, definimos o plano de
amplitudes constantes $p_{m}^{I}x_m = const$, ao longo do qual as
amplitudes n\ao\ variam. J\'a mostramos que o vetor de vagarosidade da
onda ac\'ustica satisfaz
\begin{eqnarray}
p_i p_i = c^{-2}, \; c=(\rho\kappa)^{-1/2}.
\end{eqnarray}
Para $p_i$ complexo temos
\begin{eqnarray}
p_i^R p_i^R - p_i^I p_i^I = Re(c^{-2}), \; 2p_i^R p_i^I = Im(c^{-2})
\end{eqnarray}
Agora exclu\ih mos o caso em que $Im(c^{-2}) \neq 0$ que corresponde a
meio com absor\cao. Ent\ao\ temos $Im(c^{-2}) = 0$, o que implica
\begin{eqnarray}
p_i^R p_i^I = 0,
\end{eqnarray}
isto \'e, as partes reais e imagin\'arias do vetor de vagarosidade s\ao\
mutuamente perpendiculares. Isto significa que o plano de fase constante
e o plano de amplitude constante s\ao\ perpendiculares. Tal onda \'e
chamada de onda inomog\^enea em contraste com a onda homog\^enea cujos
planos de fase constante e amplitude constante coincidem. As ondas
planas consideradas at\'e agora s\ao\ hom\^ogeneas. Considerando que h\'a
absor\cao, que o vetor de vagarosidade possui valores complexos junto
com velocidade de fase complexa, podemos obter um vetor de vagarosidade
com $p_i^R$ e $p_i^I$ paralelos.

Ondas planas inomog\^eneas n\ao\ podem existir em meios n\ao\ limitados
sem fontes. Elas podem vir durante o processo de reflex\ao/transmiss\ao\
de uma onda plana num plano de interface, e possuem papel importante na
expans\ao\ de ondas esf\'ericas em ondas planas.

Vimos que a amplitude da onda plana inomog\^enea harm\^onica no tempo varia ao longo da frente de fase.
A frente de fase $p_{m}^{R}x_m = const$ se move com a velocidade de fase $c_R$ que, de acordo com as 
equa\coes\ acima, satisfaz
\begin{eqnarray}
\frac{1}{c_R^2} = p_i^R p_i^R = \frac{1}{c^2} + p_i^I p_i^I > \frac{1}{c^2}
\end{eqnarray} 
Isto significa que a velocidade de fase de uma onda plana inomog\^enea \'e menor que a velocidade de fase 
de uma onda homog\^enea.
