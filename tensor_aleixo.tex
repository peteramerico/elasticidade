
\section{Tensores Cartesianos}

Quantidades f\'\i sicas podem ser matematicamente representadas
por tensores. As equa\c{c}\~oes que descrevem essas leis f\'isicas s\~ao, ent\~ao, equa\c{c}\~oes tensoriais.
Em nosso trabalho vamos nos concentrar em tensores no espa\c{c}o
euclidiano $3D$.

Alguns exemplos de tensores: \begin{equation}
  \begin{array}{l}
               \mbox{\emph{temperatura, densidade} - escalares - tensor de ordem zero}; \\
               \mbox{\emph{velocidade, deslocamento, for\c{c}a} -vetores - tensor de primeira ordem}; \\
               \mbox{\emph{tensor de tens\~ao, tensor de deforma\c{c}\~ao} - tensor de segunda ordem}; \\
               \mbox{\emph{tensor de par\^ametros el\'asticos} -  tensor de quarta ordem}. \\

\end{array}  
\end{equation}


\subsection{Transforma\c{c}\~ao de Coordenadas}

Considere dois pontos $A$ e $B$ no espa\c{c}o euclidiano tridimensional, os
pontos t\^em as respectivas coordenadas $[x_{1}^{A}, x_{2}^{A},
x_{3}^{A}]$, $[x_{1}^{B}, x_{2}^{B}, x_{3}^{B}]$. Definimos a
dist\^ancia entre estes pontos como,
\begin{equation}
d(A,B)=\left[(x_{i}^{B}-x_{i}^{A})(x_{i}^{B}-x_{i}^{A})\right]^{1/2}.
\end{equation}


Aqui estamos utilizando a conve\c{c}\~ao de Einstein, que diz,
quando um \'\i ndice se repete duas vezes num produto, t\^em-se
 impl\'\i cito uma soma sobre todos valores deste \'\i ndice.
 Note que o mesmo \'\i ndice pode aparecer no m\'aximo duas vezes na mesma equa\c{c}\~ao,
 ent\~ao, coisas do tipo $u_{i}v_{i}w_{i}$ n\~ao s\~ao permitidas.
Tamb\'em utilizaremos $u_{i}$ para representar uma componente gen\'erica do vetor $(u_{1},
u_{2}, u_{3})$.

Note que a fun\c{c}\~ao dist\^ancia satisfaz:
\begin{equation}
  \begin{array}{l}
            d(A,B) \mbox{ \'e real e n\~ao-negativa};\\
            d(A,A)=0;\\
             d(A,B)= d(B,A);\\
             d(A,B) \leq d(A,C)+ d(C,B).
\end{array}  
\end{equation}

Com essas defini\c{c}\~oes podemos considerar as
transfoma\c{c}\~oes de coordenadas. Consideremos dois sistemas de
 coordenadas $x_{i}$ e $x'_{i}$. A rela\c{c}\~ao entre os dois sistemas de coordenadas \'e
\begin{equation}
x'_{i}=\alpha_{ij}(x_{j}-x_{oj}),
\end{equation}
onde $x_{oj}$ \'e a coordenada da origem do sistema $x'_{i}$ nas coordenadas de
$x_{i}$ e os $\alpha_{ij}$ s\~ao os elementos da matriz de
transfoma\c{c}\~ao.

Lembremos da difini\c{c}\~ao do delta de Kronecker,
\begin{equation}
\delta_{ij}=\left \{ \begin{array}{l}
                0, \mbox{   }\mbox{   }\mbox{se } i \neq j,\\
                1, \mbox{   }\mbox{   }\mbox{se } i= j.\\
\end{array} \right.
\end{equation}
Pela conven\c{c}\~ao de Einstein temos que
$\delta_{ii}=\delta_{11}+\delta_{22}+\delta_{33}=3$.

Se fizermos $\alpha_{ij}=\delta_{ij}$, a tranforma\c{c}\~ao de
coordenadas ser\'a
\begin{equation}
x'_{i}=(x_{i}-x_{oi}).
\end{equation}
Esta transforma\c{c}\~ao descreve uma transla\c{c}\~ao paralela de um sistem de coordenada
com respeito ao outro (veja a figura \ref{fig:translacao-paralela}). 
\begin{figure}[!tb]
\begin{tikzpicture}
\draw[thick,->] (0,0) -- (4.5,0) node[anchor=north west] {$x_2$};
\draw[thick,->] (0,0) -- (0,4.5) node[anchor=south east] {$x_3$};
\draw[thick,->] (0,0) -- (-2.5,-2.5) node[anchor=north west] {$x_1$};
\node[below] at (0,0) {$O$};
\end{tikzpicture}
\begin{tikzpicture}
\begin{scope}[rotate=0]
\draw[thick,->] (5,0) -- (9.5,0) node[anchor=north west] {$x_2'$};
\draw[thick,->] (5,0) -- (5,4.5) node[anchor=south east] {$x_3'$};
\draw[thick,->] (5,0) -- (2.5,-2.5) node[anchor=north west] {$x_1'$};
\node[below right] at (5,0) {$O'$};
\end{scope}
\end{tikzpicture}
  \caption{Representa\c{c}\~ao de dois sistemas de coordenadas ortogonais
  distintos $\vec{e_j}$ e $\vec{e_k'}$.}
  \label{fig:translacao-paralela}
\end{figure}
Se fizermos  $x_{oj}=0$, a rela\c{c}\~ao de
transforma\c{c}\~ao ser\'a
\begin{equation}
x'_{i}=\alpha_{ij}x_{j},
\end{equation}
que descreve uma rota\c{c}\~ao em torno da origem do primeiro
sistema (veja a Figura \ref{fig:rotacaocoord}).

Agora iremos investigar o significado dos elementos da matriz de transforma\c{c}\~ao
$\alpha_{ij}$. Vamos introduzir o vetor raio $r_{i}$, partindo da
origem de um sistema de coordenadas at\'e um ponto $x_{i}$, isto
\'e, $x_{i}=r_{i}$. Sejam $\vec{e}_{1}$, $\vec{e}_{2}$ e
$\vec{e}_{3}$ tr\^es vetores base do espa\c{c}o cartesiano 
orientados ao longo dos eixos $x_{1}$, $x_{2}$,
$x_{3}$, respectivamente. Assim como $\vec{e_{1}}'$,
$\vec{e_{2}}'$, $\vec{e_{3}}'$ s\~ao vetores unit\'arios no
sistema $x'_{i}$.
Podemos expressar o vetor $r_{i}$ nos dois sistemas de
coordenadas
\begin{equation}
  \vec{r}=x_{j}\vec{e_{j}}=x_{1}\vec{e}_{1}+x_{2}\vec{e}_{2}+x_{3}\vec{e}_{3}=x'_{1}\vec{e_{1}}'+
x'_{2}\vec{e_{2}}'+x'_{3}\vec{e_{3}}'\, .
\end{equation}
Se substituimos, no lado direito, a rela\c{c}\~ao de transforma\c{c}\~ao
\begin{equation}
x'_{i}=\alpha_{ij}x_{j},
\end{equation}
teremos
\begin{equation}
\vec{r}=x_{1}\vec{e}_{1}+x_{2}\vec{e}_{2}+
x_{3}\vec{e}_{3}=\alpha_{1j}x_{j}\vec{e_{1}}'+
\alpha_{2j}x_{j}\vec{e_{2}}'+\alpha_{3j}x_{j}\vec{e_{3}}'.
\end{equation}
Ao tomarmos o produto escalar entre a equa\c{c}\~ao acima e $\vec{e_{k}}'$, teremos
\begin{align}
%  x_j\vec{e_{j}}\vec{e_{k}} &= \alpha_{ij}x_j\vec{e_{i}}'\vec{e_{k}}' \, ,
%  \label{eq:doteiek} \\
%  x_j\vec{e_{j}}\vec{e_{k}}' &= \alpha_{ij}x_j\delta_{ik} \, .
   x_j\vec{e_{j}} \vec{e_{k}}' &= \alpha_{ij}x_j\vec{e_{i}}'\vec{e_{k}}' \, ,
  \label{eq:doteiek} \\
  x_j\vec{e_{j}}\vec{e_{k}}' &= \alpha_{ij}x_j\delta_{ik} \, .
\end{align}
Note que na equa\c{c}\~ao \ref{eq:doteiek}, $\,\vec{e_{i}}'$ e $\vec{e_{k}}'$ s\~ao vetores 
unit\'arios e ortogonais. Portanto, o
produto escalar entre eles \'e diferente de zero e igual a 1, se e somente se,
$i=k$. Que \'e exatamente a defini\c{c}\~ao do delta de Kronecker.
Assim, resolvendo o delta de Kronecker e simplificando os termos $x_j$.
Obtemos uma express\~ao para os termos de $\alpha_{kj}$
\begin{equation}
  \alpha_{kj}=\vec{e_{j}}\vec{e_{k}}'=\vec{e}_{j} \cdot \vec{e_{k}}' = 
  \lvert e_{j}\rvert\lvert e'_{k}\rvert \cos{\theta} \, .
\end{equation}

\begin{figure}[!tb]
  \centering
\begin{tikzpicture}
\begin{scope}
  \draw[thick,->] (0,0) -- (4.5,0) node[anchor=north west] {$x_2$};
  \draw[thick,->] (0,0) -- (0,4.5) node[anchor=south east] {$x_3$};
  \draw[thick,->] (0,0) -- (-2.5,-2.5) node[anchor=north east] {$x_1$};
  \coordinate (e2) at (2,0);
  \coordinate (e3) at (0,2);
  \coordinate (e1) at (-1.3,-1.3);
\end{scope}
  \begin{scope}[rotate=20]
  \draw[thick,->] (0,0) -- (4.5,0) node[anchor=north west] {$x'_2$};
  \draw[thick,->] (0,0) -- (0,4.5) node[anchor=south east] {$x'_3$};
  \draw[thick,->] (0,0) -- (-2.5,-2.5) node[anchor=north west] {$x'_1$};
  \coordinate (e2rot) at (2,0);
  \coordinate (e3rot) at (0,2);
  \coordinate (e1rot) at (-1.3,-1.3);
\end{scope}
\node[below right] at (0,0) {$O = O'$};
  \draw[bend right=30,thick, ->] (e1) to (e1rot); 
  \node at ($(e1)-(-0.05,0.5)$) {$\alpha_{11}$};
  \draw[bend right=30,thick, ->] (e2)  to node[right]{$\alpha_{22}$} (e2rot);
  \draw[bend right=30,thick, ->] (e3)  to node[above]{$\alpha_{33}$} (e3rot);
  \draw[bend right=50,thick, ->,red] ($(e1)+(0,0)$)  to node[below]{$\alpha_{21}$}
  ($(e2rot)-(0,0)$);
  \draw[bend left=25,thick, <-,blue] ($(e1rot)$)  to
  node[left]{$\alpha_{13}$}
  ($(e3)-(0,0)$);
  \draw[bend right=25,thick, ->,OliveGreen] ($(e2)$)  to
  node[above]{$\alpha_{32}$} ($(e3rot)-(0,0)$);
\end{tikzpicture}
  \caption{Ilustra\c{c}\~ao de uma rota\c{c}\~ao dos eixos de coordenadas de base
  $\vec{e_j}$ em um novo sistema de coordenadas ortogonais $\vec{e_k}'$. Note
  que os cossenos dos \^angulos entre os eixos determinam os termos da matriz
  $\alpha_{kj}$.}
  \label{fig:rotacaocoord}
\end{figure}

Note que os elementos de
$\alpha_{ij}$ s\~ao cossenos, denominados cossenos direcionais, onde $\theta$
representa o \^angulo entre os vetores $\vec{e_{j}}$ e $\vec{e_{k}}'$ (veja a
Figura \ref{fig:rotacaocoord}).

Considerando a transforma\c{c}\~ao inversa a
\begin{equation}
x'_{i}=\alpha_{ij}(x_{j}-x_{oj}).
  \label{eq:rotacao}
\end{equation}
Iremos mostrar que para a matriz $\alpha_{ij}$, existe uma inversa,
$(\alpha^{-1})_{ij}$, tal que
\begin{equation}
\alpha_{ij}(\alpha^{-1})_{jk}=\delta_{ik}, \quad (\alpha^{-1})_{jk}\alpha_{kl}=\delta_{jl}.
\label{eq:inversakronecker}
\end{equation}
Aqui vale relembrar que o produto entre uma matriz qualquer $\pmb{A}$ e sua inversa
$\pmb{A}^{-1}$ \'e igual a matriz identidade $\pmb{\cal{I}}$. E que tamb\'em
podemos definir o delta de Kronecker na forma matricial. Portanto,
\begin{equation}
  \delta_{ij} =
  \begin{bmatrix}
    1 & 0 & 0 \\
    0 & 1 & 0 \\
    0 & 0 & 1 
  \end{bmatrix}
  \equiv \pmb{\cal{I}}
  \, .
\end{equation}
Desta maneira, podemos escrever uma forma de obter a transforma\c{c}\~ao que
relaciona $x_j$ e $x_i'$. Multiplicando a equa\c{c}\~ao \ref{eq:rotacao} por
$(\alpha^{-1})_{ij}$, temos
\begin{equation}
x_{i}=x_{oi}+(\alpha^{-1})_{ij}x'_{j}.
\end{equation}

Como a dist\^ancia entre $A$ e $B$ deve ser independente do
sistema de coordenadas utilizado, temos
\begin{equation}
 (x_{i}^{B}-x_{i}^{A})(x_{i}^{B}-x_{i}^{A})=(x'_i{}^{B}-x'_i{}^{A})(x'_i{}^{B}-x'_i{}^{A}),
\end{equation}
reescrevendo os termos do lado direito com a rela\c{c}\~ao da equa\c{c}\~ao \ref{eq:rotacao},
temos
\begin{equation}
 (x_{i}^{B}-x_{i}^{A})(x_{i}^{B}-x_{i}^{A})=\alpha_{ij}\alpha_{il}(x_{j}^{B}-x_{j}^{A})(x_{l}^{B}-x_{l}^{A}).
\end{equation}

Veja que, para mantermos a consist\^encia das equa\c{c}\~oes, 
queremos apenas o resultado de quando $i=j$ e $i=l$. Desta maneira
a igualdade \'e mantida se
\begin{equation}
  \alpha_{ij}\alpha_{il}=\delta_{ij}\delta_{il} = \delta_{jl} \, .
\end{equation}
onde \'e possivel provar que $\delta_{ij}\delta_{il} = \delta_{jl}$.

Agora supondo que exista a inversa, podemos relacionar o resultado obtido acima
com o resultado da equa\c{c}\~ao \ref{eq:inversakronecker}
\begin{equation}
(\alpha^{-1})_{ji}\alpha_{il}=\delta_{jl}\, ,
\end{equation}
e portanto,
\begin{equation}
(\alpha^{-1})_{ji}=\alpha_{ij}\, ,
\end{equation}
trocando a ordem dos \'indices chegamos a forma final
\begin{equation}
(\alpha^{-1})_{ij}=\alpha_{ji}\, .
\end{equation}

Isso significa que a matriz inversa de $\alpha_{ij}$ \'e obtida
transpondo a matriz $\alpha_{ij}$, por esta propridade vemos que a
matriz em discuss\~ao \'e ortogonal. Pela ortogonalidade, temos
que
\begin{equation}
|det(\alpha_{ij})|=1\, .
\end{equation}

\subsection{Defini\c{c}\~ao e Propriedades dos Tensores Cartesianos}

Nesta se\c{c}\~ao vamos definir um tensor cartesiano e algumas de
suas propriedades.

Para cada ponto do espa\c{c}o, o tensor cartesiano \'e uma
sequ\^encia
finita de n\'umeros os quais correspondem, de maneira \'unica, as coordenadas
cartesianas do ponto. A sequ\^encia representa um tensor se
satisfaz as seguintes  condi\c{c}\~oes:

\begin{enumerate}
  \item A sequ\^encia cont\'em $n^{k}$ n\'umeros onde $n$ representa a
dimens\~ao do espa\c{c}o e $k$ \'e um inteiro n\~ao-negativo
chamado ordem:
\begin{equation}
T_{m_{1},m_{2},...,m_{k}}, \quad m_{i}=1,2,3,\dots, k.
\end{equation}
    A representa\c{c}\~ao matricial do tensor $T$ se d\'a na forma de uma matriz
    $\pmb{\cal{T}}_{n\times n\times ...\times n}$ de $k$ dimens\~oes. Por exemplo, um tensor de
    ordem 2 no espa\c{c}o tridimensional, $T_{ij}$ tem a representa\c{c}\~ao matricial:
    \begin{equation}
      T_{ij} =
      \begin{bmatrix}
        \tau_{11} & \tau_{12} &\tau_{13} \\
        \tau_{21} & \tau_{22} &\tau_{23} \\
        \tau_{31} & \tau_{32} &\tau_{33} 
      \end{bmatrix}
      \, .
    \end{equation}

\item Um elemento do tensor no sistema de coordenadas $x'_{i}$ \'e
associado ao tensor no sistema de coordenadas $x_{i}$ pela
rela\c{c}\~ao:
\begin{equation}
T'_{p_{1},p_{2},\dots,p_{k}}=
\alpha_{p_{1}m_{1}}\alpha_{p_{2}m_{2}}\dots\alpha_{p_{k}m_{k}}T_{m_{1},m_{2},\dots,m_{k}}.
\end{equation}
\end{enumerate}


Um tensor de ordem zero, i.e $k=0$, \'e a denomina\c{c}\~ao tensorial de um
escalar. J\'a um tensor de ordem $k=1$ \'e a representa\c{c}\~ao tensorial de um
vetor, onde a equa\c{c}\~ao \ref{eq:rotacao} permanece v\'alida.

A rota\c{c}\~ao de um tensor de segunda ordem \'e dada por
\begin{equation}
T'_{ij}=\alpha_{im}\alpha_{jn}T_{mn} \, .
\end{equation}

Para tensores de mesma ordem somente, adi\c{c}\~ao e subtra\c{c}\~ao
correspondem a
\begin{equation}
A_{ij} \pm B_{ij}=C_{ij}.
\end{equation}


Em contraste com adi\c{c}\~ao e subtra\c{c}\~ao, na multiplica\c{c}\~ao, n\~ao
necessitamos que os tensores tenham a
mesma ordem. Se todos os \'\i ndices s\~ao distintos a
multiplica\c{c}\~ao soma os \'\i ndices, como por exemplo,
\begin{equation}
A_{ij}B_{klm}=C_{ijklm}.
\end{equation}

J\'a a multiplica\c{c}\~ao de tensores com \'\i ndices iguais reduz
a ordem do tensor resultante por $2$, como em
\begin{equation}
A_{ij}B_{j}=C_{i}.
\end{equation}

A redu\c{c}\~ao da ordem \'e dita contra\c{c}\~ao. Um t\'\i pico
exemplo \'e o produto escalar de dois vetores.

Posteriormente iremos trabalhar com com tensores sim\'etricos e
anti-sim\'etricos. Entendemos um  tensor sim\'etrico com a seguinte
propriedade
\begin{equation}
T_{m_{1},m_{2},...,m_{k}}=T_{m_{2},m_{1},...,m_{k}},
\end{equation}
esse tensor \'e sim\'etrico nos \'\i ndices $m_{1}$ e
$m_{2}$. Entendemos um  tensor anti-sim\'etrico com a seguinte
propriedade
\begin{equation}
T_{m_{1},m_{2},...,m_{k}}=-T_{m_{k},m_{2},...,m_{1}},
\end{equation}
esse tensor \'e anti-sim\'etrico nos \'\i ndices $m_{1}$ e $m_{k}$.

Um tensor de
ordem $2$, pode ser escrito da seguinte forma (soma de um tensor
sim\'etrico com um anti-sim\'etrico):
\begin{equation}
B_{ij}=\frac{1}{2}(B_{ij}+B_{ji})+\frac{1}{2}(B_{ij}-B_{ji})=S_{ij}+A_{ij}.
\end{equation}

Simetria e anti-simetria s\~ao preservadas ap\'os uma
transforma\c{c}\~ao de coordenadas.
 Se $T_{mnl}=T_{nml}$, temos
\begin{equation}
T'_{ijk}=\alpha_{im}\alpha_{jn}\alpha_{kl}T_{mnl}=\alpha_{im}\alpha_{jn}\alpha_{kl}T_{nml}=
\alpha_{jn}\alpha_{im}\alpha_{kl}T_{nml}=T'_{jik}.
\end{equation}
 
O racioc\'\i nio \'e o mesmo para tensores anti-sim\'etricos.

\subsection{Tensores Isotr\'opicos (especiais)}


Um tensor \'e dito isotr\'opico se ap\'os uma transforma\c{c}\~ao
de coordenadas o tensor se mant\'em inalterado.
Por exemplo o delta de Kronecker \'e um tensor de ordem $2$
isotr\'opico, pois
\begin{equation}
\delta'_{ij}=\alpha_{im}\alpha_{jn}\delta_{mn}=\alpha_{im}\alpha_{jm}=\delta_{ij}.
\end{equation}

Um outro tensor isotr\'opico \'e o s\'\i mbolo de Levi-Civita,
$\epsilon_{ijk}$, que \'e um tensor de ordem $3$, definido como:
\begin{equation}
\epsilon_{ijk}=\left\{ \begin{array}{l}
           1,\mbox{   }\mbox{   }\mbox{   }\mbox{se $i,j,k$ formam uma
permuta\c{c}\~ao par de $1,2,3$};\\
          -1,\mbox{   }\mbox{   }\mbox{   }\mbox{se $i,j,k$ formam uma
permuta\c{c}\~ao \'\i mpar de $1,2,3$};\\
           0,\mbox{   }\mbox{   }\mbox{   }\mbox{se dois \'\i ndices s\~ao repetidos}.\\
\end{array} \right.
\end{equation}

Pode-se provar a seguinte identidade
\begin{equation}
\epsilon_{ijk}\epsilon_{lmn}= \begin{array}{lllll}
                \vline & \delta_{il} & \delta_{im} & \delta_{in} & \vline\\
                \vline &\delta_{jl} & \delta_{jm} & \delta_{jn} &\vline\\
                 \vline &\delta_{kl} & \delta_{km} & \delta_{kn} &\vline\\
\end{array}.
\end{equation}

E, portanto,
\begin{equation}
\epsilon_{ijk}\epsilon_{imn}= \begin{array}{lllll}
                \vline & 3 & \delta_{im} & \delta_{in}&\vline \\
                \vline & \delta_{ji} & \delta_{jm} & \delta_{jn}& \vline\\
               \vline & \delta_{ki} & \delta_{km} & \delta_{kn}& \vline\\
\end{array}=\delta_{jm}\delta_{kn}-\delta_{jn}\delta_{km},
\end{equation}
que \'e uma identidade importante. Esta identidade, tamb\'em \'e chamada de
regra de "E$\delta$", de
f\'acil memoriza\c{c}\~ao. Basta lembrar que em um produto de dois s\'imbolos de
Levi-Civita que compartilham um \'indice e possuem os outros distintos, o resultado
desse produto \'e uma subtra\c{c}\~ao entre dois produtos de duas deltas que tem
os \'indices da forma: "\textcolor{red}{Primeiro Primeiro} $*$ 
\textcolor{blue}{Segundo Segundo} - \textcolor{red}{Primeiro}
\textcolor{blue}{Segundo} $*$
\textcolor{blue}{Segundo} \textcolor{red}{Primeiro}", i.e:
\begin{equation}
  \epsilon_{\textcolor{gray}{i}
  \textcolor{red}{j}
  \textcolor{blue}{k}}
  \epsilon_{\textcolor{gray}{i}
  \textcolor{red}{m}
  \textcolor{blue}{n}}
 =
  \delta_{\textcolor{red}{j}
  \textcolor{red}{m}}
  \delta_{\textcolor{blue}{k}
  \textcolor{blue}{n}}
  -
  \delta_{\textcolor{red}{j}
  \textcolor{blue}{n}}
  \delta_{\textcolor{blue}{k}
  \textcolor{red}{m}}
   \,.
\end{equation}

A partir da regra de "E$\delta$", tomando $m=j$ e lembrando que $\delta_{jj}=3$, temos que,
\begin{align}
  \epsilon_{ijk}\epsilon_{ijn}&=\delta_{jj}\delta_{kn}-\delta_{jn}\delta_{kj}\\
  \epsilon_{ijk}\epsilon_{ijn}&=3\delta_{kn}-\delta_{kn} = 2\delta_{kn} \, ,
\end{align}
e de maneira similar, tomando $n=k$,
\begin{equation}
\epsilon_{ijk}\epsilon_{ijk}=6.
\end{equation}


Para uma matriz arbitr\'aria $a_{ij}$, usando as identidades  para
$\epsilon_{ijk}\epsilon_{lmn}$, podemos escrever
\begin{equation}
\epsilon_{ijk}a_{1i}a_{2j}a_{3k}=\epsilon_{ijk}\epsilon_{123}a_{1i}a_{2j}a_{3k}=
\begin{array}{lllll}
                \vline&\delta_{i1} & \delta_{i2} & \delta_{i3}&\vline \\
                \vline&\delta_{j1} & \delta_{j2} & \delta_{j3}&\vline \\
                \vline&\delta_{k1} & \delta_{k2} & \delta_{k3}& \vline\\
\end{array}a_{1i}a_{2j}a_{3k}=
\end{equation}
 \begin{equation}
 =\begin{array}{lllll}
                \vline&a_{11} & a_{12} & a_{13}& \vline\\
                \vline&a_{21} & a_{22} & a_{23}&\vline \\
                \vline&a_{31} & a_{32} & a_{33}&\vline \\
\end{array}=det(a_{ij}).
\end{equation}

A rela\c{c}\~ao acima pode ser generalizada
\begin{equation}
\epsilon_{ijk}a_{mi}a_{nj}a_{lk}=det(a_{pq})\epsilon_{mnl}.
\end{equation}

Com a rela\c{c}\~ao anterior podemos demonstrar que  o s\'\i mbolo
de Levi-Civita \'e um tensor isotr\'opico de ordem $3$. De fato,
\begin{equation}
\epsilon'_{ijk}=\alpha_{im}\alpha_{jn}\alpha_{kl}\epsilon_{mnl}=det(\alpha_{pq})\epsilon_{ijk}=\epsilon_{ijk}.
\end{equation}

E tamb\'em temos a seguinte rela\c{c}\~ao
\begin{equation}
det(a_{pq})=\frac{1}{6}\epsilon_{mnl}\epsilon_{ijk}a_{mi}a_{nj}a_{lk}.
\end{equation}

%\begin{thebibliography}{bib}
%\bibitem{pse}{I. P\v{s}en\v{c}\'\i k - \emph{Introduction to seismic methods}, Lecture Notes, PPPG/UFBa, 1994.}
%\end{thebibliography}

