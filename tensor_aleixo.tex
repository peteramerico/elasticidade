
\section{Tensores Cartesianos}

Quantidades f\'\i sicas podem ser matematicamente representadas
por tensores. As equa\c{c}\~oes que descrevem essas leis f\'isicas s\~ao, ent\~ao, equa\c{c}\~oes tensoriais.
Em nosso trabalho vamos nos concentrar em tensores no espa\c{c}o
euclidiano $3D$.

Alguns exemplos de tensores: \[  \begin{array}{l}
               \mbox{\emph{temperatura, densidade} - escalares - tensor de ordem zero}; \\
               \mbox{\emph{velocidade, deslocamento, for\c{c}a} -vetores - tensor de primeira ordem}; \\
               \mbox{\emph{tensor de tens\~ao, tensor de deforma\c{c}\~ao} - tensor de segunda ordem}; \\
               \mbox{\emph{tensor de par\^ametros el\'asticos} -  tensor de quarta ordem}. \\

\end{array}  \]

\subsection{Transforma\c{c}\~ao de Coordenadas}

Considere dois pontos $A$ e $B$ no espa\c{c}o euclidiano $3D$, os
pontos t\^em as respectivas coordenadas $[x_{1}^{A}, x_{2}^{A},
x_{3}^{A}]$, $[x_{1}^{B}, x_{2}^{B}, x_{3}^{B}]$. Definimos a
dist\^ancia entre estes pontos como,
\[d(A,B)=\left[(x_{i}^{B}-x_{i}^{A})(x_{i}^{B}-x_{i}^{A})\right]^{1/2}.\]

Aqui estamos utilizando a conve\c{c}\~ao de Einstein, que diz,
quando um \'\i ndice se repete duas vezes num produto, t\^em-se
 impl\'\i cito uma soma sobre todos valores deste \'\i ndice.
 Note que o mesmo \'\i ndice pode aparecer no m\'aximo duas vezes na mesma equa\c{c}\~ao,
 ent\~ao, coisas do tipo $u_{i}v_{i}w_{i}$ n\~ao s\~ao permitidas.
Tamb\'em utilizaremos $u_{i}$ para representar uma componente gen\'erica do vetor $(u_{1},
u_{2}, u_{3})$.

Note que a fun\c{c}\~ao dist\^ancia satisfaz:
\[  \begin{array}{l}
            d(A,B) \mbox{ \'e real e n\~ao-negativa};\\
            d(A,A)=0;\\
             d(A,B)= d(B,A);\\
             d(A,B) \leq d(A,C)+ d(C,B).
\end{array}  \]


Com essas defini\c{c}\~oes podemos considerar as
transfoma\c{c}\~oes de coordenadas. Consideremos dois sistemas de
 coordenadas $x_{i}$ e $x'_{i}$. A rela\c{c}\~ao entre os dois sistemas de coordenadas \'e
\[x'_{i}=\alpha_{ij}(x_{j}-x_{oj}),\]
onde $x_{oj}$ \'e a coordenada da origem do sistema $x'_{i}$ nas coordenadas de
$x_{i}$ e os $\alpha_{ij}$ s\~ao os elementos da matriz de
transfoma\c{c}\~ao.

Lembremos da difini\c{c}\~ao do delta de Kronecker,
\[\delta_{ij}=\left \{ \begin{array}{l}
                0, \mbox{   }\mbox{   }\mbox{se } i \neq j,\\
                1, \mbox{   }\mbox{   }\mbox{se } i= j.\\
\end{array} \right.\]
Pela conven\c{c}\~ao de Einstein temos que
$\delta_{ii}=\delta_{11}+\delta_{22}+\delta_{33}=3$.

Se fizermos $\alpha_{ij}=\delta_{ij}$, a tranforma\c{c}\~ao de
coordenadas ser\'a
\[x'_{i}=(x_{i}-x_{oi}).\]
Esta transforma\c{c}\~ao descreve uma transla\c{c}\~ao paralela de um sistem de coordenada
com respeito ao outro (veja a figura). 

\begin{tikzpicture}
\draw[thick,->] (0,0) -- (4.5,0) node[anchor=north west] {$x_2$};
\draw[thick,->] (0,0) -- (0,4.5) node[anchor=south east] {$x_3$};
\draw[thick,->] (0,0) -- (-2.5,-2.5) node[anchor=north west] {$x_1$};
\node[below] at (0,0) {$0$};
\end{tikzpicture}
\begin{tikzpicture}
\draw[thick,->] (5,0) -- (9.5,0) node[anchor=north west] {$x_2'$};
\draw[thick,->] (5,0) -- (5,4.5) node[anchor=south east] {$x_3'$};
\draw[thick,->] (5,0) -- (2.5,-2.5) node[anchor=north west] {$x_1'$};
\node[below] at (5,0) {$0'$};
\end{tikzpicture}

Se fizermos  $x_{oj}=0$, a rela\c{c}\~ao de
transforma\c{c}\~ao ser\'a
\[x'_{i}=\alpha_{ij}x_{j},\]
que descreve uma rota\c{c}\~ao em torno da origem do primeiro
sistema (ver figura).

\begin{tikzpicture}
\draw[thick,->] (0,0) -- (4.5,0) node[anchor=north west] {$x_2$};
\draw[thick,->] (0,0) -- (0,4.5) node[anchor=south east] {$x_3$};
\draw[thick,->] (0,0) -- (-2.5,-2.5) node[anchor=north west] {$x_1$};
%\node[below] at (0,0) {$0$};
%\end{tikzpicture}
%\begin{tikzpicture}
\draw[thick,->] (0,0) -- (4.5,0.5) node[anchor=north west] {$x_2'$};
\draw[thick,->] (0,0) -- (-0.5,4.5) node[anchor=south east] {$x_3'$};
\draw[thick,->] (0,0) -- (-2,-2.5) node[anchor=north west] {$x_1'$};
\node[below right] at (0,0) {$0 = 0'$};
\end{tikzpicture}

Agora iremos investigar o significado dos elementos da matriz de transforma\c{c}\~ao
$\alpha_{ij}$. Vamos introduzir o vetor raio $r_{i}$, partindo da
origem de um sistema de coordenadas at\'e um ponto $x_{i}$, isto
\'e, $x_{i}=r_{i}$. Sejam $\vec{i}_{1}$, $\vec{i}_{2}$ e
$\vec{i}_{3}$ tr\^es vetores mutuamente ortogonais unit\'arios
perpendiculares e orientados ao longo dos eixos $x_{1}$, $x_{2}$,
$x_{3}$, respectivamente. Assim como $\vec{i'_{1}}$,
$\vec{i'_{2}}$, $\vec{i'_{3}}$ s\~ao vetores unit\'arios no
sistema $x'_{i}$.
Podemos expressar o vetor $r_{i}$ nos dois sistemas de
coordenadas
\[ \vec{r}=x_{j}\vec{i_{j}}=x_{1}\vec{i}_{1}+x_{2}\vec{i}_{2}+x_{3}\vec{i}_{3}=x'_{1}\vec{i'}_{1}+
x'_{2}\vec{i'}_{2}+x'_{3}\vec{i'}_{3}.\]%=x_{j}\vec{i_{j}}.\]
Se substituimos, no lado direito, a rela\c{c}\~ao de transforma\c{c}\~ao
\[x'_{i}=\alpha_{ij}x_{j},\]
teremos
\[\vec{r}=x_{1}\vec{i}_{1}+x_{2}\vec{i}_{2}+
x_{3}\vec{i}_{3}=\alpha_{1j}x_{j}\vec{i'}_{1}+
\alpha_{2j}x_{j}\vec{i'}_{2}+\alpha_{3j}x_{j}\vec{i'}_{3}.\]
Se multiplicarmos escalarmente a equa\c{c}\~ao acima por $\vec{i'}_{k}$, teremos
%\[x_{j}\vec{i}_{j}\vec{i'}_{k}=
%\alpha_{kj}x_{j} \Rightarrow
\[ \alpha_{kj}=\vec{i}_{j} \cdot \vec{i'}_{k}.\] Note que os elementos de
$\alpha_{ij}$ s\~ao cossenos, denominados cossenos direcionais.

Considerando a transforma\c{c}\~ao inversa a
\[x'_{i}=\alpha_{ij}(x_{j}-x_{oj}).\]
Iremos mostrar que para a matriz $\alpha_{ij}$, existe uma inversa,
$(\alpha^{-1})_{ij}$, tal que
\[\alpha_{ij}(\alpha^{-1})_{jk}=\delta_{ik}, \mbox{    }\mbox{    }\mbox{    }(\alpha^{-1})_{jk}\alpha_{kl}=\delta_{jl}.\]
Com a matriz inversa podemos escrever a transfoma\c{c}\~ao inversa da seguinte forma
\[x_{i}=x_{oi}+(\alpha^{-1})_{ij}x'_{j}.\]

Como a dist\^ancia entre $A$ e $B$ deve ser independente do
sistema de coordenadas utilizado, temos
\[ (x_{i}^{B}-x_{i}^{A})(x_{i}^{B}-x_{i}^{A})=(x'_i{}^{B}-x'_i{}^{A})(x'_i{}^{B}-x'_i{}^{A}),\]
o que nos d\'a
\[ (x_{i}^{B}-x_{i}^{A})(x_{i}^{B}-x_{i}^{A})=\alpha_{kj}\alpha_{kl}(x_{j}^{B}-x_{j}^{A})(x_{l}^{B}-x_{l}^{A}).\]
Ent\~ao,
\[ \alpha_{kj}\alpha_{kl}=\delta_{jl}.\]

Agora supondo que exista a inversa, teremos
\[(\alpha^{-1})_{jk}\alpha_{kl}=\delta_{jl},\]
e portanto,
\[(\alpha^{-1})_{jk}=\alpha_{kj}.\]

Isso significa que a matriz inversa de $\alpha_{ij}$ \'e obtida
transpondo a matriz $\alpha_{ij}$, por esta propridade vemos que a
matriz em discuss\~ao \'e ortogonal. Pela ortogonalidade, temos
que
\[|det(\alpha_{ij})|=1\].

\subsection{Defini\c{c}\~ao e Propriedades dos Tensores Cartesianos}

Nesta se\c{c}\~ao vamos definir um tensor cartesiano e algumas de
suas propriedades.

Para cada ponto do espa\c{c}o, o tensor cartesiano \'e uma
sequ\^encia
 finita de n\'umeros os quais correspondem, de maneira \'unica, as coordenadas
cartesianas do ponto. A sequ\^encia representa um tensor se
satisfaz as seguintes  condi\c{c}\~oes:

1-) A sequ\^encia cont\'em $3^{k}$ n\'umeros onde $3$ representa a
dimens\~ao do espa\c{c}o e $k$ \'e um inteiro n\~ao-negativo
chamado ordem:
\[T_{m_{1},m_{2},...,m_{k}}, \mbox{   }\mbox{   }\mbox{   }\mbox{   } m_{i}=1,2,3.\]

2-) Um elemento do tensor no sistema de coordenadas $x'_{i}$ \'e
associado ao tensor no sistema de coordenadas $x_{i}$ pela
rela\c{c}\~ao:
\[T'_{p_{1},p_{2},...,p_{k}}=
\alpha_{p_{1}m_{1}}\alpha_{p_{2}m_{2}}...\alpha_{p_{k}m_{k}}T_{m_{1},m_{2},...,m_{k}}.\]

Para um tensor de ordem zero, definimos a transforma\c{c}\~ao como $T=T'$. J\'a para tensores
de primeira (vetor) e de segunda ordem, temos as seguintes formas de transforma\c{c}\~ao
\[T'_{i}=\alpha_{ij}T_{j},\mbox{   }\mbox{   }\mbox{   }\mbox{   }
T'_{ij}=\alpha_{im}\alpha_{jn}T_{mn}.\]

Para tensores de mesma ordem somente, adi\c{c}\~ao e subtra\c{c}\~ao
correspondem a
\[A_{ij} \pm B_{ij}=C_{ij}.\]

Em contraste com adi\c{c}\~ao e subtra\c{c}\~ao, na multiplica\c{c}\~ao, n\~ao
necessitamos que os tensores tenham a
mesma ordem. Se todos os \'\i ndices s\~ao distintos a
multiplica\c{c}\~ao soma os \'\i ndices, como por exemplo,
\[A_{ij}B_{klm}=C_{ijklm}.\]
J\'a a multiplica\c{c}\~ao de tensores com \'\i ndices iguais reduz
a ordem do tensor resultante por $2$, como em
\[A_{ij}B_{j}=C_{i}.\]
A redu\c{c}\~ao da ordem \'e dita contra\c{c}\~ao. Um t\'\i pico
exemplo \'e o produto escalar de dois vetores.

Posteriormente iremos trabalhar com com tensores sim\'etricos e
anti-sim\'etricos. Entendemos um  tensor sim\'etrico com a seguinte
propriedade
\[T_{m_{1},m_{2},...,m_{k}}=T_{m_{2},m_{1},...,m_{k}},\]
esse tensor \'e sim\'etrico nos \'\i ndices $m_{1}$ e
$m_{2}$. Entendemos um  tensor anti-sim\'etrico com a seguinte
propriedade
\[T_{m_{1},m_{2},...,m_{k}}=-T_{m_{k},m_{2},...,m_{1}},\]
esse tensor \'e anti-sim\'etrico nos \'\i ndices $m_{1}$ e $m_{k}$.

Um tensor de
ordem $2$, pode ser escrito da seguinte forma (soma de um tensor
sim\'etrico com um anti-sim\'etrico):
\[B_{ij}=\frac{1}{2}(B_{ij}+B_{ji})+\frac{1}{2}(B_{ij}-B_{ji})=S_{ij}+A_{ij}.\]

Simetria e anti-simetria s\~ao preservadas ap\'os uma
transforma\c{c}\~ao de coordenadas.
 Se $T_{mnl}=T_{nml}$, temos
\[T'_{ijk}=\alpha_{im}\alpha_{jn}\alpha_{kl}T_{mnl}=\alpha_{im}\alpha_{jn}\alpha_{kl}T_{nml}=
\alpha_{jn}\alpha_{im}\alpha_{kl}T_{nml}=T'_{jik}.\] 
O racioc\'\i nio \'e o mesmo para tensores anti-sim\'etricos.

\subsection{Tensores Isotr\'opicos (especiais)}


Um tensor \'e dito isotr\'opico se ap\'os uma transforma\c{c}\~ao
de coordenadas o tensor se mant\'em inalterado.
Por exemplo o delta de Kronecker \'e um tensor de ordem $2$
isotr\'opico, pois
\[\delta'_{ij}=\alpha_{im}\alpha_{jn}\delta_{mn}=\alpha_{im}\alpha_{jm}=\delta_{ij}.\]
Um outro tensor isotr\'opico \'e o s\'\i mbolo de Levi-Civita,
$\epsilon_{ijk}$, que \'e um tensor de ordem $3$, definido como:
\[\epsilon_{ijk}=\left\{ \begin{array}{l}
           1,\mbox{   }\mbox{   }\mbox{   }\mbox{se $i,j,k$ formam uma
permuta\c{c}\~ao par de $1,2,3$};\\
          -1,\mbox{   }\mbox{   }\mbox{   }\mbox{se $i,j,k$ formam uma
permuta\c{c}\~ao \'\i mpar de $1,2,3$};\\
           0,\mbox{   }\mbox{   }\mbox{   }\mbox{se dois \'\i ndices s\~ao repetidos}.\\
\end{array} \right.\]
Pode-se provar a seguinte identidade
\[\epsilon_{ijk}\epsilon_{lmn}= \begin{array}{lllll}
                \vline & \delta_{il} & \delta_{im} & \delta_{in} & \vline\\
                \vline &\delta_{jl} & \delta_{jm} & \delta_{jn} &\vline\\
                 \vline &\delta_{kl} & \delta_{km} & \delta_{kn} &\vline\\
\end{array}.\]
E, portanto,
\[\epsilon_{ijk}\epsilon_{imn}= \begin{array}{lllll}
                \vline & 3 & \delta_{im} & \delta_{in}&\vline \\
                \vline & \delta_{ji} & \delta_{jm} & \delta_{jn}& \vline\\
               \vline & \delta_{ki} & \delta_{km} & \delta_{kn}& \vline\\
\end{array}=\delta_{jm}\delta_{kn}-\delta_{jn}\delta_{km},\]
que \'e uma identidade importante. A partir disto, temos que
\[\epsilon_{ijk}\epsilon_{ijn}=3\delta_{kn}-\delta_{kn}=2\delta_{kn}\]
e
\[\epsilon_{ijk}\epsilon_{ijk}=6.\]

Para uma matriz arbitr\'aria $a_{ij}$, usando as identidades  para
$\epsilon_{ijk}\epsilon_{lmn}$, podemos escrever
\[\epsilon_{ijk}a_{1i}a_{2j}a_{3k}=\epsilon_{ijk}\epsilon_{123}a_{1i}a_{2j}a_{3k}=
\begin{array}{lllll}
                \vline&\delta_{i1} & \delta_{i2} & \delta_{i3}&\vline \\
                \vline&\delta_{j1} & \delta_{j2} & \delta_{j3}&\vline \\
                \vline&\delta_{k1} & \delta_{k2} & \delta_{k3}& \vline\\
\end{array}a_{1i}a_{2j}a_{3k}=\] \[ =\begin{array}{lllll}
                \vline&a_{11} & a_{12} & a_{13}& \vline\\
                \vline&a_{21} & a_{22} & a_{23}&\vline \\
                \vline&a_{31} & a_{32} & a_{33}&\vline \\
\end{array}=det(a_{ij}).\]
A rela\c{c}\~ao acima pode ser generalizada
\[\epsilon_{ijk}a_{mi}a_{nj}a_{lk}=det(a_{pq})\epsilon_{mnl}.\]
Com a rela\c{c}\~ao anterior podemos demonstrar que  o s\'\i mbolo
de Levi-Civita \'e um tensor isotr\'opico de ordem $3$. De fato,
\[\epsilon'_{ijk}=\alpha_{im}\alpha_{jn}\alpha_{kl}\epsilon_{mnl}=det(\alpha_{pq})\epsilon_{ijk}=\epsilon_{ijk}.\]
E tamb\'em temos a seguinte rela\c{c}\~ao
\[det(a_{pq})=\frac{1}{6}\epsilon_{mnl}\epsilon_{ijk}a_{mi}a_{nj}a_{lk}.\]



%\begin{thebibliography}{bib}
%\bibitem{pse}{I. P\v{s}en\v{c}\'\i k - \emph{Introduction to seismic methods}, Lecture Notes, PPPG/UFBa, 1994.}
%\end{thebibliography}

