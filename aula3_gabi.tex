
\section{Par\^ametros El\'asticos para Meios Isotr\'opicos}

Como foi visto anteriormente, o meio isotr\'opico pode ser descrito por
dois par\^ametros, $\mu$ e $\lambda$, conhecidos como par\^ametros
de Lam\'e. Pode-se introduzir outros
par\^ametros al\'em destes, mas comecemos estudando a rigidez $\mu$.
Pela lei de Hooke
\begin{eqnarray}
\tau_{ij} =
c_{ijkl}e_{kl} \mbox{, onde } c_{ijkl} = \lambda\delta_{kl}\delta_{ij} 
+ \mu(\delta_{ik}\delta_{jl} + \delta_{jk}\delta_{il})
\end{eqnarray}
para $i \neq j$ temos
\begin{eqnarray}
\tau_{ij} = 2\mu e_{ij}.
\end{eqnarray}

Vemos que a rigidez $\mu$ relaciona a tens\ao\ de cisalhamento com a
deforma\cao\ de cisalhamento, e as vezes $\mu$ \'e chamado de m\'odulo
de cisalhamento. Juntamente com o fato da energia de deforma\cao\ $W$
ser definida positiva, a rigidez de um s\'olido isotr\'opico \'e sempre
positiva, $\mu > 0$. Agora, vejamos o m\'odulo bulk $k$ e a
compressibilidade $\kappa$. Considere uma deforma\cao\ esfericamente
sim\'etrica, onde $e_{11}=e_{22}=e_{33} \neq 0$ e
$e_{12}=e_{13}=e_{23}=0$, que significa que o material foi estirado
igualmente em todas as dire\coes. Neste caso, a tens\ao\ correspondente
\'e chamada hidrost\'atica. Pela lei de Hooke para meios isotr\'opicos
temos, como visto anteriormente
\begin{eqnarray}
\tau_{ii} = (3\lambda+2\mu)\theta = 3k\theta, \; k = \lambda + 2/3\mu.
\end{eqnarray}

O m\'odulo bulk $k$ tamb\'em \'e conhecido por incompressibilidade e \'e
sempre positivo, ent\ao\ uma tens\ao\ positiva $\tau_{ii}>0$ causa uma
dilata\cao\ positiva $\theta>0$.

A compressibilidade $\kappa$ \'e definida como
\begin{eqnarray}
\kappa = \frac{1}{k}.
\end{eqnarray}

Considere agora uma haste orientada ao longo do eixo $x_{1}$ e puxada
nesta dire\cao, provocando uma tens\ao\ ao longo de $x_{1}$ e portanto
$\tau_{11} \neq 0$, e considere tamb\'em que todos os outros componentes
do tensor de tens\ao\ s\ao\ nulos. J\'a vimos da lei de Hooke que
\begin{eqnarray}
e_{ij} &=& \frac{1}{2\mu}\tau_{ij} - \lambda\delta_{ij}\frac{\tau_{kk}}{2\mu(3\lambda + 2\mu)} \Longrightarrow \nonumber \\
e_{11} &=& \frac{\lambda + \mu}{\mu(3\lambda+2\mu)}\tau_{11}, \; 
e_{22} = e_{33} = -\frac{\lambda\tau_{11}}{2\mu(3\lambda+2\mu)}, \; e_{12} = e_{13} = e_{23} = 0.
\end{eqnarray}

Introduzimos agora os par\^ametros $E$ e $\sigma$ da seguinte forma,
\begin{eqnarray}
\tau_{11} &=& Ee_{11}, \; \sigma = -\frac{e_{22}}{e_{11}} = -\frac{e_{33}}{e_{11}} \Longrightarrow \nonumber \\
E &=& \mu\frac{(3\lambda + 2\mu)}{\lambda + \mu}, \; \sigma = \frac{1}{2}\frac{\lambda}{\lambda + \mu}
\end{eqnarray}
onde $E$ \'e chamado de m\'odulo de Young e representa a raz\ao\ entre a
tens\ao\ e a extens\ao\ relativa causada pela tens\ao\, e $\sigma$ \'e o
raio de Poisson que representa a raz\ao\ entre o raio da contra\cao\
relativa da espessura da haste e a sua extens\ao\ relativa,
respectivamente. Por defini\cao\ s\ao\ ambos positivos. Para $\lambda$
grande e/ou $\mu$ pequeno temos $0<\sigma<1/2$, e portanto, $\lambda>0$.

\subsection{Unidades, Valores Num\'ericos}

Da lei de Hooke vemos que os par\^ametros $c_{ijkl}$, ${\mu}$ e
${\lambda}$ possuem as mesmas unidades da tens\ao, ou seja, s\ao\
medidos em Pascal $(1Pa = kg\ m^{-1} s^{-2})$. Por suas defini\coes\ $k$ e
$E$ tamb\'em o s\~ao. O raio de Poisson e $\sigma$ s\ao\ adimensionais e
$\kappa$ \'e medido em $(Pa)^{-1}$.

\section{Equa\coes\ Elastodin\^amicas}

A teoria estudada at\'e agora envolve propaga\cao\ de ondas em s\'olidos
el\'asticos. No entanto, \`as vezes precisamos considerar propaga\cao\
de ondas atrav\'es de fluidos el\'asticos, como por exemplo os oceanos e
o n\'ucleo externo da Terra. \`As vezes a propaga\cao\ em materiais
s\'olidos \'e tamb\'em considerada como propaga\cao\ em fluidos (o
ent\ao\ chamado caso ac\'ustico em sismologia de reflex\~ao). Podemos
derivar as equa\coes\ de movimento para fluidos de equa\coes\
hidromec\^anicas. Outra maneira \'e fazer $\mu \longrightarrow 0$ nas
equa\coes\ obtidas anteriormente.

\subsection{Equa\coes\ Elastodin\^amicas para Meios S\'olidos}

Introduzimos a forma geral da lei de Hooke generalizada
\begin{eqnarray*}
\tau_{ij} = c_{ijkl}e_{kl} = c_{ijkl}\frac{\partial u_k}{\partial x_l},
\end{eqnarray*}
na equa\cao\ do movimento
\begin{eqnarray*}
\tau_{ji,j} + f_{i} = \rho\frac{\partial^2 u_i}{\partial t^2}
\end{eqnarray*}
e obtemos a forma mais geral da equa\cao\ elastodin\^amica ou equa\cao\
da onda el\'astica
\begin{eqnarray}
\frac{\partial}{\partial x_j}\left(c_{ijkl}\frac{\partial u_k}{\partial x_l}\right) + f_i
= \rho\frac{\partial^2 u_i}{\partial t^2}
\end{eqnarray}

Como $c_{ijkl} = c_{ijkl}(x_m)$ e $\rho = \rho(x_m)$ esta equa\cao\
descreve um meio anisotr\'opico n\~ao-homog\~eneo geral. Esta \'e uma
equa\cao\ diferencial parcial vetorial linear de segunda ordem com
coeficientes vari\'aveis, e n\ao\ possui uma solu\cao\ de forma fechada.
Pode-se usar m\'etodos num\'ericos como Diferen\c{c}as Finitas ou
Elementos Finitos para obter solu\coes\ num\'ericas. Outra maneira \'e
aplicar m\'etodos assint\'oticos de alta frequ\^encias como m\'etodo de
raios.

Para um meio anisotr\'opico homog\^eneo onde os par\^ametros el\'asticos e
a densidade s\ao\ constantes, a equa\cao\ acima fica
\begin{eqnarray}
c_{ijkl}\frac{\partial^2 u_k}{\partial x_j\partial x_l} + f_i = \rho\frac{\partial^2 u_i}{\partial t^2},
\end{eqnarray}
e agora os coeficientes s\ao\ constantes. Para esta equa\cao\ podemos
encontrar uma solu\cao\ na forma de ondas planas.

A equa\cao\
elastodin\^amica para um meio isotr\'opico n\~ao homog\^eneo pode ser
obtida inserindo
\begin{eqnarray}
c_{ijkl} = \lambda\delta_{kl}\delta_{ij} + \mu(\delta_{ik}\delta_{kl} + \delta_{jk}\delta_{kl})
\end{eqnarray}
onde $c_{ijkl} = c_{ijkl}(x_m)$, $\lambda = \lambda(x_m)$, $\mu =
\mu(x_m)$ na equa\cao\ do movimento. Reescrevemos a equa\cao\ do
movimento da seguinte maneira
\begin{eqnarray}
(c_{ijkl}u_{k,l})_{,j} + f_i = \rho u_{i,tt}
\end{eqnarray}
e obtemos
\begin{eqnarray}
\lambda_{,i}u_{k,k} + \mu_{,l}u_{i,l} + \mu_{,k}u_{k,i} + \lambda u_{l,li} + \mu u_{i,ll} + \mu u_{j,ij} + f_i 
= \rho u_{i,tt}.
\label{eqelaiso}
\end{eqnarray}
Isto pode ser reescrito em nota\cao\ vetorial se observarmos que
\begin{eqnarray}
(\nabla\mu\times\mbox{rot}\vec{u})_i = \mu_{,k}u_{k,i} - \mu_{,k}u_{i,k}.
\end{eqnarray}
Temos ent\ao\
\begin{eqnarray}
(\lambda + \mu)\nabla(\mbox{div}\vec{u}) + \mu\Delta\vec{u} + \nabla\lambda\mbox{div}\vec{u} 
+ \nabla\mu\times\mbox{rot}\vec{u} + 2(\nabla\mu\cdot\nabla)\vec{u} + \vec{f} = \rho\frac{\partial^2 u_i}{\partial t^2}.
\end{eqnarray}

Fazendo $\lambda$, $\mu$ e $\rho$ constantes, isto \'e,
$\lambda_{,i}=\mu_{,i}=\rho_{,i}=0$, obtemos a equa\cao\ de movimento
para meios isotr\'opicos homog\^eneos,
\begin{eqnarray}
(\lambda + \mu)u_{k,ki} + \mu u_{i,kk} + f_i = \rho u_{i,tt}
\end{eqnarray}
ou em nota\cao\ vetorial
\begin{eqnarray}
(\lambda + \mu)\nabla(\mbox{div}\vec{u}) + \mu\Delta\vec{u} + \vec{f} = \rho\frac{\partial^2\vec{u}}{\partial t^2}
\end{eqnarray}
Ainda podemos usar a identidade
\begin{eqnarray}
\Delta\vec{u} = \nabla(\mbox{div}\vec{u}) - \mbox{rot}(\mbox{rot}\vec{u})
\end{eqnarray}
para reescrever como
\begin{eqnarray}
(\lambda + 2\mu)\nabla(\mbox{div}\vec{u}) - \mu\mbox{rot}(\mbox{rot}\vec{u}) + \vec{f} 
= \rho\frac{\partial^2\vec{u}}{\partial t^2}
\end{eqnarray}

\subsection{Equa\coes\ de Movimento para Fluidos - Caso Ac\'ustico}

Como foi dito, podemos obter a equa\cao\ de movimento
para fluidos fazendo $\mu \longrightarrow 0$. Isto nos
fornece $E = 0$, $\sigma = 1/2$ e $k = \lambda$. Nos fluidos, a tens\ao\
hidrost\'atica e a tens\ao\ m\'edia $\tau_{ii}/3$ normalmente s\ao\
negativos, e esta \'ultima \'e denotada por $-p$, onde $p$ \'e chamado
de press\ao. A lei de Hooke
\begin{eqnarray}
\tau_{ij} = \lambda\theta\delta_{ij} = k\theta\delta_{ij} \ ; \   (\mu = 0, \; \lambda = k)
\end{eqnarray}
pode ent\ao\ ser escrita como
\begin{eqnarray}
p = -k\theta \mbox{ ou } \theta = -\kappa p
\end{eqnarray}

Em ac\'ustica \'e comum trabalhar com press\ao\ $p$ e velocidade de
part\'icula $v_i = \partial u_i/\partial t$ no lugar do deslocamento
$u_i$ da part\'icula. Reescrevemos ent\ao\ a equa\cao\ elastodin\^amica
como
\begin{eqnarray}
-p_{,i} + f_i = \rho\frac{\partial v_i}{\partial t}
\end{eqnarray}
Ent\~ao, em vez de tr\^es componentes desconhecidos de $u_i$,
temos tr\^es componentes desconhecidos de $v_i$ e a press\~ao
$p$ \'e a quarta
quantidade desconhecida. Assim, precisamos de mais uma equa\cao\ que
\'e obtida derivando a lei de Hooke $\theta = -\kappa p$ com rela\cao\
ao tempo. O sistema completo com 4 equa\coes\ \'e ent\ao
\begin{eqnarray}
p_{,i} + \rho\frac{\partial v_i}{\partial t} = f_i,
\label{eqondpres} \\
v_{k,k} + \kappa\frac{\partial p}{\partial t} = 0 .
\label{eqondvel}
\end{eqnarray}
Derivando as equa\coes\ acima para $x_i$ e $t$ respectivamente obtemos:
\begin{eqnarray}
(p_{,i} + \rho\frac{\partial v_i}{\partial t} = f_i)_{,i} &\Rightarrow& 
(p_{,i}/\rho)_{,i} + \frac{\partial v_{i,i}}{\partial t} = (f_i/\rho)_{,i} \;\; , \\
(v_{k,k} + \kappa\frac{\partial p}{\partial t} = 0)_{,t} &\Rightarrow&
\frac{\partial v_{k,k}}{\partial t} + \kappa\frac{\partial^2 p}{\partial t^2} = 0.
\end{eqnarray}
Ent\ao\ temos
\begin{eqnarray}
\frac{\partial v_{k,k}}{\partial t} = -\kappa\frac{\partial^2 p}{\partial t^2},
\end{eqnarray}
portanto
\begin{eqnarray}
(p_{,i}/\rho)_{,i} - \kappa\frac{\partial^2 p}{\partial t^2} = (f_i/\rho)_{,i},
\end{eqnarray}
ou na forma vetorial
\begin{eqnarray}
\nabla(\nabla p/\rho) - \kappa\frac{\partial^2 p}{\partial t^2} = \nabla(\vec{f}/\rho),
\end{eqnarray}
que \'e a equa\cao\ da onda ac\'ustica.
Caso a densidade for constante, podemos multiplicar a equa\c{c}\~ao por
$\rho$ e simplificar esta express\~ao, 
obtendo
\begin{eqnarray}
\nabla(\nabla p) - \frac{1}{c^2}\frac{\partial^2 p}{\partial t^2} = \nabla(\vec{f}),
\label{ondaacu}
\end{eqnarray}
onde introduzimos a velocidede de propaga\c{c}\~ao
$c=1/\sqrt{\kappa\rho}=\sqrt{k/\rho}$. Equa\c{c}\~ao (\ref{ondaacu})
\'e a equa\cao\ da onda ac\'ustica com densidade constante, muitas
as vezes simplesmente denominada equa\c{c}\~ao da onda.
%\end{document}
