\section{Par\^ametros El\'asticos}

J\'a discutimos que $c_{ijkl}$ s\ao\ elementos de um
tensor de quarta ordem com 81 elementos. Mostraremos a seguir 
que nem todos eles s\ao\ independentes, devido \`a
simetria que a matriz dos par\^ametros el\'asticos $c_{ijkl}$ 
apresenta em alguns de seus \'indices.

Devido a simetria do tensor de tens\ao\ e do tensor de
deforma\cao, conclu\'imos imediatamente da lei de Hooke, que
\begin{eqnarray}
c_{ijkl} = c_{jikl} = c_{ijlk} \; .
\end{eqnarray}
Outras condi\coes\ de simetria seguem da express\ao\ para
energia de deforma\cao
\begin{eqnarray}
c_{ijkl} = \frac{\partial \tau_{ij}}{\partial e_{kl}}
= \frac{\partial^2 W}{\partial e_{ij} \partial e_{kl}}
= \frac{\partial^2 W}{\partial e_{kl} \partial e_{ij}}
= c_{klij} \; .
\end{eqnarray}
Todas as tr\^es condi\coes\ de simetria reduzem os
par\^ametros el\'aticos independentes de 81 para 21.

Devido a simetria nos \indices\ $i,j$ e $k,l$, existem
6 componentes de $i,j$ e $k,l$ (ao inv\'es  de 9 no caso
n\ao\ sim\'etrico) especificando par\^ametros el\'asticos
independentes. Estas 6 combina\coes\ de $i,j$ e $k,l$
s\ao\ algumas vezes substitu\ih das por inteiros de 1 a 6
e o tensor de par\^ametros el\'asticos \'e ent\ao\
substitu\ih do por uma matriz 6 x 6. Denotamos os
elementos desta matriz por $C_{\alpha \beta}$ ($\alpha,
\beta = 1,2,...,6;$ os \indices\ Gregos diferindo assim
dos \indices\ em Latim que assumem valores 1,2,3). O
\indice\ $\alpha$ corresponde aos \indices\ $i,j$ e 
$\beta$ aos \indices\ $k,l$ para o tensor $c_{ijkl}$. Essa
representa\c{c}\~ao \'e denominada nota\c{c}\~ao de Voigt, e possui
uma forma de f\'acil memoriza\c{c}\~ao.
os \indices\ s\ao\ relacionados como segue
\begin{eqnarray*}
1 \leftrightarrow 1,1 \;\;\;\;\;\;
2 \leftrightarrow 2,2 \;\;\;\;\;\;
3 \leftrightarrow 3,3 \;\;\;\;\;\;
4 \leftrightarrow 2,3 \;\;\;\;\;\;
5 \leftrightarrow 3,1 \;\;\;\;\;\;
6 \leftrightarrow 1,2 \;.
\end{eqnarray*} 

Devido a simetria $c_{ijkl} = c_{klij}$ da matriz
$c_{ijkl}$ a matriz $C_{\alpha \beta}$ \'e tamb\'em
sim\'etrica $C_{\alpha \beta} = C_{\beta \alpha}$, mas $C$
n\ao\ \'e um tensor.

Podemos introduzir uma nota\cao\ comprimida similar \`a de Voigt para
os elementos do tensor de deforma\cao\ e de tens\ao, 
$e_{kl}$ e $\tau_{ij}$. Introduzimos $e_{\beta}$ e
$\tau_{\alpha}$ de tal maneira que a Lei de Hooke
generalizada e a express\ao\ para a energia de deforma\cao\
tem a forma
\begin{eqnarray}
\tau_{\alpha} = C_{\alpha \beta} e_{\beta} \;\;\;\;\;\;
W = \frac{1}{2}\tau_{\alpha}e_{\alpha} \; .
\end{eqnarray}

Por compara\cao\ da express\ao\ para $W$ na nota\cao\
padr\ao\ e na nota\cao\ comprimida encontramos que
\begin{eqnarray*}
e_{11} \leftrightarrow e_{1} \;\;\;\;\;\;
e_{22} \leftrightarrow e_{2} \;\;\;\;\;\;
e_{33} \leftrightarrow e_{3} \;\;\;\;\;\;
2e_{23} \leftrightarrow e_{4} \;\;\;\;\;\;
2e_{13} \leftrightarrow e_{5} \;\;\;\;\;\;
2e_{12} \leftrightarrow e_{6} \\
\tau_{11} \leftrightarrow \tau_{1} \;\;\;\;\;\;
\tau_{22} \leftrightarrow \tau_{2} \;\;\;\;\;\;
\tau_{33} \leftrightarrow \tau_{3} \;\;\;\;\;\;
\tau_{23} \leftrightarrow \tau_{4} \;\;\;\;\;\;
\tau_{13} \leftrightarrow \tau_{5} \;\;\;\;\;\;
\tau_{12} \leftrightarrow \tau_{6} \; .
\end{eqnarray*}

A energia de deforma\cao\ \'e m\ih nima se n\ao\ existe
deforma\cao. Portanto definimos $W=0$. Tamb\'em fizemos algumas
considera\c{c}\~oes sobre o comportamento termodin\^amico  da deforma\c{c}\~ao,
 e para que as solu\c{c}\~oes do problema de elasticidade
linear sejam \'unicas, \'e necess\'ario que $W>0$. Esta desigualdade
implica que $W$ \'e uma matriz positiva definida, i.e.,
\begin{eqnarray}
W = \frac{1}{2}C_{\alpha \beta}e_{\alpha}e_{\beta}>0 \; .
\end{eqnarray}

Como consequ\^encia disto, todo menor principal da matriz $C_{\alpha
\beta}$ s\ao\ positivos (menor principal de uma matriz \'e a matriz que
permanece depois da elimina\cao\ das $n$ \'ultimas linhas e colunas).
Isto significa que todos os elementos da diagonal da matriz $C_{\alpha
\beta}$ devem ser positivos. Outra consequ\^encia da defini\cao\
positiva da express\ao\ para a energia de deforma\cao\ \'e
\begin{eqnarray}
|C_{\alpha \beta}| = \mbox{det}(C_{\alpha \beta}) > 0 \; ,
\end{eqnarray}
que nos indica que a matriz inversa de $C_{\alpha \beta}$ exite tal que
$C_{\alpha \beta} S_{\beta \gamma} = \delta_{\alpha \gamma}$. A matriz 6
x 6 $S_{\beta \gamma}$ tem sua equival\^encia no tensor $s_{ijkl}$ de
quarta ordem que tem as mesmas propriedades de simetria que o tensor
$c_{ijkl}$. Com esta matriz a Lei de Hooke generalizada pode ser escrita
como
\begin{eqnarray}
e_{ij} = s_{ijkl} \tau_{ij} \; .
\end{eqnarray}
Vamos notar que $s_{ijkl}$ \'e conhecido como tensor compliance
(for\ca\ de resist\^encia de um material contra a deforma\cao).

\section{V\'arios tipos de simetria anisotr\'opica}

Anisotropia se refere \`a depend\^encia direcional das propriedades do meio.
Os diferentes tipos de anisotropia em materiais s\~ao determinados
pela exist\^encia de simetria na estrutural interna do material.
Quanto mais simetria interna, mais simples \'e a estrutura do tensor de
rigidez ($c_{ijkl}$). Cada tipo de simetria resulta na invari\^ancia
do tensor de rigidez a certas transforma\c{c}\~oes de simetria
(rota\c{c}\~oes sobre um eixo espec\'ifico, reflex\~oes sobre planos
espec\'ificos).

Um material anisotr\'opico pode ter diferentes graus de
simetria. Com o aumento do grau de simetria, o n\'umero
de par\^ametros el\'asticos independentes decresce. Sobre
simetria de um material, n\'os entendemos que depois de uma
transforma\cao\ de um sistema de coordenadas em que o tensor
$c_{ijkl}$ \'e especificado, suas propriedades permanecem
as mesmas.

H\'a um sistema inteiro de materiais anisotropicos com
graus diferentes de simetria. O material mais geral que
\'e especificado por 21 par\^ametros el\'asticos
independentes \'e chamado {\bf tricl\'inico}. O material que
\'e especificado por 13 par\^ametros independentes \'e
chamado {\bf monocl\'inico}. Estes sistemas n\ao\ s\ao\
considerados em aplica\coes\ s\ismicas.

A mais complexa anisotropia considerada algumas vezes em 
sismologia \'e a simetria {\bf ortorr\^ombica} com 9
par\^ametros el\'asticos independentes. Esta \'e
caracterizada por tr\^es eixos de simetria mutuamente
perpendiculares. Uma rota\cao\ de $180^{o}$ em
torno de algum eixo n\ao\ altera o tensor $c_{ijkl}$.
Esta condi\cao\ leva a matriz $C_{\alpha \beta}$ em
\begin{eqnarray}
C_{\alpha \beta} = \left(
\begin{array}{cccccc}
C_{11} & C_{12} & C_{13} & 0      & 0      & 0 \\
       & C_{22} & C_{23} & 0      & 0      & 0 \\
       &        & C_{33} & 0      & 0      & 0 \\
       &        &        & C_{44} & 0      & 0 \\
       &        &        &        & C_{55} & 0 \\
       &        &        &        &        & C_{66}
\end{array} \right) \; .
\end{eqnarray}

A matriz acima \'e completamente especificada quando o
sistema de coordenadas em que esta \'e indicada \'e
tamb\'em dado. No caso acima, o sistema de coordenadas foi
escolhido tal que os eixos de coordenada coincidem com os
eixos de simetria. Em um sistema de coordenadas diferente
haveriam elementos menores que zero, mas o n\'umero de
par\^ametros independentes permaneceria 9.

A simetria anisotr\'opica freq\"uentemente usada \'e a
simetria {\bf hexagonal}. Este sistema tem uma simetria
mais elevada que um ortorr\^ombico. Este sistema tem um eixo
de simetria tal que a rota\cao\ por um \^angulo
arbitr\'ario ao redor deste eixo n\ao\ altera o tensor
$c_{ijkl}$. Isto significa que, em um plano perpendicular
a este eixo, o tensor comporta-se isotropicamente. Por
causa disto, a simetria \'e tamb\'em algumas vezes
chamada de isotropia transversal, especialmente em
casos em que o eixo de simetria rotacional coincide com o
eixo $x_3$ do sistema de coordenadas. A matriz
$C_{\alpha \beta}$ de um material sim\'etrico
hexagonalmente, com o eixo vertical de simetria, tem a
forma
\begin{eqnarray}
C_{\alpha \beta} = \left(
\begin{array}{cccccc}
C_{11} & C_{12} & C_{13} & 0      & 0      & 0 \\
       & C_{11} & C_{13} & 0      & 0      & 0 \\
       &        & C_{33} & 0      & 0      & 0 \\
       &        &        & C_{44} & 0      & 0 \\
       &        &        &        & C_{44} & 0 \\
       &        &        & & & \frac{C_{11}-C_{12}}{2}
\end{array} \right) \; .
\end{eqnarray}

A simetria hexagonal \'e descrita por 5 par\^ametros
el\'asticos independentes. Novamente o sistema de
coordenadas em que a matriz $C_{\alpha \beta}$ \'e
definida deve ser especificado. No caso acima, o eixo
$x_3$ coincide com o eixo de simetria rotacional, os
eixos $x_1$ e $x_2$ est\ao\ situados no plano de
isometria.

Algumas vezes, a ent\ao\ chamada nota\cao\ de Love para
cinco par\^ametros el\'asticos independentes de um meio
 isotr\'opico transversal \'e usada. Nele
\begin{eqnarray}
A=C_{11}, \;\;\;\; C=C_{33} \;\;\;\; L=C_{44} \;\;\;\;
N=C_{66}, \;\;\;\; F=C_{13} \; .
\end{eqnarray}

O meio {\bf isotr\'opico} possui a mais elevada simetria entre
os materiais anisotr\'opicos. Material isotr\'opico \'e
invariante para alguma rota\cao. Este \'e descrito por
dois par\^ametros el\'asticos independentes. A matriz
$C_{\alpha \beta}$ para um material isotr\'opico tem a
forma
\begin{eqnarray}
C_{\alpha \beta} = \left(
\begin{array}{cccccc}
C_{11} & C_{11}-2C_{44} & C_{11}-2C_{44} & 0 & 0 & 0 \\
       & C_{11} & C_{11}-2C_{44} & 0 & 0 & 0 \\
       &        & C_{11} & 0 & 0 & 0 \\
       &        &        & C_{44} & 0      & 0 \\
       &        &        &        & C_{44} & 0 \\
       &        &        &  &  & C_{44}
\end{array} \right) \; .
\end{eqnarray}

Em vez de $C_{11}$ e de $C_{44}$, os ent\ao\ chamados
par\^ametros de Lam\'e $\lambda$ e $\mu$ s\ao\ usados.
A matriz $C_{\alpha \beta}$ tem ent\ao\ a forma
\begin{eqnarray}
C_{\alpha \beta} = \left(
\begin{array}{cccccc}
\lambda+2\mu & \lambda  & \lambda & 0 & 0 & 0 \\
     & \lambda+2\mu & \lambda & 0  & 0 & 0 \\
     &       & \lambda+2\mu & 0 & 0 & 0 \\
     &       &        & \mu & 0 & 0 \\
     &       &        &     & \mu & 0 \\
     &       &        &     &     & \mu
\end{array} \right) \; .
\end{eqnarray}

A lei de Hooke pode agora ser reescrita como segue
\begin{eqnarray*}
\tau_{11} &=& \tau_1 = \lambda(e_1 + e_2 + e_3)+2\mu e_1=
\lambda \theta + 2 \mu e_{11} \; , \\
\tau_{22} &=& \tau_2 = \lambda(e_1 + e_2 + e_3)+2\mu e_2=
\lambda \theta + 2 \mu e_{22} \; , \\
\tau_{33} &=& \tau_3 = \lambda(e_1 + e_2 + e_3)+2\mu e_3=
\lambda \theta + 2 \mu e_{33} \; , \\
\tau_{23} &=& \tau_4 = \mu e_4 = 2 \mu e_{23} \; , \\
\tau_{13} &=& \tau_5 = \mu e_5 = 2 \mu e_{13} \; , \\
\tau_{12} &=& \tau_6 = \mu e_6 = 2 \mu e_{12} \; ,
\end{eqnarray*}
ou para uma equa\cao
\begin{eqnarray}
\tau_{ij} = \lambda \theta \delta_{ij} + 2 \mu e_{ij} \; ,
\end{eqnarray}
onde usamos $\theta = e_{ii}$.

Para o tensor de par\^ametros el\'asticos $c_{ijkl}$
chegamos (usando $2e_{ij} = e_{ij} + e_{ji}$)
\begin{eqnarray} \label{tpelas}
c_{ijkl} &=& \fracpp{\tau_{ij}}{e_{kl}} = \lambda
\fracpp{e_{mm}}{e_{kl}} \delta_{ij} + \mu \left(
\fracpp{e_{ij}}{e_{kl}} + \fracpp{e_{ji}}{e_{kl}}
\right) \nonumber \\
&=& \lambda \delta_{kl} \delta_{ij} + \mu (\delta_{ik}
\delta_{jl} + \delta_{jk} \delta_{il}) \; .
\end{eqnarray}

Vamos notar que a lei de Hooke para meio anisotr\'opico
e isotr\'opico pode tamb\'em ser reescrita em termos do
vetor deslocamento. Temos ent\ao
\begin{eqnarray}
\mbox{anisotr\'opico:} &\hspace{0.6cm}& \tau_{ij}=\frac{1}{2}
c_{ijkl} \left( \fracpp{u_k}{x_l} + \fracpp{u_l}{x_k}\right)
= c_{ijkl}\fracpp{u_k}{x_l} \; , \label{lhanis} \\
\mbox{isotr\'opico:} &\hspace{0.6cm}& \tau_{ij} = 
\lambda \theta \delta_{ij} + \mu \left( \fracpp{u_i}{x_j} +
\fracpp{u_j}{x_i} \right) \; , \label{lhiso}
\end{eqnarray} 
onde $\theta = \fracpp{u_k}{x_k}$.

Com as rela\coes\ (\ref{tpelas}) e (\ref{lhiso}), podemos
especificar a forma inversa da lei de Hooke para o caso
isotr\'opico,
\begin{eqnarray}
e_{ij} = s_{ijkl}\tau_{kl} \; ,
\end{eqnarray}
isto \'e, determinar a forma isotr\'opica do tensor
$s_{ijkl}$. Vamos primeiro determinar $\tau_{ii}$,
\begin{eqnarray}
\tau_{ii} = (3\lambda + 2\mu) \theta \; .
\end{eqnarray}
Assumindo que $3\lambda + 2\mu \neq 0$, podemos escrever
\begin{eqnarray}
\theta = \frac{\tau_{ii}}{3\lambda + 2\mu} \; .
\end{eqnarray} 
Se substituirmos na lei de Hooke, achamos para $e_{ij}$
\begin{eqnarray}
e_{ij} = \frac{1}{2\mu} \tau_{ij} - \lambda \delta_{ij}
\frac{\tau_{kk}}{2\mu(3\lambda + 2\mu)} \; .
\end{eqnarray}
Usando as identidades $\tau_{ij} = \delta_{ik}\delta_{jl}
\tau_{kl}$ e $\tau_{kk}=\delta_{kl} \tau_{kl}$, temos
similarmente
\begin{eqnarray}
e_{ij} = \left( \frac{\delta_{ik}\delta_{jl}}{2\mu} -
\frac{\lambda \delta_{ij} \delta_{kl}}{2\mu(3\lambda +
2\mu)} \right) \tau_{kl} \; .
\end{eqnarray}
Logo podemos escrever o tensor $s_{ijkl}$ em um meio isotr\'opico como
\begin{eqnarray}
s_{ijkl} = \frac{\delta_{ik}\delta_{jl}}{2\mu} -
\frac{\lambda \delta_{ij} \delta_{kl}}{2\mu(3\lambda +
2\mu)} \; .
\end{eqnarray}


Vamos agora escolher um sistema de coordenadas tal que
este coincide com o eixo principal do tensor de
deforma\cao. Ent\ao\ temos que $e_{12}=e_{13}=e_{23}=0$.
Da lei de Hooke para meio isotr\'opico chegamos
imediatamente que $\tau_{12}=\tau_{13}=\tau_{23}=0$.
Isto significa que nosso sistema de coordenadas tamb\'em
coincide com o eixo principal do tensor de deforma\cao.
Em um meio isotr\'opico, o eixo principal de um tensor
de deforma\cao\ coincide com o eixo principal do tensor
de tens\ao.

