
\section{Operadores Diferenciais e Teoremas Integrais }

\subsection {Coordenadas Cartesianas}

\begin{enumerate}

\item {\bf Operador Gradiente}\\
O operador gradiente pode ser escrito como
\begin{eqnarray}
\frac{\partial A}{\partial x_{i}} = A_{,i},
\end{eqnarray}
onde a v\ih rgula e o \ih ndice $i$ denotam a derivada com
rela\cao\ \`a coordenada espacial $x_{i}$. A nota\cao\ vetorial
deste operador pode ser $\nabla A$ ou grad$A$.

\item {\bf Operador Divergente}\\
Aplicado ao vetor $u_{i}$ o operador divergente pode ser escrito
como
\begin{eqnarray}
\frac{\partial u_{i}}{\partial x_{i}} = u_{i,i},
\end{eqnarray}
e sua nota\cao\ vetorial \'e $\nabla\!\cdot\!\vec{u}$ ou div$\vec{u}$.

\item {\bf Operador Rotacional}\\
Aplicado ao vetor $u_{i}$ o operador rotacional pode ser escrito
como
\begin{eqnarray}
\epsilon_{ijk}\frac{\partial u_{k}}{\partial x_{j}} =
\epsilon_{ijk}u_{k,j},
\end{eqnarray}
e sua nota\cao\ vetorial \'e $\nabla\!\times\!\vec{u}$, rot$\vec{u}$ ou
curl$\vec{u}$.

\item {\bf Operador de Laplace}\\
Pode ser escrito como uma combina\cao\ do operador divergente com
o gradiente. Pode ser escrito como
\begin{eqnarray}
\frac{\partial^{2}A}{\partial x_{i}^{2}} = A_{,ii},
\end{eqnarray}
e sua nota\cao\ vetorial \'e $\Delta A$ ou $\nabla^{2}A$.
\\

Note que $\mbox{div}\;\mbox{rot}\;\vec{u} = 0$, $\mbox{rot}\;\mbox{grad}\; A = 0$.

\item {\bf Teorema de Gauss}\\
Mostra como transformar uma integral de volume numa integral de
superf\ih cie e vice versa,
\begin{eqnarray}
\int\!\!\!\int\!\!\!\int_{V}\mbox{div}\vec{u}dV = \int\!\!\!\int_{\Sigma}\vec{u}\! \vec{n}dS
\end{eqnarray}
onde $V$ \'e um volume envolvido pela superf\ih cie fechada
$\Sigma$ e $\vec{n}$ \'e um vetor normal externo unit\'ario de
$\Sigma$.

\item {\bf Teorema de Stokes}\\
Indica como transformar uma integral de superf\ih cie em uma
integral de linha e vice versa,
\begin{eqnarray}
\int\!\!\!\int_{\Sigma}\vec{n}\cdot\mbox{rot}\vec{u}dS = \int_{L}\vec{u}\; \vec{t}dl,
\end{eqnarray}
onde $\Sigma$ \'e uma superf\ih cie envolta por uma linha fechada
$L$ com um vetor tangente unit\'ario $\vec{t}$.

\end{enumerate}

\subsection {Coordenadas Ortogonais Curvil\ih neas}

A rela\cao\ de transforma\cao\ entre coordenadas curvil\ih neas
$\gamma_{k}$, $k = 1,2,3$, e coordenadas cartesianas pode ser
escrita como
\begin{eqnarray}
x_{i} = x_{i}(\gamma_{k}).
\end{eqnarray}
Um elemento diferencial de comprimento ao longo de $x_{i}$ \'e
\begin{eqnarray}
dx_{i} = \frac{\partial x_{i}}{\partial \gamma_{k}}d\gamma_{k},
\end{eqnarray}
portanto temos
\begin{eqnarray}
(ds)^{2} = dx_{i}dx_{i} = \frac{\partial x_{i}}{\partial
\gamma_{k}}\frac{\partial x_{i}}{\partial
\gamma_{l}}d\gamma_{k}d\gamma_{l} = g_{kl}d\gamma_{k}d\gamma_{l},
\end{eqnarray}
onde $s$ \'e o comprimento da curva e
\begin{eqnarray}
g_{kl} = \frac{\partial x_{i}}{\partial \gamma_{k}}\frac{\partial
x_{i}}{\partial \gamma_{l}}.
\end{eqnarray}

Se $g_{kl} \neq 0$ somente para $k = l$ e $g_{kl} = 0$ se $k \neq
l$, temos coordenadas curvil\ih neas ortogonais. Isto significa que
o vetor $\partial x_{i}$/$ \partial \gamma_{k}$ que \'e tangente a
$\gamma_{k}$, \'e ortogonal a $\partial x_{i}$/$\partial
\gamma_{l}$, tangente \`a coordenada $\gamma_{l}$. Ent\ao\
$\gamma_{k}$ e $\gamma_{l}$ s\ao\ ortogonais em qualquer ponto.
Nessas coordenadas temos
\begin{eqnarray}
(ds)^{2} = h_{1}^{2}(d\gamma_{1})^{2} + h_{2}^{2}(d\gamma_{2})^{2}
+ h_{3}^{2}(d\gamma_{3})^{2},
\end{eqnarray}
onde
\begin{eqnarray}
h_{1}^{2} = \frac{\partial x_{1}}{\partial
\gamma_{1}}\frac{\partial x_{1}}{\partial \gamma_{1}} +
\frac{\partial x_{2}}{\partial \gamma_{1}}\frac{\partial
x_{2}}{\partial \gamma_{1}} + \frac{\partial x_{3}}{\partial
\gamma_{1}}\frac{\partial x_{3}}{\partial \gamma_{1}}.
\end{eqnarray}

Um diferencial de elemento ao longo de $\gamma_{i}$ \'e
\begin{eqnarray}
ds_{1} = h_{1}d\gamma_{1}.
\end{eqnarray}
Podemos escrever
\begin{eqnarray}
\frac{\partial}{\partial s_{1}} \Leftrightarrow
\frac{1}{h_{1}}\frac{\partial}{\partial\gamma_{1}}.
\end{eqnarray}
Assim o operador gradiente fica
\begin{eqnarray}
\nabla = \left(
\frac{1}{h_{1}}\frac{\partial}{\partial\gamma_{1}},
\frac{1}{h_{2}}\frac{\partial}{\partial\gamma_{2}},
\frac{1}{h_{3}}\frac{\partial}{\partial\gamma_{3}}\right)
\end{eqnarray}

Vejamos agora como fica a express\~ao do Laplaciano. Para isso usamos o Teorema de Gauss
\begin{eqnarray*}
\int\!\!\!\int\!\!\!\int_{V} u_{i,i}dV = \int\!\!\!\int_{S} u_{i}n_{i}dS
\end{eqnarray*}
onde $n_{i}$ \'e vetor normal externo unit\'ario da superf\ih cie $S$ que envolve o volume $V$. 
Considere um sistema de coordenadas ortogonais curvil\ih neas $\gamma_{1}$,$\gamma_{2}$ e $\gamma_{3}$, 
e um volume elementar $dV = ds_{1}ds_{2}ds_{3}$. As contribui\coes\ das superf\ih cies mais pr\'oximas aos eixos s\ao:
\begin{eqnarray*}
-u_{1}ds_{2}ds_{3}, \; -u_{2}ds_{1}ds_{3}, \; -u_{3}ds_{1}ds_{2},
\end{eqnarray*}
onde os sinais s\~ao negativos porque as orienta\coes\ dos vetores normais s\ao\ contr\'arias \`as dos eixos.
As contribui\coes\ das outras superf\ih cies s\ao:
\begin{eqnarray*}
u_{1}ds_{2}ds_{3}+\frac{\partial}{\partial s_{1}}(u_{1}ds_{2}ds_{3})ds_{1},\\
u_{2}ds_{1}ds_{3}+\frac{\partial}{\partial s_{2}}(u_{2}ds_{1}ds_{3})ds_{2},\\
u_{3}ds_{1}ds_{2}+\frac{\partial}{\partial s_{3}}(u_{3}ds_{1}ds_{2})ds_{3}.
\end{eqnarray*}

Somando as contribui\coes\ das seis faces e igualando \`a contribui\cao\ da integral de volume no Teorema de Gauss temos,
\begin{eqnarray*}
u_{i,i} ds_1 ds_2 ds_3 &=& \frac{1}{h_1}\frac{\partial}{\partial \gamma_1} (h_2 h_3 u_1) d\gamma_2 d\gamma_3 ds_1 
 + \frac{1}{h_{2}}\frac{\partial}{\partial \gamma_{2}}(h_{1}h_{3}u_{2})d\gamma_{1}d\gamma_{3}ds_{2} \\ 
&+&\frac{1}{h_{3}}\frac{\partial}{\partial \gamma_{3}}(h_{1}h_{2}u_{3})d\gamma_{1}d\gamma_{2}ds_{3}
\end{eqnarray*}

Usando $ds_{1} = h_{1}d\gamma_{1}$, etc., podemos escrever
\begin{eqnarray*}
u_{i,i}=\frac{1}{h_{1}h_{2}h_{3}}\left[\frac{\partial}{\partial\gamma_{1}}(h_{2}h_{3}u_{1}) 
       +\frac{\partial}{\partial \gamma_{2}}(h_{1}h_{3}u_{2})  
       +\frac{\partial}{\partial \gamma_{3}}(h_{1}h_{2}u_{3})\right].
\end{eqnarray*}

Combinando as express\~oes para o gradiente e para o divergente em coordenadas curvil\ih neas ortogonais obtemos a express\ao\ para o 
Laplaciano $\Delta = \frac{\partial^{2}}{\partial x_{i}^{2}}$
\begin{eqnarray}
\Delta = \frac{1}{h_1 h_2 h_3}\left[\frac{\partial}{\partial\gamma_1}\left(\frac{h_2 h_3}{h_1}\frac{\partial}{\partial\gamma_1}\right)
+\frac{\partial}{\partial\gamma_{2}}\left(\frac{h_{1}h_{3}}{h_{2}}\frac{\partial}{\partial\gamma_{2}}\right)+
 \frac{\partial}{\partial\gamma_{3}}\left(\frac{h_{1}h_{2}}{h_{3}}\frac{\partial}{\partial\gamma_{3}}\right)\right].
\end{eqnarray}

\section{Sinal Anal\ih tico}

A fun\cao\ sinal anal\ih tico $F(\xi)$ \'e definida como
\begin{eqnarray}
F(\xi) = g(\xi) + ih(\xi)
\end{eqnarray}
e ser\'a usada no estudo de ondas transientes. Aqui $g(\xi)$ \'e o
sinal transiente real para o qual o sinal anal\ih tico \'e
constru\ih do e $h(\xi)$ \'e a transformada de Hilbert de
$g(\xi)$,
\begin{eqnarray}
h(\xi) =
\frac{1}{\pi}\int_{-\infty}^{\infty}\frac{g(\sigma)}{\sigma-\xi}d\sigma,
\end{eqnarray}
onde $g(\xi)$ e $h(\xi)$ formam um par da transformada de Hilbert.
Antes de entrar na discuss\ao\ do sinal anal\ih tico na teoria de
propaga\cao\ de ondas planas transientes, vejamos alguns fatos
importantes da an\'alise de Fourier. Considerando o grupo de
fun\coes\ de Fourier do tipo padr\ao, que s\ao\ as absolutamente
integr\'aveis em $(-\infty,\infty)$ e satisfazem as condi\coes\ de
Dirichlet num intervalo finito. Uma fun\cao\ \'e dita
absolutamente integr\'avel se
\begin{eqnarray}
\int_{-\infty}^{\infty}\mid g(t)\mid dt \leq A,
\end{eqnarray}
onde $A$ \'e uma constante real positiva. A condi\cao\ de
Dirichlet requer a continuidade de $g(\xi)$ em um intervalo finito
com a possibilidade de finitas descontinuidades do primeiro tipo
(onde existem limites finitos \`a esquerda e \`a direita), e
n\'umero finito de m\'aximos e m\ih nimos. Sob essas codi\coes\ a
transformada de Fourier \'e definida como
\begin{eqnarray}
G(f) = \int_{-\infty}^{\infty}g(t)e^{i2\pi ft}dt,\; g(t) =
\int_{-\infty}^{\infty}G(f)e^{-i2\pi ft}df
\end{eqnarray}

Seja a fun\cao
\begin{eqnarray}
c(t) = \int_{-\infty}^{\infty} a(\tau)b(t-\tau)d\tau,
\end{eqnarray}
que tamb\'em \'e escrita como $c(t) = a(t)*b(t)$. O teorema da
convolu\cao\ diz que a transformada de Fourier de uma convolu\cao\
\'e igual ao produto da transformada das fun\coes\ $a(t)$ e
$b(t)$,
\begin{eqnarray}
c(f) = a(f)b(f).
\end{eqnarray}

A fun\cao\ sinal \'e definida como
\begin{eqnarray}
\sgn(f) = \left\{
\begin{array}{ll}
1 , f > 0 \\
0 , f = 0 \\
-1 , f < 0
\end{array}
\right.
\end{eqnarray}
Como esta fun\cao\ n\ao\ \'e absolutamente integr\'avel, \'e
preciso achar a inversa da transformada de Fourier como um caso
limite de uma fun\cao\ auxiliar que no limite se aproxima da
fun\cao\ $\sgn(f)$. Esta fun\cao\ pode ser
\begin{eqnarray}
g_{\alpha}(f) = e^{-\alpha \mid f\mid}i \sgn(f)
\end{eqnarray}
com $\alpha > 0$. Esta \'e uma fun\cao\ integr\'avel e quando
$\alpha\rightarrow 0$ a fun\cao\ se aproxima de $i$ $\sgn(f)$.
Ent\ao\ basta tomar a transformada de Fourier inversa e fazer
$\alpha\rightarrow 0$ para obter a transformada inversa de $i\sgn(f)$.
\begin{eqnarray*}
g_{\alpha}(t) &=& \int_{-\infty}^{\infty}e^{-\alpha \mid
f\mid}i\sgn(f)e^{-i2\pi ft}df \\
&=& -\int_{-\infty}^{0}e^{\alpha f}e^{-i2\pi ft}i df +
\int_{0}^{\infty}e^{-\alpha f}e^{-i2\pi ft}i df \\
&=& -\int_{0}^{\infty}e^{-\alpha f^{'}}e^{i2\pi f^{'}t}i df^{'} +
\int_{0}^{\infty}e^{-(\alpha+i2\pi t)f}i df \\
&=& -\int_{0}^{\infty}e^{(-\alpha+2i\pi t)f}i df +
\int_{0}^{\infty}e^{-(\alpha + 2i\pi t)f}i df \\
&=& \frac{i}{i2\pi t - \alpha} + \frac{i}{i2\pi t + \alpha}.
\end{eqnarray*}
Para $\alpha\rightarrow 0$ temos
\begin{eqnarray}
g_{\alpha}(t)\rightarrow g_{0}(t) = \int_{-\infty}^{\infty}i\sgn(f)e^{-i2\pi ft}dt =
\frac{1}{\pi t}.
\end{eqnarray}
Temos ent\ao\ um par de Fourier
\begin{eqnarray}
g_{0}(t) = (\pi t)^{-1},\; G_{0}(f) = i\sgn(f).
\end{eqnarray}

Sabemos que, para uma fun\cao\ real $g(t)$
\begin{eqnarray}
G^{*}(f) = \int_{-\infty}^{\infty}g(t)e^{-i2\pi ft}dt = G(-f).
\end{eqnarray}

Usando esta propriedade reescrevemos a expressao para o sinal real
da seguinte maneira
\begin{eqnarray}
g(t) &=& \int_{-\infty}^{\infty}G(f)e^{-i2\pi ft}df \\
&=& \int_{-\infty}^{0}G(f)e^{-i2\pi ft}df +
\int_{0}^{\infty}G(f)e^{-i2\pi ft}df \\
&=& \int_{0}^{\infty}G(-f^{'})e^{-i2\pi f^{'}t}df +
\int_{0}^{\infty}
G(f)e^{-i2\pi ft}df \\
&=& \int_{0}^{\infty}G^{*}(f)e^{i2\pi ft}df +
\int_{0}^{\infty}G(f)e^{-i2\pi ft}df \\
&=& 2Re\int_{0}^{\infty}G(f)e^{-i2\pi ft}df.
\end{eqnarray}

Ent\ao,
\begin{eqnarray}
g(t) = 2Re\int_{0}^{\infty}G(f)e^{-i2\pi ft}df
\end{eqnarray}

Introduzindo o sinal complexo
\begin{eqnarray}
F(t) = 2\int_{0}^{\infty}G(f)e^{-i2\pi ft}df,
\end{eqnarray}
que pode ser escrito como
\begin{eqnarray}
F(t) = g(t) + ih(t),
\label{sinan}
\end{eqnarray}
onde
\begin{eqnarray}
h(t) = 2Im\int_{0}^{\infty}G(f)e^{-i2\pi ft}df.
\end{eqnarray}
Esta express\ao\ pode ser reescrita da seguinte maneira
\begin{eqnarray}
h(t) &=& -i\left[\int_{0}^{\infty}G(f)e^{-i2\pi ft}df -
\int_{0}^{\infty}G^{*}(f)e^{i2\pi ft}df  \right]\nonumber \\
&=& -i\left[\int_{0}^{\infty}G(f)e^{-i2\pi ft}df -
\int_{-\infty}^{0}G(f^{'})e^{-i2\pi f^{'}t}df^{'}  \right]\nonumber \\
&=& -i\left[\int_{-\infty}^{\infty}G(f)\sgn(f)e^{-i2\pi ft}df
\right]\nonumber \\
&=& -g(t) * \frac{1}{\pi t}\nonumber \\
&=&
-\frac{1}{\pi}\int_{-\infty}^{\infty}\frac{g(\sigma)}{t-\sigma}d\sigma
\nonumber \\
&=& \frac{1}{\pi}\int_{-\infty}^{\infty}\frac{g(\sigma)}{\sigma -
t}d\sigma.
\label{thil}
\end{eqnarray}

Ent\ao\ $g(t)$ e $h(t)$ formam um par de Hilbert, isto \'e, $h(t)$ \'e
a transformada de Hilbert de $g(t)$. O sinal complexo (\ref{sinan})
\'e o sinal anal\ih tico.

Devemos observar que a Transformada de Hilbert da Transformada de
Hilbert de um sinal $g(t)$ fornece $-g(t)$, ou seja,
\begin{equation}
g(t) = - \frac{1}{\pi} \int_{-\infty}^{\infty} \frac{h(\xi)}{\xi-t}d\xi
= \frac{1}{\pi} \int_{-\infty}^{\infty} \frac{h(\xi)}{t-\xi}d\xi \; .
\label{thth}
\end{equation}
Isso pode ser provado usando o sinal anal\itico\ $Y(t) = -iF(t) = -i
(g(t) + ih(t)) = h(t) - i g(t)$. Usando o mesmo racioc\inio\ das equa\coes\
(\ref{thil}), mostra-se que $-g(t)=Im \{Y(t)\}$ \'e a transformada de
Hilbert de $h(t)=Re\{Y(t)\}$.

Um importante par de Hilbert e seu correspondente sinal anal\ih
tico est\'a relacionado ao sinal real na forma da fun\cao\
$\delta$ de Dirac. Para $g(t) = \delta(t)$, a equa\cao\ para
$h(t)$ fica
\begin{eqnarray}
h(t) = \int_{-\infty}^{\infty}\frac{\delta(\sigma)}{\sigma - t}d\delta
= -\frac{1}{\pi t}.
\end{eqnarray}
O correspondente sinal anal\ih tico tem a forma
\begin{eqnarray}
F(t) = \delta(t) - \frac{i}{\pi t}.
\end{eqnarray}

Uma propriedade importante da transformada de Hilbert \'e que a
transformada de Hilbert da derivada $dg(t)/dt$ \'e a derivada
$dh(t)/dt$ da transformada de Hilbert do sinal $g(t)$.
\begin{eqnarray}
\frac{dh(t)}{dt} &=& 2Im\int_{0}^{\infty}g(f)(-i2\pi f)e^{-i2\pi
ft}df, \\
\frac{dg(t)}{dt} &=& 2Re\int_{0}^{\infty}g(f)(-i2\pi f)e^{-i2\pi
ft}df.
\end{eqnarray}
Da primeira equa\cao\ vemos que a derivada da transformada de
Hilbert $h(t)$ corresponde a um "novo" sinal cuja transformada de
Fourier \'e $g(f)(-i2\pi f)$. Da segunda equa\cao\ vemos que o
"novo" sinal \'e $dg(t)/dt$.


%\begin{thebibliography}{99}
%\bibitem{Psencik} P\v{s}en\v{c}ik, I., 1994, {\em Introduction to seismic
%methods - Lecture Notes}, PPPG / UFBa.
%\end{thebibliography}

