\subsection{A solu\c{c}\~ao da Equa\c{c}\~ao da Onda Ac\'ustica
unidimensional}

Nesta se\c{c}\~ao, resolvemos a equa\c{c}\~ao da onda ac\'ustica em 1D
sem fontes para um meio homog\^eneo, i.e., com densidade e velocidade
constantes, a saber
\begin{equation}
\frac{\partial^2 u}{\partial x^2} - \frac{1}{c^2}
\frac{\partial^2 u}{\partial t^2} = 0 \ .
\label{ondaacu1d}
\end{equation}

Para poder resolver uma equa\c{c}\~ao diferencial parcial de segunda
ordem, necessitamos de condi\c{c}\~oes iniciais.
Devemos considerar, inicialmente, as condi\c{c}\~oes em $t=0$
\begin{equation}
u(x,0)=u_0 (x) \qquad \mbox{e}\qquad  \frac{\partial u}{\partial t}(x,0)=v_0 (x).
\end{equation}
Agora, resolvemos a EDP pelo m\'etodo de mudan\c{c}as de vari\'aveis,
isto \'e, definiremos vari\'aveis independentes entre si, como
fun\c{c}\~oes de $x$ e $t$
\begin{equation}
 \xi =\xi (x,t) \qquad\mbox{e}\qquad \mu =\mu (x,t).
\end{equation}
Particularmente,
\begin{equation}
 \xi (x,t)=x-ct \qquad\mbox{e}\qquad \mu (x,t)=x+ct,
\end{equation}
com seus respectivos operadores diferenciais
\begin{equation}
 \frac{\partial}{\partial x}=\frac{\partial \xi}{\partial x}\frac{\partial}{\partial  \xi}+
\frac{\partial \mu}{\partial x}\frac{\partial}{\partial \mu}=\frac{\partial}{\partial  \xi}+
\frac{\partial}{\partial \mu},
\end{equation}
pois $\frac{\partial \xi}{\partial x}=\frac{\partial \mu}{\partial x}=1$.

Derivando esta rela\c{c}\~ao novamente com respeito a $x$, obtemos para
a segunda derivada
\begin{equation}\label{opx}
 \frac{\partial ^2}{\partial x^2}=\frac{\partial ^2}{\partial \xi ^2}+2\frac{\partial  ^2}
{\partial \xi \partial \mu}+\frac{\partial ^2}{\partial \mu ^2}.
\end{equation}

Analogamente em rela\c{c}\~ao a $t$
\begin{equation}
 \frac{\partial}{\partial t}=-c\frac{\partial}{\partial \xi}+c\frac{\partial}{\partial \mu},
\end{equation}
e, derivando novamente,
\begin{equation}\label{opt}
 \frac{1}{c^2}\frac{\partial ^2}{\partial t^2}=\frac{\partial ^2}
{\partial \xi  ^2}-2\frac{\partial ^2}{\partial \xi \partial \mu}+\frac{\partial ^2}
{\partial \mu ^2}.
\end{equation}
Substituindo~(\ref{opx}) e~(\ref{opt}) na EDP da onda~(\ref{ondaacu1d}), temos,
\begin{equation}
 \frac{\partial ^2u}{\partial \xi \partial \mu}=0.
\end{equation}
Logo, $\frac{\partial u}{\partial \mu}$ deve ser independente de $\xi$. Ent\~ao podemos 
definir
\begin{equation}
 \frac{\partial u}{\partial \mu}=\Phi (\mu),
\end{equation}
onde $\Phi (\mu)$ \'e uma fun\c{c}\~ao arbitr\'aria de $\mu$. Integrando com respeito a $\mu$
\begin{equation}\label{integral}
 u(\xi,\mu)=(\int^\mu \Phi (\mu ')d\mu ')+f(\xi),
\end{equation}
onde $f(\xi)$ \'e uma fun\c{c}\~ao arbitr\'aria de $\xi$.  Denominando a integral indefinida 
em ~(\ref{integral}) como $g(\mu )$ temos
\begin{equation}
 u(\xi,\mu)=f(\xi)+g(\mu),
\end{equation}
em termos dessas vari\'aveis j\'a definidas
\begin{equation}\label{peqonda}
 u(x,t)=f(x-ct)+g(x+ct),
\end{equation}
obtivemos a solu\c{c}\~ao geral da equa\c{c}\~ao (\ref{ondaacu1d}). Esta
forma revela a natureza f\'{\i}sica das solu\c{c}\~oes. Aqui, o termo
$f(x-ct)$ representa o deslocamento na dire\c{c}\~ao de $x$ positivo,
com velocidade $c$. E similarmente, o termo $g(x+ct)$ representa o
deslocamento na dire\c{c}\~ao oposta, com a mesma velocidade $c$. Ou
seja, a solu\c{c}\~ao da equa\c{c}\~ao da onda com velocidade e
densidade constantes \'e uma superposi\c{c}\~ao de uma onda com qualquer
formato propagando para a direita com uma outra onda de qualquer
formato, possivelmente diferente, propagando para a esquerda.

A quest\~ao \'e determinar as fun\c{c}\~oes $f$ e $g$ para satisfazerem
as condi\c{c}\~oes iniciais. Em $t=0$, temos
\begin{equation}
 u_0 (x)=f(x)+g(x)    \qquad   v_0 (x)=c\left[\frac{dg}{dx}-\frac{df}{dx}\right].
 \label{condini}
\end{equation}

Podemos, agora, integrar $v_0 (x)$. O limite inferior dessa
integra\c{c}\~ao \'e uma constante $a$ indeterminada, que poder ser
adicionada \`a $f$ e subtra\'{\i}da da $g$ sem afetar a solu\c{c}\~ao.
Assim, as fun\c{c}\~oes $f$ e $g$ n\~ao s\~ao \'unicas. Ao integrar a
direita das equa\c{c}\~oes (\ref{condini}, obtemos
\begin{equation}
 c[g(x)-f(x)]=\int_{a}^{x} v_0 (x')dx'.
\end{equation}
Resolvendo o conjunto dessa equa\c{c}\~ao com a da esquerda de
(\ref{condini}) para $f(x)$ e $g(x)$, encontramos
\begin{equation}
 f(x)=\frac{1}{2}u_0 (x)-\frac{1}{2c}\int_{a}^{x} v_0 (x')dx'
\end{equation}
\begin{equation}
 g(x)=\frac{1}{2}u_0 (x)+\frac{1}{2c}\int_{a}^{x} v_0 (x')dx'.
\end{equation}
Substituindo $f(x)$ por $f(x-ct)$ e $g(x)$ por $g(x+ct)$ e combinando de
acordo com~(\ref{peqonda}), temos
\begin{equation}
u(x,t)=\frac{1}{2}[u_0 (x-ct)+u_0 (x+ct)]+\frac{1}{2c}\int_{x-ct}^{x+ct} v_0 (x')dx'.
\end{equation}
Esta f\'ormula, que representa a solu\c{c}\~ao geral do problema, \'e
conhecida como a {\it F\'ormula de D'Alembert}. Ela mostra que dadas
condi\c{c}\~oes iniciais, a solu\c{c}\~ao deve ter a forma
$f(x-ct)+g(x+ct)$. Podemos dizer, ent\~ao que o comportamento
f\'{\i}sico do deslocamento da onda, j\'a descrito, \'e aplicado a esta
equa\c{c}\~ao obtida.

