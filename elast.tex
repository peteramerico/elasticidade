
\section{Teoria Elastodin\^amica}

\subsection{Introdu\cao}

Sob uma carga externa, um corpo pode ser transladado,
rotacionado ou deformado. Na seq\"u\^encia, estudamos
o efeito chamado de ``deforma\cao'', pois estamos
trabalhando pontualmente e assim n\ao\ \'e poss\ih vel
que haja transla\cao\ ou rota\cao.

A deforma\cao\ de um material \'e um processo no qual
dist\^ancias entre pontos individuais do material s\ao\
alteradas. Em um material real\ih stico, a aplica\cao\
de uma for\ca\ em um lugar particular causa deforma\coes,
primeiro nas proximidades deste lugar e sucessivamente em
partes mais distantes. Este processo \'e chamado de
``propaga\cao\ de onda''.

A propaga\cao\ de onda deve superar a resit\^encia do
material causada pela consist\^encia e a resist\^encia
causada pela in\'ercia. Se a consist\^encia \'e tal que o
material \'e n\ao\ deform\'avel (r\ih gido), o efeito da
aplica\cao\ de for\cas\ externas em um ponto do corpo
poderia ser sentido imediatamente em cada ponto do
material, i.e., o corpo seria movido. Se o material \'e
deform\'avel, ent\ao\ todas as part\ih culas do material
seriam excitadas simultaneamente. N\'os consideramos aqui
um material real\ih stico que \'e deform\'avel e que,
depois de removida a carga, retorna para um estado, que
\'e o mesmo ou similar ao estado antes do carregamento.
Neste \'ultimo caso n\'os falamos sobre materiais
``imperfeitamente el\'asticos'', e no caso anterior sobre
materias ``perfeitamente el\'asticos'', no qual
concentramos nossos estudos.

Uma propaga\cao\ de onda \'e conectada por transmiss\ao\
de energia. A energia \'e transportada de uma part\ih cula
do material para outra, e este transporte n\ao\ \'e pelo
fluxo das part\ih culas. As part\ih culas do material
oscilam ao redor de suas posi\coes\ m\'edias.

Na mec\^anica cl\'assica, o movimento de onda \'e assumido
ser pequenas pertuba\coes\ no estado inicial do material,
chamado de ``estado natural''. No estado inicial n\ao\ h\'a
tens\oes\ e deforma\coes, ent\ao\ consideramos o estado
inicial sendo o estado de equil\ih brio est\'atico em que a
deforma\cao\ n\ao\ muda com o tempo.

Estudamos aqui como a deforma\cao\ altera rela\coes\ entre
duas part\ih culas pr\'oximas da m\'edia. Esta mudan\ca\
\'e descrita pelo ``tensor de deforma\cao''. Antes
introduzimos o {\it vetor deslocamento} descrevendo um
deslocamento de uma part\ih cula simples sob uma
deforma\cao. Consideramos para ambos o sistema de
coordenadas $x_1$, $x_2$, $x_3$ com origem no ponto 0.


\subsection{Vetor deslocamento}

Consideramos uma part\ih cula, que, no estado
inicial, est\'a situada no ponto $P[x_1,x_2,x_3]$. Na deforma\cao\
do material no tempo $t$, esta part\ih cula ser\'a movida
para o ponto $P'[x_1',x_2',x_3']$. O movimento do ponto $P$
para $P'$ \'e especificado pelo ``vetor deslocamento''
$u_i(x_k,t)$ tal que
\begin{eqnarray}
x_i' = x_i + u_i(x_k,t) = x_i + u_i(P,t) \; .
\end{eqnarray}
Aqui, o vetor deslocamento \'e especificado como uma
fun\cao\ vetorial de coordenadas $x_i$ do ponto $P$ (a
posi\cao\ da part\ih cula no estado inicial) e do tempo
$t$. Esta descri\cao\ \'e extremamente usada em
sismologia e prospec\cao\ s\ih smica e \'e conhecida como
``descri\cao\ Lagrangeana''. Nela, o sistema de coordenadas usado para
descrever o movimento \'e fixo e a part\icula\ se move nele.

Consideramos, agora, o vetor deslocamento em fun\cao\
das coordenadas $x_i'$ do ponto $P'$ (posi\cao\ da
part\ih cula no estado deformado) e do tempo $t$,
$u_i(x_k',t)$. Em tal caso, $u_i$ depende do tempo
atrav\'es das coordenadas $x_k'$, que est\ao\ dependendo do tempo. Esta
descri\cao\ \'e conhecida como
``descri\cao\ de Euler'' e esta \'e mais usada em
hidrodin\^amica. Nela, o sistema de coordenadas usado para
descrever o movimento acompanha a part\icula\ e esta, portanto tem
posi\cao\ fixa nele. Aqui, usamos a descri\cao\ Lagrangeana.

Na descri\cao\ Lagrangeana, a velocidade de movimento da
part\ih cula pode ser simplismente obtida por deriva\cao\
parcial de $u_i(x_k,t)$ com respeito ao tempo $t$
\begin{eqnarray}
v_i(x_k,t) = \dot{u}_i(x_k,t) = \frac{\partial u_i(x_k,t)}
{\partial t} \; .
\end{eqnarray}

Similarmente, para a acelera\cao\ da part\ih cula chegamos em
\begin{eqnarray}
a_i(x_k,t) = \ddot{u}_i(x_k,t) = \frac{\partial^2 u_i(x_k,t)}
{\partial t^2} \; .
\end{eqnarray}

A quantidade $v_i(x_k,t)$ \'e conhecida como ``velocidade
da part\ih cula'' e $a_i(x_k,t)$ como a ``acelera\cao\ da
part\ih cula''.


\subsection{Tensor de deforma\cao}

Em adi\cao\ ao ponto $P$, n\'os consideramos agora um ponto
$Q[x_1+dx_1,x_2+dx_2,x_3+dx_3]$ situado nas proximidades do
ponto $P$, estando o material ainda n\ao\ deformado. O
ponto $Q$ ser\'a movido durante a deforma\cao\ para o ponto
$Q'[x_1'+dx_1',x_2'+dx_2',x_3'+dx_3']$. O vetor
deslocamento do ponto $Q$ pode ser expresso em termos do
vetor deslocamento $u_i(P)$. Segue, por aproxima\cao\ de Taylor de
primeira ordem (uma vez que $dx_i$ \'e supostamente infinitesimalmente
pequeno), que
\begin{eqnarray} \label{vdq}
u_i(Q) = u_i(x_k+dx_k) \sim u_i(x_k) +
\frac{\partial u_i(x_k)}
{\partial x_j} dx_j = u_i(P) + \frac{\partial u_i(P)}
{\partial x_j} dx_j \; ,
\end{eqnarray}
para todos os pontos $Q$ na vizinhan\ca\ de $P$.

Agora investigamos a mudan\ca\ da dist\^ancia entre os pontos
$P$ e $Q$ devido \`a deforma\cao. Para isso fazemos a
compara\cao\ do quadrado das dist\^ancias $\overline{PQ}$
e $\overline{P'Q'}$.
Para $\overline{PQ}^2$, n\'os temos
\begin{eqnarray}
\overline{PQ}^2 = dx_i \; dx_i \; .
\end{eqnarray}
Para determinarmos $dx_i'$, observamos que
\begin{eqnarray}
dx_i' = dx_i + u_i(Q) - u_i(P) \sim dx_i +
\frac{\partial u_i(P)}{\partial x_j} dx_j \; .
\label{dxidxi}
\end{eqnarray}

Logo, para $\overline{P'Q'}^2$, temos
\begin{eqnarray}
\overline{P'Q'}^2 = dx_i' \; dx_i' &\sim& \left(dx_i +
\frac{\partial u_i}{\partial x_j} dx_j\right) \left(dx_i +
\frac{\partial u_i}{\partial x_k} dx_k\right) \\
&=& dx_i dx_i +  \frac{\partial u_i}{\partial x_j} dx_j
dx_i + \frac{\partial u_i}{\partial x_k} dx_k  dx_i
+ \frac{\partial u_i}{\partial x_k} \frac{\partial u_i}
{\partial x_j} dx_j dx_k \\
&=& dx_i dx_i + \left( \frac{\partial u_k}{\partial x_j} +
\frac{\partial u_j}{\partial x_k} + \frac{\partial u_i}
{\partial x_j} \frac{\partial u_i}{\partial x_k}\right)
dx_j dx_k \; .
\end{eqnarray}

Para $\overline{P'Q'}^2 - \overline{PQ}^2$, chegamos em
\begin{eqnarray}
\overline{P'Q'}^2 - \overline{PQ}^2 \sim 2 E_{jk}
dx_j dx_k \; ,
\end{eqnarray}
onde
\begin{eqnarray} \label{tensorE}
E_{jk} = \frac{1}{2} \left( \frac{\partial u_k}
{\partial x_j} + \frac{\partial u_j}{\partial x_k} +
\frac{\partial u_i}{\partial x_j} \frac{\partial u_i}
{\partial x_k}\right) \; .
\end{eqnarray}


A quantidade $E_{jk}$ \'e um tensor de segunda ordem.
Este \'e chamado de ``tensor de deforma\cao\ finita''.
Desde que todas as derivadas para $E_{jk}$ sejam tomadas
de $P$, o tensor (\ref{tensorE}) caracteriza a
deforma\cao\ nas proximidades do ponto $P$. O tensor
$E_{jk}$ \'e sim\'etrico ($E_{jk} = E_{kj}$) e ent\ao\ \'e
especificado por somente 6 componentes independentes. Por
causa do terceiro termo em (\ref{tensorE}), o tensor de
deforma\cao\ finita \'e considerado n\ao\ linear.

Mas, aqui, consideramos somente processos de ondas
em que a deforma\cao\ seja pequena, isto \'e,
\begin{eqnarray}
\left|\frac{\partial u_i}{\partial x_j}\right| \ll 1 \; .
\label{def_peq}
\end{eqnarray}
Logo, \'e poss\ih vel desconsiderar o termo n\ao\ linear
na express\ao\ (\ref{tensorE}) desde que \'e da segunda
ondem com respeito a  $|\partial u_i / \partial x_j|$.
Apesar desta lineariza\cao\ simplificar as opera\coes\
matem\'aticas devemos manter a aproxima\cao\ acima sempre
em mente nas aplica\coes.

N\'os denotaremos a lineariza\cao\ do tensor de
deforma\cao\ finita, $e_{jk}$, como
\begin{eqnarray} \label{td}
e_{jk} = \frac{1}{2} \left( \frac{\partial u_j}
{\partial x_k} +\frac{\partial u_k}{\partial x_j}\right)
\; ,
\end{eqnarray}
que \'e comumente chamado de ``tensor de deforma\cao\
pequena'' ou abreviadamente ``tensor de deforma\cao''.
Este \'e sim\'etrico e linear. Notamos tamb\'em que o
tensor de deforma\cao\ pequena $e_{ik}$ tem a mesma forma
Lagrangeana e de formula\cao\ de Euler.

Em decorr\^encia da f\'ormula (\ref{td}) obtida para o
tensor de deforma\cao\ podemos escrever a express\ao\
(\ref{vdq}) na forma
\begin{eqnarray} \label{vdq2}
u_i(Q) = u_i(P) + \frac{1}{2}\left( \frac{\partial u_i}
{\partial x_j} + \frac{\partial u_j}{\partial x_i}\right)
dx_j + \frac{1}{2}\left( \frac{\partial u_i}{\partial x_j}
- \frac{\partial u_j}{\partial x_i}\right)dx_j \; ,
\end{eqnarray}
onde temos que o primeiro termo do lado direito ($u_i(P)$)
pode ser identificado como transla\cao, uma vez que \'e
igual para todos os pontos na vizinhan\ca\ de $P$. O segundo
termo cont\'em o conhecido tensor de deforma\cao\ e descreve, portanto a
deforma\cao\ do meio. Mostramos a seguir que
o terceiro termo  representa uma rota\cao.
Os elementos $\xi_{ij} = \frac{1}{2} \left(\frac{\partial u_i}{\partial x_j} -
\frac{\partial u_j}{\partial x_i}\right)$ deste termo formam a matriz
anti-sim\'etrica $\Xi$, conhecida como o tensor de rota\cao.

\subsection{Tensor de rota\cao}

Vejamos como o termo $\left( \Xi d\vec{x} \right)_i=
\frac{1}{2} \left( \frac{\partial u_i}{\partial x_j} -
\frac{\partial u_j}{\partial x_i}\right) dx_j = \xi_{ij}
dx_j$ representa uma rota\cao.
Para tal, observamos que o produto
\begin{eqnarray}
\Xi d\vec{x} = \left(
\begin{array}{ccc}
0 & \xi_{12} & \xi_{13} \\
\xi_{21} & 0 & \xi_{23} \\
\xi_{31} & \xi_{32} & 0
\end{array} \right)
\left(
\begin{array}{c}
dx_1 \\
dx_2 \\
dx_3
\end{array} \right)
= \left(
\begin{array}{c}
\xi_{12} dx_2 + \xi_{13} dx_3 \\
\xi_{21} dx_1 + \xi_{23} dx_3 \\
\xi_{31} dx_1 + \xi_{32} dx_2
\end{array}
\right) \;
\end{eqnarray}
pode ser representado como produto vetorial de um vetor
$\vec{a}$  com $d\vec{x}$:
\begin{eqnarray}
\vec{a} \times d\vec{x} = \left|
\begin{array}{ccc}
i & j & k \\
a_1 & a_2 & a_3 \\
dx_1 & dx_2 & dx_3
\end{array} \right|
= \left(
\begin{array}{c}
a_2 dx_3 - a_3 dx_2 \\
-a_1 dx_3 + a_3 dx_1 \\
a_1 dx_2 - a_2 dx_1
\end{array}
\right) \; .
\end{eqnarray}

Basta identificarmos:
\begin{eqnarray}
\left\{
\begin{array}{ccccc}
a_1 & = & -\xi_{23} & = & \xi_{32} \\
a_2 & = & \xi_{13} & = & -\xi_{31} \\
a_3 & = & -\xi_{12} & = & \xi_{21}
\end{array} \right.
\end{eqnarray}
que reflete corretamente a anti-simetria de $\Xi$.
Observamos ainda que
\begin{eqnarray}
\vec{a} = \left(
\begin{array}{c}
\xi_{32} \\
\xi_{13} \\
\xi_{21}
\end{array} \right)
= \frac{1}{2}  \left(
\begin{array}{c}
\frac{\partial u_3}{\partial x_2}-\frac{\partial u_2}
{\partial x_3}\\
\frac{\partial u_1}{\partial x_3}-\frac{\partial u_3}
{\partial x_1}\\
\frac{\partial u_2}{\partial x_1}-\frac{\partial u_1}
{\partial x_2}
\end{array} \right)
= \frac{1}{2} \; \mbox{rot} \; \vec{u} \; .
\end{eqnarray}
Portanto, $\Xi d\vec{x} = \frac{1}{2} \; \mbox{rot}
\;\vec{u} \times d\vec{x}$.

Assim, quando n\ao\ temos nem transla\cao\ ($u_i(P)=0$) nem deforma\cao\
($e_{ij}=0$), a equa\cao\ (\ref{vdq2}) pode ser escrita como
\begin{eqnarray}
u_i(Q) = \xi_{ij} dx_j \; \; \; \Longrightarrow \; \; \;
\vec{u}(Q) = \frac{1}{2} \;\mbox{rot}\; \vec{u}\times
d\vec{x} \; ,
\end{eqnarray}
que \'e perpendicular a $d\vec{x}$.
Logo, neste caso a equa\cao\ (\ref{dxidxi}
tem a forma
\begin{eqnarray}
d\vec{x}' = d\vec{x} + \frac{1}{2} \;\mbox{rot}\; \vec{u}\times
d\vec{x} \; .
\end{eqnarray}
No tri\^angulo $PQQ'$, observamos, para $|\partial u_i/\partial x_j|\ll 1$,
que $\overline{QQ'} = \tan \alpha |d\vec{x}| \approx
\alpha |d\vec{x}|$, onde $\alpha$ denota o \^angulo entre $PQ$ e $PQ'$.
Mas, por outro lado, $\overline{QQ'} = |\vec{u}(Q)|=
\mbox{$|\frac{1}{2} \; \mbox{rot} \; \vec{u} \times
d\vec{x}|$} = |\frac{1}{2} \; \mbox{rot} \; \vec{u}|
|d\vec{x}| \sin \frac{\pi}{2}$. Logo, $\alpha \approx
|\frac{1}{2} \mbox{rot} \; \vec{u}|$. 

Desta forma, vimos que o \^angulo $\alpha$ que a desloca\cao\ de $Q$
para $Q'$ cobra depende somente do rotacional do vetor deslocamento. Em
particular, n\ao\ depende da posi\cao\ original de $Q$ em rela\cao\ a
$P$, descrita por $dx_i$. Isso implica que o \^angulo $\alpha$ \'e o
mesmo para todos os pontos $Q$ na vizinhan\ca\ de $P$.
Conseq\"uentemente, o terceiro termo da equa\cao\ (\ref{vdq2}) descreve
uma rota\cao\ de todos os pontos $Q$ em volta de $P$ pelo \^angulo
$\alpha$.

\subsection{Interpreta\cao\ f\ih sica dos elementos do %
tensor de deforma\cao}

O tensor de deforma\cao\ descreve deforma\cao\ pura, este
n\ao\ cont\'em informa\cao\ sobre deslocamento ou
rota\cao\ do corpo deformado como um todo. N\'os esperamos
ent\ao\ que a interpreta\cao\ f\ih sica dos elementos que
est\ao\ na diagonal e daqueles que n\ao\ est\ao\ seja
diferente. Vejamos essas diferen\cas.

\begin{itemize}
\item Significado  dos elementos $e_{11}$, $e_{22}$ e
$e_{33}$

Considerando os pontos $P$ e $Q$ situados ao longo do eixo
$x_1$, temos $dx_i = (dx_1,0,0)$ e $\overline{PQ} = dx_1$.
Da defini\cao\ de deforma\cao\ (pequena), chegamos, no
nosso caso, em
\begin{eqnarray}
\overline{P'Q'}^2 -\overline{PQ}^2 \sim 2e_{ij}dx_idx_j
= 2e_{11}(dx_1)^2 \; ,
\end{eqnarray}
de onde podemos concluir que
\begin{eqnarray}
\overline{P'Q'} \sim \sqrt{1 + 2 e_{11}} dx_1 \; .
\end{eqnarray}

Se n\'os definirmos a extens\ao\ relativa como
$(\overline{P'Q'}-\overline{PQ}) / \overline{PQ}$ e usarmos
a aproxima\cao\ de Taylor de primeira ordem na raiz quadrada, i.e.,
$\sqrt{y}\approx 1 + \frac{1}{2}y$, temos a aproxima\cao
\begin{eqnarray}
e_r \sim \frac{\sqrt{1+2 e_{11}} dx_1-dx_1}{dx_1}
= \sqrt{1+2 e_{11}} - 1 \approx e_{11} \; .
\end{eqnarray}

Ent\ao\ o elemento $e_{11}$ do tensor de deforma\cao\
representa aproximadamente a extens\ao\ (contra\cao)
relativa do material ao longo do eixo $x_1$. As
componentes $e_{22}$ e $e_{33}$ tem interpreta\coes\
similares.

\item Significado  dos elementos $e_{12}$, $e_{13}$ e
$e_{23}$

Vamos considerar dois pontos $Q$ e $R$ nas proximidades
do ponto $P$. Vamos especificar os pontos $Q$ e $R$ no
estado inicial da seguinte maneira:
\begin{eqnarray}
dx_i^{Q}=(dx_1,0,0) \hspace{0.5cm}, \hspace{0.5cm} dx_i^{R}
=(0,dx_2,0) \; .
\end{eqnarray}
Ent\ao, os vetores $dx_i^Q$ e $dx_i^R$ s\ao\ perpendiculares.
Depois da deforma\cao, os pontos $Q$ e $R$ ser\ao\ movidos
para novas posi\coes\ expecificadas pelas aproxima\coes\
seguintes
\begin{eqnarray}
dx_i^{'Q} \sim dx_1 \delta_{1i} + \frac{\partial u_i}
{\partial x_1} dx_1
\hspace{0.5cm}, \hspace{0.5cm}
dx_i^{'R} \sim dx_2 \delta_{i2} + \frac{\partial u_i}
{\partial x_2} dx_2 \; .
\end{eqnarray}

Vamos agora investigar como a deforma\cao\ altera a orienta\cao\
das linhas $\overline{PQ}$ e $\overline{PR}$, que s\ao\
originalmente perpendiculares. Para isto propomos, considerar o
produto escalar dos vetores $dx_i^{'Q}$ e $dx_i^{'R}$ dado por
\begin{eqnarray}
dx_i^{'Q} dx_i^{'R} \sim \left( \frac{\partial u_2}{\partial x_1}
+ \frac{\partial u_1}{\partial x_2}\right) dx_1 dx_2 +
\frac{\partial u_i}{\partial x_1}  \frac{\partial u_i}
{\partial x_2} dx_1 dx_2 \; .
\end{eqnarray}

O \'ultimo termo pode ser desconsiderado desde que consideremos
as deforma\coes\ pequenas e \'e de uma ordem mais elevada em
$|\partial u_i/ \partial x_j|$.

Do produto interno sabemos que
\begin{eqnarray}
dx_i^{'Q} dx_i^{'R} = |dx_i^{'Q}| |dx_i^{'R}| \cos(\gamma)
\sim 2e_{12}dx_1 dx_2 \; ,
\end{eqnarray}
onde $\gamma$ \'e o \^angulo entre os vetores $dx_i^{'Q}$ e
$dx_i^{'R}$. Logo,
\begin{eqnarray} \label{cosg}
\cos(\gamma) \sim \frac{2e_{12}dx_1 dx_2}{|dx_i^{'Q}|
|dx_i^{'R}|} \; .
\end{eqnarray}

De $|dx_i^{'Q}|$ chegamos em
\begin{eqnarray*}
|dx_i^{'Q}| = \sqrt{(dx_1)^2 + 2 \frac{\partial u_1}{\partial x_1}
(dx_1)^2 + \left( \frac{\partial u_1}{\partial x_1} \right)^2
(dx_1)^2} = dx_1 \sqrt{1 + 2 \frac{\partial u_1}{\partial x_1}
+ \left( \frac{\partial u_1}{\partial x_1} \right)^2} \; ,
\end{eqnarray*}
onde, para $|\partial u_i/ \partial x_j| \ll 1$ e usando Taylor na
raiz quadrada, temos
\begin{eqnarray} \label{dxQR}
|dx_i^{'Q}| \approx dx_1 \left(1+\frac{\partial u_1}{\partial x_1}\right)
\mbox{ e similarmente } |dx_i^{'R}|
\approx dx_2 \left(1+\frac{\partial u_2}{\partial x_2}\right) \; .
\end{eqnarray}

Assim, substituindo as express\oes\ de (\ref{dxQR}) em (\ref{cosg})
obtemos
\begin{eqnarray} \label{app}
\cos(\gamma) \sim \frac{2e_{12}dx_1 dx_2}{|dx_i^{'Q}|
|dx_i^{'R}|} &\approx& \frac{2e_{12}dx_1 dx_2}{\left(1+\frac{\partial u_1}
{\partial x_1}\right) \left(1+\frac{\partial u_2}{\partial x_2}
\right) dx_1  dx_2} = \frac{2e_{12}}{1 + \frac{\partial u_1}
{\partial x_1} + \frac{\partial u_2}{\partial x_2}} \nonumber \\
&\approx& 2e_{12} \left(1 - \frac{\partial u_1}{\partial x_1} -
\frac{\partial u_2}{\partial x_2} \right) \approx 2e_{12} \; .
\end{eqnarray}

Se chamamos $\alpha_{12} = \frac{\pi}{2} - \gamma$, temos,
equa\cao\ (\ref{app}),
\begin{eqnarray}
\sin \alpha_{12} \sim \alpha_{12} \approx 2e_{12} \;\;\;\Rightarrow
\;\;\; e_{12} \approx \frac{1}{2} \alpha_{12} \; ,
\end{eqnarray}
pois a aproxima\cao\ $\sin \alpha_{12} \sim \alpha_{12}$ \'e
poss\ih vel j\'a que $\gamma \sim \frac{\pi}{2} \Rightarrow
\alpha_{12} \ll 1$.

Ent\ao, o elemento $e_{12}$ do tensor de deforma\cao\ representa
metade da mudan\ca\ do \^angulo correto entre as dire\coes, que
eram originalmente perperdiculares. Esta mudan\ca\ \'e chamada
de ``cisalhamento''  e o elemento $e_{12}$ de ``deforma\cao\ de
cisalhamento''. Os elementos $e_{13}$ e $e_{23}$ tem uma
interpreta\cao\ similar.

\end{itemize}

\subsection{Deforma\coes\ quadr\'aticas}

A superf\ih cie de um tensor de deforma\cao\ pode ser escrita
na forma
\begin{eqnarray}
e_{ij} x_i x_j = \pm 1 \; ,
\end{eqnarray}
onde depois de uma rota\cao\ apropriada nas coordenadas, esta
pode ser diagonalizada tal que
\begin{eqnarray}
e_{11}' x_1^{'2}+e_{22}' x_2^{'2}+e_{33}' x_3^{'2} = \pm 1 \; .
\end{eqnarray}

Os eixos desta quadr\'atica s\ao\ os eixos principais do tensor
de deforma\cao, os valores $e_{11}', e_{22}',  e_{33}'$ s\ao\ as
{\it deforma\coes\ principais}. As deforma\coes\ de cisalhamento
s\ao\ zero no primeiro sistema de coordenadas.


\subsection{Dilata\cao\ de volume}

Vamos considerar o sistema de coordenadas Cartesianas $x_i$
escolhido tal que este eixo coincide com o eixo principal do
tensor de deforma\cao\ e vamos considerar uma mudan\ca\ de
um volume elementar $dV$ durante a deforma\cao. No estado
inicial n\'os temos
\begin{eqnarray}
dV = dx_1 dx_2 dx_3 \; ,
\end{eqnarray}
que depois da deforma\cao, $dV$ mudar\'a para $dV'$
\begin{eqnarray}
dV' = dx_1' dx_2' dx_3' \; .
\end{eqnarray}

Para $dx_i'$ n\'os derivamos sobre
\begin{eqnarray}
dx_i' \sim dx_i + \frac{\partial u_i}{\partial x_k} dx_k \; .
\end{eqnarray}

Em nosso sistema de coordenadas isto dar\'a para $dx_i'$
(no sistema principal, as deforma\coes\ de cisalhamento
s\ao\ zero),
\begin{eqnarray}
dx_1' \sim dx_1 + \frac{\partial u_1}{\partial x_1} dx_1
= dx_1(1+e_{11}) \;,
\end{eqnarray}
e similarmente para $dx_2'$ e $dx_3'$. Para $dV'$ temos
\begin{eqnarray}
dV' \sim (1+e_{11})(1+e_{22})(1+e_{33})dx_1 dx_2 dx_3
\sim dV + (e_{11} + e_{22} + e_{33}) dV \; ,
\end{eqnarray}
onde foi ignorado os temos de maior ordem (como
$e_{11}e_{22}, ...$). Logo,
\begin{eqnarray}
\theta = \mbox{div}\; \vec{u} = (e_{11} + e_{22} + e_{33})
\sim \frac{dV'-dV}{dV} \; .
\end{eqnarray}

A quantidade $\theta$ \'e chamada de {\it dilata\cao\ de
volume} ou abreviadamente {\it dilata\cao}. Ela representa
aproximadamente a mudan\ca\ relativa do volume durante a
deforma\cao. Ela n\ao\ depende das deforma\coes\ de cisalhamento
$e_{12}$, $e_{13}$ e $e_{23}$ porque um cisalhamento n\ao\ altera o
volume de um corpo. Como $\theta$ \'e o tra\c co do tensor de
deforma\cao\ que \'e invariante sob transforma\cao\ de
coordenadas, temos que $\theta$ descreve a dilata\cao\ 
de volume em qualquer sistema de coordenadas cartesianas.


%\begin{thebibliography}{99}
%\bibitem{Psencik} P\v{s}en\v{c}ik, I., 1994, {\em Introduction
%to seismic methods} - Lecture Notes, PPPG / UFBa.
%\end{thebibliography}

