
\section{Absor\cao\ e dispers\ao}

A propaga\cao\ de uma onda poderia continuar para sempre se n\ao\
houvesse perda de energia el\'astica. Na realidade a energia el\'astica
se transforma em outros tipos de energia e a amplitude das ondas que se
propagam em meios reais decresce. Aqui explicamos a perda de energia em
termos de fric\cao\ interna. Usamos o fator de qualidade $Q$ para
descrever a fric\cao\ interna. Para tens\oes\ que variam harmonicamente
e s\ao\ aplicadas num volume de um meio, definimos $Q$ como
\begin{eqnarray}
\frac{1}{Q(\omega)} = Q^{-1}(\omega) = -\Delta E/2\pi\overline{E}
\end{eqnarray}
onde $\overline{E}$ \'e a quantidade de energia el\'astica m\'edia no
tempo de um volume de um meio, $\Delta E$ \'e a perda de energia num
volume durante um ciclo e $\omega$ \'e a frequ\^encia circular. Em um
meio perfeitamente el\'astico n\ao\ h\'a perda de energia e $\Delta E =
0$ e $Q(\omega) \rightarrow \infty$. Observamos tamb\'em que $Q(\omega)$
decresce com o aumento de $\Delta E$.

Como j\'a foi visto, a energia el\'astica \'e proporcioal ao quadrado da
amplitude de uma onda plana harm\^onica. Consideremos uma onda plana
ac\'ustica propagando com velocidade $c$ na dire\cao\ $N_i$
\begin{eqnarray}
p(x_m,t) = P\exp\left[ -i\omega \left( t - \frac{N_m x_m}{c} \right) \right],
\end{eqnarray}
entrando na regi\ao\ de um meio ac\'ustico com absor\cao. Reescrevemos a
equa\cao\ para $Q^{-1}(\omega)$ em termos da amplitude $P$,
\begin{eqnarray}
Q^{-1}(\omega) = -\Delta P/\pi P.
\end{eqnarray}

Consideramos que a atenua\cao\ e a mudan\ca\ de amplitude s\ao\
isotr\'opicos. Ent\ao\ podemos escrever
\begin{eqnarray}
\Delta P = \frac{dP}{ds}\lambda = (2\pi c/\omega)\frac{dP}{ds},
\end{eqnarray}
onde $ds$ \'e o elemento diferencial na dire\cao\ $N_i$. Inserindo
$\Delta P$ na equa\cao\ para $Q^{-1}(\omega)$ temos
\begin{eqnarray}
P^{-1}\frac{dP}{ds} = -\frac{\omega}{2cQ(\omega)}\Rightarrow
P = P_0\exp\left[-\frac{\omega s}{2cQ(\omega)}\right] = P_0\exp\left[
-\frac{\omega N_i x_i}{2cQ(\omega)}\right],
\end{eqnarray}
onde $s$ \'e a dist\^ancia at\'e a origem e $P_0$ \'e a amplitude da
onda na origem. Inserindo a express\ao\ para $P$ na equa\cao\ da onda
plana ac\'ustica temos
\begin{eqnarray}
p(x_m,t) = P_0\exp\left[-i\omega\left(t -\frac{N_m x_m}{c_a}\right)\right],
\end{eqnarray}
onde
\begin{eqnarray}
\frac{1}{c_a} = \frac{1}{c} + \frac{i}{2cQ(\omega)},
\end{eqnarray}
ou ainda, podemos reescrever a equa\cao\ acima na forma
\begin{eqnarray}
p(x_m,t) = P_0\exp[-i\omega(t -p_m^{(a)}x_m)] = P_0\exp(-\omega p_m^I x_m)
\exp[-i\omega(t - p_m^R x_m))],
\end{eqnarray}
onde
\begin{eqnarray}
p_i^{(a)} = \frac{N_i}{c_a} = p_i^R + ip_i^I.
\end{eqnarray}

A express\ao\ acima \'e semelhante \`a express\ao\ da onda plana
inomog\^enea em um meio perfeitamente el\'astico. A diferen\ca\ \'e que
aqui os vetores $p_i^R$ e $p_i^I$ s\ao\ paralelos, onde $p_i^R = N_i/c,
p_i^I = N_i/2cQ$. Vimos at\'e agora a f\'ormula que descreve a
propaga\cao\ de uma onda plana homog\^enea em um meio com absor\cao\ na
dire\cao\ $N_i$ e com velocidade de fase $c$. A amplitude desta onda
decresce na mesma dire\cao, i.e., os planos de fase constante e
amplitude constante s\ao\ paralelos. Os efeitos de atenua\cao\ podem ser
descritos atrav\'es de outra quantidade chamada coeficiente de
absor\cao\ $\alpha_{ABS}$ definido como
\begin{eqnarray}
\alpha_{ABS}(\omega) = \omega/2cQ(\omega).
\end{eqnarray}
Para $Q \rightarrow \infty$ (meio perfeitamente el\'astico),
$\alpha_{ABS} \rightarrow 0$. Usando $\alpha_{ABS}$ podemos reescrever
$p(x_m,t)$ como
\begin{eqnarray}
p(x_m,t) = P_0\exp(-\alpha_{ABS}(\omega)N_m x_m)\exp\left[
-i\omega\left(t - \frac{N_m x_m}{c}\right)\right].
\end{eqnarray}

Consideremos agora os efeitos de atenua\cao\ numa onda impulso cuja
forma de onda \'e a $\delta$-fun\cao,
\begin{eqnarray}
p(x_m,t) = P\delta(t - p_m x_m).
\end{eqnarray}
O espectro da fun\cao\ $\delta(t - p_m x_m)$ \'e
\begin{eqnarray}
\delta(\omega) = \intii\delta(t - p_m x_m)e^{i\omega t}dt = e^{i\omega p_m x_m}
\end{eqnarray}
ent\ao\ o onda plana impulso pode ser expressa por
\begin{eqnarray}
p(x_m,t) = (2\pi)^{-1}P\intii e^{-i\omega(t-p_m x_m)}d\omega.
\end{eqnarray}
Vemos que a onda $\delta-$impulso pode ser obtida pela s\'intese de
ondas harm\^onicas de amplitudes iguais. Estendamos esta s\'intese para
meios n\ao\ el\'asticos, para a qual achamos que a onda plana
harm\^onica com frequ\^encia $\omega = 2\pi f$ tem a forma
\begin{eqnarray}
p(x_m,t) = P_0\exp\left[-\frac{1}{2Q(\omega)}|\omega p_m x_m| \right]\exp[-i\omega(t-p_m x_m)]
\end{eqnarray}
e vejamos como a rela\cao\ das amplitudes \'e afetada pela atenua\cao.
Por simplicidade consideremos $Q$ independente da frequ\^encia. Ent\ao,
temos
\begin{eqnarray}
p(x_m,t) = \frac{P_0}{2\pi}\intii e^{-i\omega(t-p_m x_m)}e^{-\frac{|\omega p_m x_m|}{2Q}}d\omega.
\end{eqnarray}
Consideremos $|\omega p_m x_m|$ para evitar crescimento infinito das
amplitudes em termos que possuem frequ\^encia negativa e $p_m x_m > 0$
(ou $\omega > 0$ e $p_m x_m < 0$). Assim, ficamos com
\begin{eqnarray}
p(x_m,t) = \frac{P_0}{2\pi}\iio e^{-i\omega(t-p_m x_m)}e^{\frac{-|\omega|}{2Q}|p_m x_m|}d\omega 
+ \frac{P_0}{2\pi}\ioi e^{-i\omega(t-p_m x_m)}e^{\frac{-\omega}{2Q}|p_m x_m|}d\omega,
\end{eqnarray}
que leva \`a
\begin{eqnarray}
p(x_m,t) &=& \frac{P_0}{2\pi}\ioi e^{i\omega(t-p_m x_m)}e^{\frac{-\omega}{2Q}|p_m x_m|}d\omega +
\frac{P_0}{2\pi}\ioi e^{-i\omega(t-p_m x_m)}e^{\frac{-\omega}{2Q}|p_m x_m|}d\omega \nonumber\\  
&=& -\frac{P_0}{i2\pi(t-p_m x_m) - \pi|p_m x_m|Q^{-1}} +
     \frac{P_0}{i2\pi(t-p_m x_m) + \pi|p_m x_m|Q^{-1}} \nonumber\\
&=& \frac{2Q|p_m x_m|P_0}{4\pi(t-p_m x_m)^2 Q^2 + \pi(p_m x_m)^2}.
\end{eqnarray}

O resultado que obtivemos viola observa\coes\ em v\'arios aspectos. A
viola\cao\ mais grave \'e a que viola o princ\ih pio da causalidade. O
sinal \'e n\ao\ nulo inclusive para $t<0$. Nosso sinal \'e sim\'etrico
enquanto as observa\coes\ indicam curto tempo de ascens\ao\ do sinal
seguido por um vagaroso decaimento. Existem muitas explica\coes\ poss\ih
veis para o desajuste de nossa teoria simplificada e observa\coes. O
motivo b\'asico do nosso problema \'e a neglig\^encia da dispers\ao\ de
ondas, i.e., a velocidade $c$ precisa ser considerada como fun\cao\ da
frequ\^encia, $c=c(\omega)$. Desta maneira removemos a simetria do
sinal. Para garantir causalidade do sinal, i.e., $p(x_m,t)=0$ para
$t<0$, a velocidade $c(\omega)$ e $Q(\omega)$ n\ao\ podem ser
independentes, e s\ao\ conectados pelas ent\ao\ chamadas rela\coes\ de
Kramers-Kronig. Estas rela\coes\ s\ao\ consequ\^encia do fato que as
partes real e imagin\'aria do espectro de uma fun\cao\ causal formam um
par de Hilbert.

Um modelo well-fitting de absor\cao\ nos d\'a a seguinte rela\cao\ entre
as velocidades de fase e duas frequ\^encias diferentes
\begin{eqnarray}
\frac{c(\omega_1)}{c(\omega_2)} = 1 + \frac{1}{\pi Q}\ln\left(\frac{\omega_1}{\omega_2}\right).
\end{eqnarray}

Aqui $Q$ depende de $\omega$ muito fracamente, ent\ao\ \'e praticamente
constante.

A lei de Hooke para um s\'olido anel\'astico anisotr\'opico pode ser
escrita como
\begin{eqnarray}
\tau_{ij} = a_{ijkl}^{(a)}e_{kl}
\end{eqnarray}
onde
\begin{eqnarray}
a_{ijkl}^{(a)} = a_{ijkl} + ia_{ijkl}^I
\end{eqnarray}    
Aqui $a_{ijkl}^I$ controla o efeito de absor\cao\ que neste caso \'e
dependente da dire\cao. Ambas componentes de $a_{ijkl}^{(a)}$ (real e
imagin\'aria) s\ao\ dependentes da frequ\^encia.

Como os par\^ametros el\'asticos dependem da frequ\^encia, a velocidade
de fase tamb\'em depende da frequ\^encia e temos novamente ondas
dispersivas.

Analogamente ao modo com que a anelasticidade afeta a propaga\cao\ da
onda num meio ilimitado, tamb\'em afeta a reflex\ao\ e a transmiss\ao\
na interface separando meios anel\'asticos. Vejamos agora os efeitos de
dispers\ao. Consideremos uma onda plana ac\'ustica
\begin{eqnarray}
p(x_m,t) = PF\left(t - \frac{N_m x_m}{c}\right)
\end{eqnarray}
propagando num meio ac\'ustico sem absor\cao. Agora, consideremos a
velocidade de fase dependendo da frequ\^encia $c = c(\omega)$. O sinal
anal\'itico tem ent\ao\ a forma
\begin{eqnarray}
F\left(t - \frac{N_m x_m}{c}\right) = 
\frac{1}{\pi}\ioi g(\omega)e^{-i\omega\left(t - \frac{N_m x_m}{c(\omega)} \right)}d\omega,
\end{eqnarray}
onde $g(\omega)$ \'e o espectro do sinal real considerado. Ent\ao, a
express\ao\ da onda plana ac\'ustica fica
\begin{eqnarray}
p(x_m,t) = \frac{P}{\pi}\ioi g(\omega)e^{-i\omega\left(t - \frac{N_m x_m}{c(\omega)} \right)}d\omega.
\end{eqnarray}
Se $c$ for independente da frequ\^encia, a f\'ormula acima se reduz \`a
f\'ormula da onda plana transiente harm\^onica
\begin{eqnarray}
p(x_m,t) = PF\left(t - \frac{N_m x_m}{c}\right).
\end{eqnarray}

Considerando $c = c(\omega)$, assumindo que $g(\omega)$ \'e efetivamente
n\ao\ nulo somente para uma faixa de frequ\^encias $\omega_0 -
\Delta\omega < \omega < \omega_0 + \Delta\omega$, podemos escrever
\begin{eqnarray}
p(x_m,t) = \frac{P}{\pi}\int_{\omega_0 - \Delta\omega}^{\omega_0 + \Delta\omega}
g(\omega)e^{-i(\omega t - k(\omega)N_m x_m)}d\omega,
\end{eqnarray}
onde $k(\omega) = \omega/c(\omega)$ \'e o n\'umero de onda. Na
vizinhan\ca\ de $\omega_0$ podemos expandir o n\'umero de onda como
\begin{eqnarray}
k(\omega) = k(\omega_0) + \left.\frac{dk}{d\omega}\right|_{\omega_0}(\omega - \omega_0) + ...
\end{eqnarray}
Definindo $U = \frac{d\omega}{dk}$ e usando $\omega = kc(k)$ temos
\begin{eqnarray}
U(k) = c(k) + k\frac{dc(k)}{dk}.
\end{eqnarray} 
Vemos que a quantidade $U$ tem a mesma dimens\ao\ da velocidade $c$.
Mostramos a seguir que $U$ \'e a velocidade de grupo, i.e., a velocidade
cim que a energia el\'astica se propaga. Usamos $U$ da seguinte forma
\begin{eqnarray}
U(\omega) = \frac{c(\omega)}{1 - \frac{\omega}{c(\omega)}\frac{dc(\omega)}{d\omega}}.
\end{eqnarray}
Se $c$ n\ao\ depende da frequ\^encia $\omega$, ent\ao\ $U = c$. Em meios
ac\'usticos, a velocidade de grupo $U$ e a velocidade de fase $c$
diferem somente se ambos dependem da frequ\^encia. Se assumirmos que o
espectro n\ao\ varia muito no intervalo $(\omega_0 -
\Delta\omega,\omega_0 + \Delta\omega)$, podemos aproximar $g(\omega)$
como $g(\omega) \sim g(\omega_0)$ e a express\ao\ para $p(x_m,t)$ fica
\begin{eqnarray}
p(x_m,t) \approx \frac{P}{\pi}e^{-i(\omega_0 t - k(\omega_0)N_m x_m)}g(\omega_0)
\int_{\omega_0 - \Delta\omega}^{\omega_0 + \Delta\omega}
e^{-i(\omega - \omega_0)(t - \frac{N_m x_m}{U(\omega_0)})}d\omega.
\end{eqnarray}
Calculamos agora a integral
\begin{eqnarray}
\int_{\omega_0 - \Delta\omega}^{\omega_0 + \Delta\omega}e^{-i(\omega - \omega_0)(t - \frac{N_m x_m}{U(\omega_0)})}d\omega &=&
\left[- \frac{e^{-i(\omega - \omega_0)(t - \frac{N_m x_m}{U(\omega_0)})}}{i(t - \frac{N_m x_m}{U(\omega_0)})} 
\right]_{\omega_0 - \Delta\omega}^{\omega_0 + \Delta\omega} \nonumber\\
&=& \frac{e^{-i\Delta\omega(t - \frac{N_m x_m}{U(\omega_0)})} + e^{i\Delta\omega(t - 
\frac{N_m x_m}{U(\omega_0)})}}{i(t - \frac{N_m x_m}{U(\omega_0)})} \nonumber\\
&=& 2\Delta\omega \mbox{sinc}\left[\Delta\omega\left(t - \frac{N_m x_m}{U(\omega_0)} \right) \right],
\end{eqnarray}
onde a fun\cao\ $sinc(t)$ \'e definida como
\begin{eqnarray}
\mbox{sinc}(t) = \frac{\sin t}{t}.
\end{eqnarray}
Esta fun\cao\ possui um m\'aximo igual a 1 em $t=0$ e se aproxima de
zero quando $t \longrightarrow \pm \infty$.

Escrevemos ent\ao\ a express\ao\ final para $p(x_m,t)$
\begin{eqnarray}
p(x_m,t) \approx C\,sinc\,\left[\Delta\omega\left(t - \frac{N_k x_k}{U(\omega_0)} \right) \right]
e^{-i\omega_0(t - \frac{N_m x_m}{c(\omega_0)}) + i\varphi_0},
\end{eqnarray}
onde
\begin{eqnarray}
C &=& \frac{2}{\pi}P|g(\omega_0)|\Delta\omega, \;\; g(\omega_0) = |g(\omega_0)|e^{i\varphi_0}, \;\; k(\omega_0) = 
\frac{\omega_0}{c(\omega_0)}, \\
U(\omega_0) &=& \frac{c(\omega_0)}{1 - \frac{\omega_0}{c(\omega_0)}\frac{dc(\omega_0)}{d\omega_0}}.
\end{eqnarray}
A f\'ormula acima descreve uma onda plana harm\^onica propagando na
dire\cao\ de $N_i$ com velocidade de fase $c(\omega_0) = \omega_0/k_0$ e
frequ\^encia $\omega_0$. A amplitude da onda plana \'e modulada pela
fun\cao\ $sinc$. A fun\cao\ forma um envelope da onda plana e este
envelope tamb\'em se propaga na dire\cao\ $N_i$ mas com velocidade
$U(\omega_0)$. Como a energia el\'astica \'e proporcional ao quadrado da
amplitude da onda plana, em nosso caso da fun\cao\ $sinc$, significa que
a energia se move com velocidade $U$ e ent\ao\ $U$ \'e a velocidade de
grupo. Poder\'iamos chegar ao mesmo resultado mesmo sem fazer
simplifica\coes, mas seria mais complicado. Dispers\ao\ n\ao\ est\'a
relacionada apenas com meios anel\'asticos, mas tamb\'em com v\'arias
ondas de interfer\^encia como, por exemplo, ondas de superf\'icie. Neste
caso falamos de dispers\ao\ geom\'etrica. No caso de dispers\ao\ devido
a comportamento anel\'astico do meio, falamos de dispers\ao\ do
material. Se $U \leq c$ temos dispers\ao\ normal. Se $U \geq c$ temos
dispers\ao\ de anomalia.

