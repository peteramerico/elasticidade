
\chapter{Introdu\cao}

\section{O que \'e uma onda?}

Uma onda \'e a perturba\cao\ de um estado n\ao\ perturbado, que pode se
deslocar no espa\co\ em fun\cao\ do tempo $t$, transportando assim energia
e a informa\cao\ sobre esta perturba\cao. Falamos de ``propaga\cao\ da
onda''. Matematicamente, denotamos uma
onda (em uma dimens\ao\ espacial $x$) por $\Phi = \Phi(x,t)$. Em um instante
fixo, $t_0$, a onda assume a forma $\Phi(x,t_0) = f(x)$ no espa\co.
Se a onda se propagar sem mudar esta forma, podemos descrev\^e-la por
uma transla\cao\ em fun\cao\ do tempo $t$, i.e., 
\begin{equation}
\Phi(x,t) = f(x-ct) \; .
\label{onda}
\end{equation}
Aqui, $c$ denota a velocidade de propaga\cao, que podemos verificar
facilmente. Para a onda descrita pela equa\cao\ (\ref{onda}) se deslocar
em uma unidade do tempo $t_2-t_1 = 1$~s por uma unidade da dist\^ancia
$x_2-x_1 = 1$~m, i.e., para propagar com a velocidade de $1$~m/s,
precisamos que os seus argumentos em $(x_1,t_1)$ e $(x_2,t_2)$ sejam
iguais, ou seja, $y_1=x_1-ct_1=y_2=x_2-ct_2$. Da\'{\i}, obtemos $x_2-x_1
- c(t_2-t_1) = 0$ que, com os valores acima, fornece $c=1$~m/s.
Geralmente, denotaremos velocidades escalares por n\'umeros positivos.
Desta forma, para permitirmos que a onda (\ref{onda}) se propague
tamb\'em na dire\cao\ negativa do eixo $x$, denotaremos a forma geral de
uma onda sem mudan\ca\ da sua forma (costuma-se usar a palavra inglesa
``wavelet'' para descrever a forma da onda) como
\begin{equation}
\Phi(x,t) = f(x-ct) + g(x+ct) \; .
\label{onda2}
\end{equation}

No caso geral, n\ao\ teremos s\'o um deslocamento da onda com uma
velocidade e forma constante. Tanto a velocidade $c$ quanto a forma da
onda podem mudar ao longo da propaga\cao\ da onda. O fen\^omeno da
mudan\ca\ da wavelet acontece, por exemplo, quando a onda sob
considera\cao\ de fato se comp\~oe de ondas parciais que s\ao\
influenciadas de maneira diferente pelo meio de propaga\cao. Se a
influ\^encia difererente se deve ao conte\'udo de freq\"u\^encia
diferente das ondas parciais, este fen\^omeno recebe o nome de
``dispers\ao''.

\section{Formas de propaga\cao}

Ondas se propagam em meios reais ao passar a informa\cao\ (energia) do 
deslocamento de uma part\icula\ (mol\'ecula, \'atomo) do meio de
propaga\cao\ para a pr\'oxima. Em meios denominados ``ac\'usticos''
(gases, liquidos), as part\iculas\ n\ao\ possuem liga\coes\ qu\imicas.
Portanto, o deslocamento de uma part\icula\ s\'o pode ser transferido para
outra se esta se encontra no caminho do deslocamento. Desta forma, a
propaga\cao\ da onda s\´o pode acontecer na dire\cao\ do deslocamento.
Ondas para as quais a dire\cao\ do deslocamento coincide com a dire\cao\
da propaga\cao\ s\ao\ chamadas de ``longitudinais''. Concluimos que em
meios ac\'usticos, somente se propagam ondas longitudinais.

J\'a em meios s\'olidos, as part\iculas\ possuem liga\coes. Assim, o
deslocamento de uma delas tamb\'em afeta outras part\iculas\ nas
dire\coes\ perpendiculares ao deslocamento. Desta forma, uma onda pode ser gerada com a
dire\cao\ de deslocamento perpendicular \`a dire\cao\ de propaga\cao.
Ondas com esta propriedade s\ao\ chamadas de ``transversais''.
Conclu\imos\ que em meios s\'olidos, podem se propagar tanto ondas
longitudinais como transversais.

A condi\cao\ necess\'aria para que ondas de fato se propaguem \'e a
exist\^encia de for\cas\ de mola que fa\cam\ as part\iculas\ deslocadas
voltarem aos seus lugares originais quando a for\ca\ que causou o
deslocamento passou. Meios s\'olidos com esta propriedade s\ao\ chamados
de ``el\'asticos'', em contraste com meios ``pl\'asticos'' que sofrem
somente uma deforma\cao\ quando uma particula \'e deslocada, sem que ela
volte ao seu lugar depois da for\ca\ passar.

Uma outra denomina\cao\ para as ondas longitudinais \'e ``ondas
compressionais'', uma vez que elas propagam uma perturba\cao\ em forma
de uma compress\ao do meio. Na s\ismica\ e sismologia, tamb\'em se usa o
nome ``ondas P'', abrevia\cao\ de ``ondas prim\'arias'', uma vez que,
num registro s\ismico\ em uma esta\cao\ distante, s\ao\ elas que d\ao\ o
primeiro sinal de um terremoto. Correspondentemente, existem os nomes
alternativos ``ondas cisalhantes'' e ``ondas S'', abrevia\cao\ de
``ondas secund\'arias'', para ondas transversais, em fun\cao\ da forma
da perturba\cao\ propagada (cisalhamento) e da ordem de chegada numa
esta\cao\ s\ismica\ (sism\'ografo) ap\'os um terromoto.

Deve-se notar que em um meio homog\^eneo, ondas longitudinais e
transversais propagam de forma desacoplada, i.e., independentemente uma
da outra. Esta situa\cao\ muda em um meio n\ao\ homog\^eneo (ou {\it
heterog\^eneo}), onde a velocidade de propaga\cao\ depende do lugar.
Mesmo em meios heterog\^eneos, geralmente propagam estes dois tipos de
ondas (P e S), mas elas n\ao\ s\ao\ mais ondas compressionais e cisalhantes
puras. Em cada heterogeneidade do meio, estes tipos de ondas s\ao\
acopladas, i.e., uma onda P propagando cria uma onda S e vice versa.

\section{Equa\cao\ da onda}

Fen\^omenos da propaga\cao\ de onda s\ao\ descritos por uma equa\cao\
diferencial, chamada de equa\cao\ da onda. Existem v\'arias formas da
equa\cao\ da onda, dependendo do grau de complexidade do meio no qual as
ondas a serem descritas se propagam. A forma da equa\cao\ da onda mais
simples \'e a da equa\cao\ da onda ac\'ustica, sem fonte, em um meio com densidade
constante. Em um meio tridimensional, ela tem a forma
\begin{equation}
\Delta \Phi = 
\frac{\partial^2 \Phi}{\partial x^2} +
\frac{\partial^2 \Phi}{\partial y^2} +
\frac{\partial^2 \Phi}{\partial z^2}
= \frac{1}{c^2}
\frac{\partial^2 \Phi}{\partial t^2} \; ,
\label{eoadc}
\end{equation}
onde $x,y,z$ denotam as coordenadas espaciais, $t$ denota o tempo e $c$
\'e a velocidade da propaga\cao. A velocidade \'e um par\^ametro do
material, i.e., depende do meio no qual a onda se propaga.

Em um meio ac\'ustico com densidade vari\'avel, a equa\cao\ da onda
assume a forma
\begin{equation}
\nabla\cdot \left(\frac{1}{\varrho} \nabla \Phi \right)
= \frac{1}{k}
\frac{\partial^2 \Phi}{\partial t^2} \; ,
\label{eoadv}
\end{equation}
onde $\varrho$ representa a densidade do meio de propaga\cao\ e $k$ \'e
conhecido como m\'odulo bulk ou incompressibilidade. A velocidade de
propaga\cao\ \'e relacionada a estes dois par\^ametros por
$c^2=k/\varrho$.

Em um meio el\'astico, a situa\cao\ \'e mais complicada. Como j\'a
mencionado acima, existem outros tipos de ondas. Este fato \'e refletido
na equa\cao\ da onda el\'astica. Em primeiro lugar, devemos observar que
ondas el\'asticas n\ao\ podem ser descritas por um escalar $\Phi$, mas
necessitam de uma descri\cao\ vetorial, $\vec\Phi$. Em um meio
homog\^eneo e isotr\'opico, i.e., com as mesmas propriedades em todas as
dire\coes, a equa\cao\ da onda el\'astica \'e dada por
\begin{equation}
(\lambda+\mu) \nabla ( \nabla \cdot \vec\Phi) + \mu (\nabla \cdot \nabla)
\vec\Phi = \varrho
\frac{\partial^2 \vec\Phi}{\partial t^2} \; ,
\label{eoehi}
\end{equation}
onde $\lambda$ e $\mu$ s\ao\ os chamados par\^ametros de Lam\'e. O
segundo par\^ametro de Lam\'e, $\mu$, tamb\'em \'e conhecido como
m\'odulo de cisalhamento. A combina\cao\ $\lambda+2\mu/3 = k$ relaciona
os par\^ametros de Lam\'e com o m\'odulo bulk. Em um meio ac\'ustico, o
m\'odulo do cisalhamento \'e zero. Assim, a equa\cao\ el\'astica se reduz
\`a equa\cao\ ac\'ustica, onde a quantitade escalar \'e a press\ao, que
se relaciona com o deslocamento da part\icula, quantidade vetorial na
equa\cao\ (\ref{eoehi}), por $\Phi=-\lambda\nabla\cdot\vec\Phi$.

Em um meio el\'astico, n\ao\ homog\^eneo mas ainda isotr\'opico,
acrescentam-se termos das derivadas dos par\^ametros de Lam\'e ao
lado direito da equa\cao\ (\ref{eoehi}), fornecendo
\begin{equation}
(\lambda+\mu) \nabla ( \nabla \cdot \vec\Phi) + \mu (\nabla \cdot \nabla)
\vec\Phi
+(\nabla \cdot \vec\Phi) \nabla \lambda 
+\nabla\mu \times (\nabla \times \vec\Phi) 
+2(\nabla\mu \cdot \nabla) \vec\Phi 
= \varrho
\frac{\partial^2 \vec\Phi}{\partial t^2} \; .
\label{eoeii}
\end{equation}

Finalmente, a equa\cao\ da onda el\'astica mais geral \'e aquela para um
meio inhomog\^eneo e anisotr\'opico. Ela cont\'em, ao inv\'es dos dois
par\^ametros de Lam\'e, 21 par\^ametros. Estas s\ao\ representadas como
componentes independentes do tensor el\'astico $c_{ijkl}$. Com a ajuda
deste tensor, a equa\cao\ da onda el\'astica para um meio heterog\^eneo
e anisotr\'opico pode ser escrita usando componentes como
\begin{equation}
\sum\limits_{j=1}^3
\sum\limits_{k=1}^3
\sum\limits_{l=1}^3
\frac{\partial}{\partial x_l}
\left(c_{ijkl} \frac{\partial \Phi_j}{\partial x_k}\right)
= \varrho
\frac{\partial^2 \Phi_i}{\partial t^2} \; ,
\label{eoeia}
\end{equation}
onde $\Phi_i$ e $\Phi_j$ ($i,j=1,2,3$) s\ao\ as componentes nas
dire\coes\ de $x_i$ e $x_j$ do vetor $\vec\Phi$.

