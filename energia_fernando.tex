
\section{Considera\coes\ sobre energia}

Em se\coes\ anteriores foram introduzidas algumas
densidades de energia como energia de deforma\cao\
$W$, energia cin\'etica $K$, energia el\'astica
$E=W+K$ e vetor de fluxo de energia $S_i$. Nesta
se\cao\ estudamos tais densidades de energia para
propaga\coes\ de ondas planas em meios ac\'ustico
homog\^eneo, anisotr\'opico e isotr\'opico. As
quantidades acima s\ao\ definidas como quantidades
instant\^aneas, isto \'e, em fun\cao\ do tempo,
apesar de, na pr\'atica trabalharmos com estas
independente do tempo. Conseguimos esta
independ\^endia no tempo pela aproxima\cao\ da
{\it m\'edia temporal} que \'e conveniente para
aplica\coes\ em ondas harm\^onicas no tempo ou
pela aproxima\cao\ da {\it integra\cao\ no tempo}
para ondas transientes, ou seja, sinais s\ismicos.

A m\'edia temporal para ondas harm\^onicas \'e feita
sobre um per\iodo. As quantidades de m\'edia temporal 
$\overline{W}$, $\overline{K}$, $\overline{E}$ e
$\overline{S}_i$ s\ao\ determinadas pelos
seus equivalentes instant\^aneos $W$, $K$, $E$ e
${S_i}$ pela seguinte regra aplicada aqui, por
exemplo, para a energia de deforma\cao\ $W$
\begin{eqnarray}
\overline{W}(x_m) = \frac{1}{T} \int_{t}^{t+T}
W(x_m,t) dt \; ,
\end{eqnarray}
e as quantidades integradas $\hat{W}$, $\hat{K}$,
$\hat{E}$ e $\hat{S}_i$ pela regra
\begin{eqnarray}
\hat{W}(x_m) = \intii W(x_m,t) dt \; .
\end{eqnarray}

Desde que um sinal s\ismico\ dure somente dentro de
um intervalo de tempo limitado, somente este
intervalo contribui para a integral, fora deste
intervalo a contribui\cao\ da integral \'e zero.

Desde que as quantidades de energia s\ao\  n\ao-%
lineares na press\ao, velocidade da part\icula\ ou
deslocamento, devemos trabalhar com partes
fisicamente significativas da solu\cao\ complexa
da equa\cao\ do movimento. Na sequ\^encia, consideramos,
ent\ao, somente a parte real da solu\cao\ complexa
\begin{eqnarray*}
w(x_m,t) = W_0 F(t-T(x_m)) \; .
\end{eqnarray*} 
Por simplicidade, denotamos a parte real por
$w(x_m,t)$, definindo
\begin{eqnarray*}
w(x_m,t) = \frac{1}{2}(W_0F +W_0^{*}F^{*}) \; ,
\end{eqnarray*}
onde o s\imbolo\ $^{*}$ denota o conjugado complexo.


\subsection{Meio ac\'ustico}
\label{enacus}

Partes reais de solu\coes\ complexas para press\ao\ e
velocidade da part\icula\ podem ser escritas, quando
consideramos $p_i$ real, como
\begin{eqnarray}
p = \frac{1}{2}(PF + P^{*}F^{*}) \; , \hspace{1cm}
v_i = \frac{1}{2}\frac{p_i}{\rho}(PF + P^{*}F^{*}) \; ,
\end{eqnarray}
onde omitimos o argumento do sinal anal\itico\
$F(t-(N_i x_i)/c)$. Vamos expressar $W$, $K$, $E$ e
$S_i$ em termos da press\ao\ e da velocidade da
part\icula\ dadas acima. Tendo em mente que no
caso ac\'ustico 
\begin{eqnarray}
p = -k\theta, \hspace{1cm}
\tau_{ij} = -p\delta_{ij} \; ,
\end{eqnarray}
podemos escrever
\begin{eqnarray}
W &=& \frac{1}{2}\tau_{ij}u_{i,j}=-\frac{1}{2}pu_{i,i}
= \frac{1}{2k}p^2 = \frac{1}{2}\kappa p^2 =
\frac{1}{8}\kappa(PF+P^{*}F^{*})^2\;,\label{enerdef}\\
K &=& \frac{1}{2}\rho\dot{u_i}\dot{u_i} =
\frac{1}{2}\rho v_i v_i = \frac{1}{8}(\rho c^2)^{-1}
(PF + P^{*}F^{*})^2 \; , \label{enercin}\\
S_i &=& -\tau_{ij}\dot{u_j} = p \delta_{ij} v_j =
p v_i =\frac{1}{4}p_i\rho^{-1}(PF+P^{*}F^{*})^2 \; ,
\label{fluxo}
\end{eqnarray}
onde usamos $v_i = \dot{u_i} = \fracpp{u_i}{t}$.
Desde que $c^2 = (\kappa\rho)^{-1}$, as equa\coes\
acima levam a um importante resultado, $W = K$.
Ent\ao, a energia de deforma\cao\ de uma onda
ac\'ustica plana \'e igual para qualquer tempo
a sua energia cin\'etica. Logo, para a energia
el\'astica temos
\begin{eqnarray} \label{enerelas}
E = \frac{1}{4}\kappa(PF + P^{*}F^{*})^2 \; .
\end{eqnarray}
Podemos tamb\'em chegar na express\ao\ para a
velocidade do fluxo de energia, conhecida tamb\'em
como {\it velocidade de grupo} $v_i^{(g)}$,
\begin{eqnarray}
v_i^{(g)} = S_i/E = c^2 p_i = c N_i = c_i \; .
\end{eqnarray} 
Portanto, em um meio a\'custico homog\^eneo, a
velocidade do fluxo de energia tem a mesma
dire\cao\ e magnitude que a velocidade de fase.

Agora vamos estudar as quantidades integradas no
tempo trabalhando com as seguintes integrais
\begin{eqnarray} \label{integs}
\intii F^2(t) dt \; , \hspace{1cm}
\intii (F^{*}(t))^2 dt \; , \hspace{1cm}
\intii F(t)F^{*}(t) dt \; .
\end{eqnarray}

As integrais de \refi{integs}\ podem ser
simplesmente avaliadas se levarmos em conta as
seguintes propriedades de uma fun\cao\
quadraticamente integr\'avel $g(t)$,
\begin{eqnarray}
\intii g^2(t) dt = \intii h^2(t) dt, \hspace{1cm}
\intii g(t) h(t) dt = 0 \; ,
\end{eqnarray}
onde $h(t)$ denota a transformada de Hilbert de
$g(t)$. As igualdadades acima s\ao\ obtidas, utilizando
a equa\cao\ \refi{thth}, da seguinte forma
\begin{eqnarray}
\intii h^2(t) dt &=& \intii h(t) \left[\frac{1}{\pi} \intii 
\frac{g(\xi)}{(\xi-t)} d\xi \right] dt \nonumber \\
&=& \intii g(\xi) \left[
\frac{1}{\pi} \intii \frac{h(t)}{(\xi-t)} dt
\right] d\xi \nonumber \\
&=& \intii g(\xi) \left[-\frac{1}{\pi} \intii
\frac{h(t)}{(t-\xi)} dt \right] d\xi \nonumber \\
&=& \intii g(\xi)g(\xi) d\xi \; = \; \intii g^2(t) dt \; ,
\end{eqnarray}
e
\begin{eqnarray}
\intii g(t) h(t) dt &=& \intii g(t) \left[ \frac{1}{\pi}
\intii \frac{g(\xi)}{t-\xi} d\xi \right] dt \nonumber \\
& = &
\intii g(\xi) \left[-\frac{1}{\pi}
\intii \frac{g(t)}{\xi-t} dt \right] d\xi \nonumber \\
&=& -\intii g(\xi) h(\xi) d\xi 
\; = \; - \intii g(t) h(t) dt \; .
\end{eqnarray}

Assim, para as integrais em \refi{integs}, conclu\imos\ que
\begin{eqnarray}
\intii F^2(t) dt 
& = &\intii[g(t) + ih(t)]^2 dt \nonumber \\
& = &
\intii [g^2(t)-h^2(t)]dt + 2i\intii g(t)h(t) dt \nonumber \\
& = &0 \; .
\end{eqnarray}
Similarmente, pode ser mostrado que
\begin{eqnarray}
\intii (F^{*}(t))^2 dt = 0 \; , \hspace{1cm}
\intii F(t)F^{*}(t) dt = 2 \intii g^2(t) dt \; .
\end{eqnarray}

Ent\ao, usando os resultados acima e as express\oes\
de \refii{enerdef}{fluxo}\, podemos achar as
quantidades integradas no tempo $\hat{W}$, $\hat{K}$,
$\hat{E}$ e $\hat{S}_i$ como segue
\begin{eqnarray}
\hat{W} = \intii W dt = \frac{\kappa}{8}
\intii(PF + P^{*}F^{*})^2 dt =
\frac{\kappa}{2} P P^{*} f_a \; ,
\end{eqnarray}
onde $f_a = \intii g^2(t) dt$. Correspondentemente,
chegamos que 
\begin{eqnarray}
\hat{K} = \frac{\kappa}{2} P P^{*} f_a \;
\hspace{1cm}\mbox{e} \hspace{1cm}
\hat{S}_i = \frac{p_i}{\rho}
 P P^{*} f_a \; .
\end{eqnarray} 
Vemos que, como esperado, a energia de deforma\cao\
e a energia cin\'etica integradas no tempo s\ao\
iguais, logo temos
\begin{eqnarray}
\hat{E} = \kappa P P^{*} f_a \; .
\end{eqnarray}

Temos tamb\'em que a velocidade de grupo $v_i^{(g)}$,
definida aqui como a velocidade do fluxo de energia
integrado no tempo, \'e dada por
\begin{eqnarray}
v_i^{(g)} = \frac{\hat{S}_i}{\hat{E}} = c^2 p_i
= c N_i \; .
\end{eqnarray}

De forma similar, podemos obter as quantidades
m\'edias temporais $\overline{W}$, $\overline{K}$,
$\overline{E}$ e $\overline{S}_i$ s\'o que agora
com a integral $f_a$ definida por
\begin{eqnarray}
f_A = \frac{1}{T} \int_{t}^{t+T} g^2(t) dt \; .
\end{eqnarray}


\subsection{Meio anisotr\'opico homog\^eneo}

Nesta se\cao\ trabalhamos com a parte real do vetor
deslocamento escrito na forma
\begin{eqnarray}
u_i = \frac{1}{2}(U_iF + U_i^{*}F^{*}) = 
\frac{1}{2} (AF + A^{*}F^{*}) g_i \; ,
\end{eqnarray}
onde $A$ \'e um escalar possivelmente complexo e
$g_i$ \'e um vetor unit\'ario especificando a
polariza\cao\ da onda considerada. Note tamb\'em
que aqui estamos o mesmo s\imbolo\ $u_i$ para
denotar a parte real do vetor deslocamento $u_i$.

As quantidades $W$, $K$, $E$ e $S_i$ podem ser
escritas em termos de $A$, $F$, $F'$ (onde
$F'(\xi)=dF/d\xi$ \'e tamb\'em um sinal anal\itico)
e $g_i$ como segue
\begin{eqnarray}
W &=& \frac{1}{2}\rho a_{ijkl} u_{i,j} u_{k,l}
= \frac{1}{8}\rho a_{ijkl} c^{-2} (U_i N_j F' + 
U_i^{*} N_j F^{*'})(U_k N_l F'+U_k^{*} N_l F^{*'})
\nonumber\\
&=& \frac{1}{8}\rho c^{-2} \Gamma_{ik}(AF' +
A^{*}F^{*'})^2 g_i g_k \; , \\
K &=& \frac{1}{2}\rho\dot{u_i}\dot{u_i} =
\frac{1}{8}\rho(AF' + A^{*}F^{*'})^2 \\
S_i &=& -\rho a_{ijkl} u_{k,l}\dot{u_j} = \frac{1}{4}
\rho a_{ijkl} c^{-1} N_l (AF'+ A^{*}F^{*'})^2
g_j g_k \; .
\end{eqnarray}

Se considerarmos a equa\cao\ de Christoffel,
$(\Gamma_{ik} - c^2 \delta_{ik}) g_k = 0$,
multiplicada pelo vetor $g_i$, chegamos a
\begin{eqnarray}
\Gamma_{ik} g_i g_k - c^2 = 0 \; ,
\end{eqnarray}
de onde podemos concluir que $c^{-2}\Gamma_{ik} g_i
g_k = 1$. Logo, ao substituirmos esta identidade
na express\ao\ para $W$, vemos que, como no caso
do meio ac\'ustico, $W = K$, isto \'e, que para ondas
planas propagando em meios anisotr\'opicos
homog\^eneos a energia de deforma\cao\ $W$ \'e igual
a energia cin\'etica $K$ para qualquer tempo. Assim,
para a energia el\'astica $E$ temos
\begin{eqnarray}
E = \frac{1}{4} \rho (AF'+ A^{*}F^{*'})^2 \; .
\end{eqnarray}

Para a express\ao\ da velocidade de grupo $v_i^{(g)}$
temos
\begin{eqnarray} \label{vgmah}
v_i^{(g)} = S_i/E = a_{ijkl} N_l c^{-1} g_j g_k =
a_{ijkl} p_l g_j g_k \; .
\end{eqnarray}

Note que, em constraste com o meio ac\'ustico, a
equa\cao\ \refi{vgmah}\ n\ao\ coincide com a
velocidade de fase. Al\'em disso, devido a
depend\^encia em $N_i$, podemos esperar que o valor
e a dire\cao\ da velocidade de grupo varie com a
varia\cao\ da dire\cao\ de $N_i$. Podemos concluir
tamb\'em que a energia de uma onda plana propagando
em um meio anisotr\'opico homog\^eneo propaga 
geralmente em uma dire\cao\ diferente da dire\cao\
da propaga\cao\ da frente de fase e a velocidade
da propaga\cao\ de energia \'e diferente da
velocidade de fase. Para esclarecer esta rela\cao\
entre velocidade de grupo e de fase, vamos
multiplicar \refi{vgmah}\ por $p_i=N_i/c$, logo
\begin{eqnarray}
v_i^{(g)} p_i = a_{ijkl} p_l p_i g_j g_k = c^{-2}
\Gamma_{jk} g_j g_k \; . 
\end{eqnarray}
Da equa\cao\ de Christoffel
$(\Gamma_{jk}-c^2 \delta_{jk})g_k = 0$
multiplicada por $g_j$ chegamos que
\begin{eqnarray}
\Gamma_{jk} g_j g_k = c^2 \; .
\end{eqnarray}
Agora, ao usarmos esta identidade na express\ao\ para
$v_i^{(g)} p_i$, temos
\begin{eqnarray}
v_i^{(g)}p_i = 1 \hspace{0.5cm} \Leftrightarrow 
\hspace{0.5cm} v_i^{(g)}N_i = c \; .
\end{eqnarray}

Desta equa\cao, vemos que a velocidade de grupo \'e
igual a velocidade de fase somente se
$\vec{v^{(g)}} \parallel \vec{N}$. Outra
consequ\^encia do resultado acima \'e que a
velocidade de grupo \'e sempre maior ou igual a
velocidade de fase, isto \'e, durante um intervalo
de tempo unit\'ario a frente de fase muda da
deposi\cao\ $t$ para a posi\cao\ $t+1$ e o vetor
$\vec{c}$ que corresponde a velocidade de fase \'e
perpendicular a frente de fase e \'e igual a
dist\^ancia entre duas frentes de onda, j\'a o 
vetor da velocidade de grupo geralmente n\ao\
\'e perpendicular a frente de fase e portanto
deve ser maior que a velocidade de fase.

Vejamos ent\ao\ as quantidades integradas no tempo
$\hat{W}$, $\hat{K}$, $\hat{E}$ e $\hat{S}_i$ levando
em conta que $F'$ e $F^{*'}$ tem as mesmas
propriedades que $F$ e $F^{*}$. Para energia de 
deforma\cao\ $\hat{W}$, temos
\begin{eqnarray}
\hat{W}= \intii W dt = \frac{1}{8}\rho c^{-2}
\Gamma_{ik} g_i g_k \intii (AF'+ A^{*}F^{*'})^2 dt
= \frac{1}{2} \rho c^{-2} \Gamma_{ik} g_i g_k
AA^{*} f_e \; ,
\end{eqnarray}
onde $f_e = \intii \dot{g}^2(t) dt$.
E, correspondentemente,
\begin{eqnarray}
\hat{K} = \frac{1}{2} \rho A A^{*} f_e 
\hspace{1cm} \mbox{e} \hspace{1cm}
\hat{S}_i = c^{-1} \rho a_{ijkl}
N_l g_j g_k A A^{*} f_e \; .
\end{eqnarray}

Novamente, vemos que as quantidades integradas no
tempo satisfazem $\hat{W} = \hat{K}$, logo
\begin{eqnarray}
\hat{E} = \rho A A^{*} f_e \; ,
\end{eqnarray}
e $v_i^{(g)} = \hat{S}_i/\hat{E} = c^{-1} a_{ijkl}
N_l g_j g_k$.

As express\oes\ para as quantidades
m\'edias temporais $\overline{W}$, $\overline{K}$,
$\overline{E}$ e $\overline{S}_i$ s\ao\ novamente
as mesmas que as obtidas para as integradas no
tempo, mudando apenas a integral $f_e$ por $f_E$
definida por
\begin{eqnarray}
f_E = \frac{1}{T} \int_{t}^{t+T} \dot{g}^2(t) dt \; .
\end{eqnarray}


\subsection{Meio isotr\'opico homog\^eneo}

As express\oes\ para os valores instant\^aneos de
$W$, $K$, $E$ e $S_i$ e seus valores para
tempo integrado e m\'edia temporal podem ser 
obtidos da mesma maneira que nas sec\oes\
anteriores. Em espec\ifico\ na \'ultima se\cao,
podemos trabalhar para o caso isotr\'opico
tomando
\begin{eqnarray}
a_{ijkl} = (\alpha^2 -2\beta^2)\delta_{ij}\delta_{kl}
+ \beta^2(\delta_{ik}\delta_{jl} + \delta_{il}
\delta_{jk}) \; .
\end{eqnarray}

Aqui, consideramos o vetor deslocamente
separadamente para a onda $P$
\begin{eqnarray}
u_i = \frac{1}{2} (AF + A^{*}F^{*}) N_i
\end{eqnarray}
e para onda $S$
\begin{eqnarray}
u_i = \frac{1}{2}[(BF + B^{*}F^{*}) g_i^{(1)} + 
(CF + C^{*}F^{*}) g_i^{(2)}] \; .
\end{eqnarray}

Logo, para cada um dos casos acima obtemos 
\begin{eqnarray}
W_P &=& \frac{1}{8} \rho (AF' + A^{*}F^{*'})^2 \; ,
\hspace{0.9cm}
W_S = \frac{1}{8} \rho [(BF' + B^{*}F^{*'})^2 +
(CF' + C^{*}F^{*'})^2] \; , \\
K_P &=& \frac{1}{8} \rho (AF' + A^{*}F^{*'})^2 \; ,
\hspace{0.9cm}
K_S = \frac{1}{8} \rho [(BF' + B^{*}F^{*'})^2 +
(CF' + C^{*}F^{*'})^2] \; , \\
S_{Pi}&=&\frac{1}{4}\rho\alpha N_i(AF'+A^{*}F^{*'})^2\; ,
\hspace{0.2cm}
S_{Si}=\frac{1}{4}\rho\beta N_i[(BF'+B^{*}F^{*'})^2 +
(CF'+C^{*}F^{*'})^2] , \hspace{1.2cm} \\
E_P &=& \frac{1}{4} \rho (AF' + A^{*}F^{*'})^2 \; ,
\hspace{0.9cm}
E_S = \frac{1}{4} \rho [(BF' + B^{*}F^{*'})^2 +
(CF' + C^{*}F^{*'})^2] \; ,
\end{eqnarray}
onde o \indice\ $P$ denota que a correspondente
quantidade est\'a relacionada com a ondas $P$ e o
\indice\ $S$ relacionada com a onda $S$.

Assim, as quantidades integradas no tempo s\ao\
agora
\begin{eqnarray}
& &\hat{W}_P=\hat{K}_P=\frac{1}{2}\rho AA^{*}f_e \; ,
\hspace{1.5cm}
\hat{W}_S=\hat{K}_S=\frac{1}{2}\rho (BB^{*} +
CC^{*})f_e\; , \\
& &\hat{E}_P=\rho AA^{*}f_e \; ,
\hspace{3.0cm}
\hat{E}_S=\rho (BB^{*} + CC^{*})f_e \; , \\
& &\hat{S}_{Pi}=\rho\alpha N_i AA^{*}f_e \; ,
\hspace{2.2cm}
\hat{S}_{Si}=\rho\beta N_i (BB^{*} + CC^{*})f_e \; . 
\end{eqnarray}

Como consequ\^encia das express\oes\ acima para 
quantidades instant\^aneas e integradas no tempo,
chegamos para velocidade de grupo das ondas $P$ e
$S$
\begin{eqnarray}
v_{Pi}^{(g)} = S_{Pi}/E_P = \hat{S}_{Pi}/\hat{E}_P
= \alpha N_i \; , \hspace{1cm} v_{Si}^{(g)} =
S_{Si}/E_S = \hat{S}_{Si}/\hat{E}_S = \beta N_i \; .
\end{eqnarray}
Ent\ao\ em um meio isotr\'opico homo\^geneo,
similarmente para o caso ac\'ustico, a velocidade
de grupo \'e igual a velocidade de fase, ambas
com mesmo tamanho e dire\cao. Finalmente, vamos
acrescentar que as f\'ormulas para as quantidades
m\'edias temporais $\overline{W}$, $\overline{K}$,
$\overline{E}$ e $\overline{S}_i$ podem ser
obtidas das f\'ormulas acima, para quantidades
integradas no tempo, pela substitui\cao\ da
integral $f_e$ por $f_E$.


\section[Compara\cao\ dos meios anisotr\'opico, isotr\'opico e
ac\'ustico]{Compara\cao\ da propaga\cao\ de ondas
nos meios anisotr\'opico, isotr\'opico e ac\'ustico}

A propaga\cao\ de onda em meios anisotr\'opicos e
isotr\'opico homog\^eneos diferem nas seguintes
caracter\isticas:

\begin{enumerate}
\item Para uma dire\cao\ especificada da fase normal
existe geralmente tr\^es ondas planas independentes
com diferentes velocidades de fase em um meio
anisotr\'opico, duas ondas planas independentes
em um meio isotr\'opico e uma onda plana em um meio
ac\'ustico.

\item Vetores de velocidade de fase e de grupo s\ao\
geralmente diferentes em um meio anisotr\'opico,
onde a velocidade de grupo \'e maior que a velocidade
de fase. A velocidade de grupo e a velocidade de
fase t\^em o mesmo tamanho e dire\cao\ em meio
isotr\'opico e ac\'ustico.

\item Vetores de velocidade de fase e de grupo s\ao\
geralmente angularmente dependentes em meio
anisotr\'opico. Especificamente, ambos dependem da
normal $N_i$ da frente de fase. A velocidade de fase
(e tamb\'em a velocidade de grupo) \'e a mesma em
todas as dire\coes\ em meio isotr\'opico e ac\'ustico.

\item A polariza\cao\ da onda em meio anisotr\'opico
\'e geralmente linear com a dire\cao\ de polariza\cao\
geralmente diferente da dire\cao\ da fase normal, da
tangente da frente de onda ou do vetor de velocidade
de grupo. Em meio isotr\'opico, a dire\cao\ de
polariza\cao\ \'e paralela a fase normal (ondas $P$)
ou no plano da frente de fase (ondas $S$). A
polariza\cao\ das ondas $P$ \'e linear, das ondas
$S$ \'e geralmente el\iptica\ ou quase-el\iptica.
A polariza\cao\ da onda no caso ac\'ustico \'e a 
mesma que para onda $P$ no caso isotr\'opico.

\item Singularidades da onda cisalhante ocorrem
em certas dire\coes\ em meio anisotr\'opico. Nestas
dire\coes, ondas $qS_1$ e $qS_2$ propagam com mesma
velocidade de fase e t\^em comportamento complicado.
Estas singularidades usualmente n\ao\ existem para
ondas $P$ em meio anisotr\'opico bem como para todo
tipo de onda em meio isotr\'opico e meio ac\'ustico.  

\end{enumerate}

