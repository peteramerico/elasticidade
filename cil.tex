\section{Ondas Cil\'{\i}ndricas}
\label{cil}

A onda cil\'{\i}ndrica \'e uma fun\c{c}\~ao que satisfaz a equa\c{c}\~ao
da onda ac\'ustica (\ref{eoadc}) e apresenta uma simetria
cil\'{\i}ndrica, ou seja, n\~ao apresenta varia\c{c}\~ao com o plano
horizontal. Analogamente ao caso anterior, antes de encontrar uma
solu\c{c}\~ao, devemos fazer uma mudan\c{c}a de coordenadas de
cartesianas para cil\'{\i}ndricas na equa\c{c}\~ao da onda ac\'ustica
(\ref{eoadc}).

Mudando as coordenadas $(x,y,z)$ para $(r,\phi,z)$, de tal forma
que
$$
x=r\cos{\phi}\;, \;\;\;\; y=r\sin{\phi} \;\;\;\;\mathrm{e}\;\;\;\; z=z \;,
$$
onde $0 \leq r <\infty $, $-\pi < \phi \leq \pi $, podemos obter a
equa\c{c}\~ao da onda ac\'ustica em coordenadas cil\'{\i}ndricas:
\begin{equation}
\nabla^2 \Phi = \frac{\partial^2 \Phi}{\partial r^2} +
\frac{1}r{}\frac{\partial \Phi}{\partial r} +
\frac{1}{r^2}\frac{\partial^2 \Phi}{\partial
\phi^2}+\frac{\partial^2 \Phi}{\partial z^2}=
\frac{1}{c^2}\frac{\partial^2 \Phi}{\partial t^2} \;.
\label{eq_onda_cil}
\end{equation}
Como procuramos solu\c{c}\~oes que j\'a s\~ao cil\'{\i}ndricas,
isto \'e, n\~ao dependam do \^angulo $\phi$ e varie conforme
aumenta o raio (an\'alogo ao caso esf\'erico), a equa\c{c}\~ao
(\ref{eq_onda_cil}) toma a forma
\begin{equation}
\frac{\partial^2 \Phi}{\partial r^2} + \frac{1}{r}\frac{\partial
\Phi}{\partial r} = \frac{1}{c^2}\frac{\partial^2 \Phi}{\partial
t^2} \;.
\label{eq_cil}
\end{equation}
A solu\c{c}\~ao geral para este problema \'e aproximadamente
\begin{equation}
\Phi(r,t) = \frac{1}{\sqrt{r}}f(kr-\omega t) \;.
\label{sol_cil_apr}
\end{equation}
Dizemos aproximadamente pois, ao calcularmos as derivadas e
substituirmos estas na equa\c{c}\~ao (\ref{eq_cil}), fica
sobrando um fator da ordem de $r^{-5/2}$, o qual pode ser
desprezado para valores grandes de $r$. A solu\c{c}\~ao exata da
referida equa\c{c}\~ao \'e dada por
\begin{equation}
\Phi(r,t) = \frac{H(t-r/c)}{\sqrt{t^2-(r/c)^2}}\ast f(t)\;,
\label{sol_cil_exata}
\end{equation}
onde $H$ \'e a fun\c{c}\~ao degrau (ou fun\c{c}\~ao Heaviside):
\begin{equation}
H(a)= \left \{
\begin{array}{ll}
0, & se  \; \; a < 0 \\
1, & se \; \;   a \geq 0
\end{array}  \right..
\label{heaviside}
\end{equation}

Assim como no caso da onda esf\'erica, podemos entender o
decaimento da amplitude da onda apenas considerando a
conserva\c{c}\~ao de energia. Como a \'area do cilindro \'e
proporcional a $r$, o fluxo de energia decai segundo a raz\~ao
$1/r$. Portando,  a amplitude da onda deve variar $1/\sqrt{r}$.
