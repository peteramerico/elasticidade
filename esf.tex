\section{Ondas Esf\'ericas}
\label{esf}

A onda esf\'erica \'e a solu\c{c}\~ao da equa\c{c}\~ao da onda
ac\'ustica (\ref{eoadc}) que apresenta simetria esf\'erica, ou seja,
n\~ao apresenta varia\c{c}\~ao com o \^angulo nos planos horizontal e
vertical. Sendo assim, o campo de onda depende somente da dist\^ancia em
rela\c{c}\~ao \`a origem. Logo, antes de encontrar uma solu\c{c}\~ao,
devemos fazer uma mudan\c{c}a de coordenadas de cartesianas para
esf\'ericas na equa\c{c}\~ao da onda ac\'ustica (\ref{eoadc}).

Mudando as coordenadas $(x,y,z)$ para $(r,\theta,\phi)$, de tal forma
que
$$
x=r\sin{\theta}\cos{\phi}\;, \;\;\;\; y=r\sin{\theta}\sin{\phi}
\;\;\;\;\mathrm{e}\;\;\;\; z=r\cos{\theta} \;,
$$
onde $0 \leq r <\infty $, $-\pi < \phi \leq \pi $,
$0 \leq \theta \leq \pi $, podemos obter a
equa\c{c}\~ao da onda ac\'ustica (\ref{eoadc}) em coordenadas esf\'ericas:
\begin{equation}
\nabla^2 \Phi = \frac{\partial^2 \Phi}{\partial r^2} +
\frac{2}r{}\frac{\partial \Phi}{\partial r} +
\frac{1}{r^2}\frac{\partial^2 \Phi}{\partial \theta^2} +
\frac{\cot{\theta}}{r^2}\frac{\partial \Phi}{\partial \theta} +
\frac{1}{r^2 \sin^2{\theta}}\frac{\partial^2 \Phi}{\partial
\phi^2} = \frac{1}{c^2}\frac{\partial^2 \Phi}{\partial t^2} \;.
\label{onda_esf}
\end{equation}
Como supusemos que $\Phi$ possui simetria esf\'erica, \'e fun\c{c}\~ao
somente da dist\^ancia $r$ em rela\c{c}\~ao \`a fonte e do tempo $t$.
Portanto, as suas derivadas parciais em rela\c{c}\~ao aos \^angulos
s\~ao nulas. Logo, a equa\c{c}\~ao da onda (\ref{onda_esf} toma a forma
\begin{equation}
\frac{\partial^2 \Phi}{\partial r^2} + \frac{2}{r}\frac{\partial
\Phi}{\partial r} = \frac{1}{c^2}\frac{\partial^2 \Phi}{\partial
t^2} \;.
\label{onda_esf_menor}
\end{equation}


Para encontrar uma solu\c{c}\~ao para este problema, primeiramente
observemos que
\begin{equation}
%\begin{array}{ccc}
\frac{\partial^2 \Phi}{\partial r^2} + \frac{2}r{}\frac{\partial
\Phi}{\partial r} =\frac{1}{r} \left (r \frac{\partial^2
\Phi}{\partial r^2} + 1\frac{\partial \Phi}{\partial r} +
\frac{\partial \Phi}{\partial r} \right) = \frac{1}{r}
\frac{\partial}{\partial r}\left( r\frac{\partial \Phi}{\partial
r} +\Phi \right)= \frac{1}{r} \left( \frac{\partial^2
(r\Phi)}{\partial r^2} \right) \;,
%\end{array}
\label{onda_esf_dem11}
\end{equation}
de onde podemos escrever que
\begin{equation}
%\begin{array}{ccc}
\frac{1}{r} \left( \frac{\partial^2 (r\Phi)}{\partial r^2} \right)
= \frac{1}{c^2} \frac{\partial^2 \Phi}{\partial t^2}\;,
%\end{array}
\label{onda_esf_dem1}
\end{equation}
ou ainda,
\begin{equation}
%\begin{array}{ccc}
\left( \frac{\partial^2 (r\Phi)}{\partial r^2} \right) =
\frac{1}{c^2} \frac{\partial^2 (r\Phi)}{\partial t^2}\;.
%\end{array}
\label{onda_esf_dem2}
\end{equation}
Ent\~ao, a fun\c{c}\~ao $r\Phi$ deve satisfazer a equa\c{c}\~ao da
onda unidimensional, conforme visto anteriormente. Logo,
\begin{equation}
r\Phi = f(kr-\omega t) \;\;\;\;\Rightarrow \;\;\;\; \Phi =
\frac{1}{r}f(kr-\omega t) \;. \label{sol_esf}
\end{equation}

Vamos agora justificar o decaimento da amplitude da solu\c{c}\~ao
(\ref{sol_esf}). Fluxo \'e a quantidade de energia que passa por
uma unidade de \'area em uma unidade de tempo, e para este fluxo
de energia em ondas, o princ\'{\i}pio de conserva\c{c}\~ao de
energia se aplica. No caso da onda plana o fluxo \'e constante,
j\'a que o fluxo de energia atravessa sempre a mesma \'area. J\'a
no caso da onda esf\'erica uniforme de raio $r$, a energia passa a
ser distribu\'{\i}da sobre a superf\'{\i}cie da esfera, cuja
\'area aumenta conforme $4\pi r^2$. Mas a energia total
transportada pela onda esf\'erica deve permanecer constante. Logo,
o fluxo de energia num ponto da superf\'{\i}cie diminui segundo a
raz\~ao $1/r^2$. Como o fluxo energ\'etico \'e proporcional ao
quadrado da amplitude, a amplitude deve ent\~ao variar $1/r$.
