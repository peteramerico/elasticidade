\chapter{Ondas Planas} \label{ondplan}

Vamos estudar as propriedades da
propaga\c{c}\~ao de ondas planas em meio
ac\'ustico homog\^eneo ou perfeitmente el\'astico, anisotr\'opico
ou isotr\'opico. Ondas planas representam uma das poss\'{\i}veis
solu\c{c}\~oes das equa\c{c}\~oes da onda e elastodin\^amica para
meio homog\^eneo. Essas ondas n\~ao existem em meios reais, mas
elas s\~ao boas aproxima\c{c}\~oes de ondas para fontes distantes.
Uma das vantagens de trabalhar com ondas planas \'e que podemos
estuda-las sem considerar sua fonte.

\section{Propriedades das ondas planas}

\subsection{Onda planas harm\^onicas no tempo}

Queremos encontrar solu\c{c}\~oes das equa\c{c}\~oes
elastodin\^amica e ac\'ustica homog\^enea na seguinte forma,
\begin{equation}
w(x_m,t)=W\exp[-i\omega(t-T(x_m))] \;,
\end{equation}
onde $w$ pode ser tamb\'em um escalar (no caso em que este
representa press\~ao) ou uma componente de um vetor (no caso em
que este representa a componente de velocidade de uma
part\'{\i}cula ou vetor de deslocamento).

O s\'{\i}mbolo $W$ \'e uma constante, que pode ser complexa,
chamada de {\it escalar} ou {\it amplitude vetorial} dependendo do
significado de $w$. O s\'{\i}mbolo $\omega$ denota a {\it
freq\"u\^encia angular}, $\omega = 2\pi f$. Aqui, $f$ \'e a {\it
freq\"u\^encia}, isto \'e, o n\'umero de oscila\c{c}\~oes da onda
por segundo. A freq\"u\^encia $f$ est\'a relacionada com o {\it
per\'{\i}odo} $T$, isto \'e, o tempo de uma oscila\c{c}\~ao, pela
rela\c{c}\~ao $f = 1/T$. O per\'{\i}odo $T$ \'e medido em segundos
($s$), $f$ em Hertz (Hz=$s^{-1}$). O s\'{\i}mbolo $t$ denota tempo
e $T(x_m)$ \'e uma fun\c{c}\~ao real linear homog\^enea de $x_i$,
ou seja,
\begin{equation}
T(x_m)=p_i x_i \;,
\end{equation}
onde $p_i$ s\~ao coeficientes reais constantes. A equa\cao\
\begin{equation}
t-T(x_m)=t-p_i x_i=cte.
\end{equation}
para $t$ fixo, \'e uma equa\cao\ de um plano e, simultaneamente,
uma equa\cao\ de fase constante da fun\cao\ exponencial,
na express\~ao para $w$. Note que, nesse plano, $w$ \'e constante.
Ent\~ao, o plano $p_i x_i=t+cte.$ \'e chamado de frente de onda, e
a onda associada a ela, \'e chamada de onda plana.
Ao variar o tempo $t$, a frente de onda se move de forma que
a normal $N_i$ ao plano, chamada de fase normal, n\~ao muda
sua dire\cao. A diferencia\cao\ da equa\cao\ de fase constante
com respeito ao par\^ametro $\xi$ de comprimento, ao longo de
uma linha perpendicular a fase nos permite
\begin{equation}
\frac{dt}{d\xi}=p_i \frac{dx_i}{d\xi}=p_i N_i .
\end{equation}
Aqui, $\frac{dt}{d\xi}=c^{-1}$, onde $c$ \'e a velocidade com que a frente
de onda se move e \'e, portanto, chamado de velocidade
de fase. Ent\~ao, o vetor $p_i$ pode ser expresso como
\begin{equation}
p_i =N_i / c , \ \ \ \  i.e. \ \ \ \  p_k p_k = c^{-2}.
\end{equation}
Uma vez que o tamanho do vetor $p_i$ \'e inversamente
proporcional a velocidade de fase $c$, o vetor $p_i$
\'e chamado de vetor vagarosidade. Podemos agora
introduzir o termo comprimento de onda $\lambda$
como o comprimento percorrido por uma onda plana
com velocidade $c$ e per\'iodo $T$,
\begin{equation}
\lambda=cT=\frac{c}{f}=\frac{2\pi c}{w} = \frac{2\pi}{k} .
\end{equation}
Aqui, $k=\frac{w}{c}$ \'e chamado de comprimento de onda.
No caso de ondas harm\^onicas, o vetor vagarosidade \'e,
as vezes, substituido pelo vetor de onda $k_i$,
\begin{equation}
k_i=wp_i .
\end{equation}

\subsection{Onda planas no dom\'{\i}nio do tempo (ondas planas
transientes)}

Podemos obter solu\c{c}\~oes para ondas planas no dom\'{\i}nio do
tempo aplicando a trasformada de Fourier nas solu\c{c}\~oes
obtidas no dom\'{\i}nio da frequ\^encia ou usando diretamente
os sinais transientes na
solu\c{c}\~ao experimental da equa\c{c}\~ao elastodin\^amica,
que \'e a forma como faremos aqui.
Esta aproxima\c{c}\~ao est\'a conectada com o {\it sinal
anal\'{\i}tico}. Vamos procurar solu\c{c}\~oes na forma
\begin{equation}
w(x_m,t)=W F(t-T(x_m)) \;.
\end{equation}
A fun\c{c}\~ao $F(\xi)$ representa o sinal anal\'{\i}tico, e ela
est\'a definida como
\begin{equation}
F(\xi)=g(\xi)+ih(\xi) \;,
\end{equation}
onde $g(\xi)$ \'e sinal transiente para o qual o sinal
anal\'{\i}tico est\'a constru\'{\i}do, e $h(\xi)$ \'e a
transformada de Hilbert de $g(\xi)$.

No caso do sinal transiente, \'e mais conveniente trabalhar com o
sinal anal\'{\i}tico
\begin{equation}
F(t)=2 \int_0^{\infty} g(f) \mathrm{e}^{-i2\pi f t} df \;,
\end{equation}
do que trabalhar com o sinal real $g(t)$. Usando o sinal
anal\'{\i}tico $F(t)$, \'e poss\'{\i}vel construir um envelope do
sinal real $g(t)$. O envelope \'e dado como
\begin{equation}
|F(t)|=\sqrt{g^2(t)+h^2(t)} \;.
\end{equation}
O envelope da onda plana transiente $w = WF$, pode ent\~ao ser
escrito como
\begin{equation}
|w(t)|=|W|\sqrt{g^2(t)+h^2(t)} \;.
\end{equation}
Podemos ver que a forma do envelope \'e a mesma em qualquer lugar,
n\~ao importando se $W$ assume valor constante complexo ou real. A
forma da onda plana transiente $w(t)$, por\'em, pode mudar se $W$
assumir valor complexo, ou seja,
\begin{equation}
Re(w(t))=Re(WF(t))=Re(W)g(t)-Im(W)h(t) \;.
\end{equation}
No caso em que $W$ assume valor real, temos
\begin{equation}
Re(w(t))=Wg(t) \;,
\end{equation}
o que nos mostra que neste caso, a forma da onda plana transiente
$w(t)$ \'e preservada.

\section{Ondas planas ac\'usticas transientes}

Vamos inserir a tentativa de solu\c{c}\~ao
\begin{equation}
p(x_m,t)=PF(t-T(x_m)) \;,
\end{equation}
\begin{equation}
v_i(x_m,t)=V_iF(t-T(x_m)) \;,
\end{equation}
nas equa\c{c}\~oes ac\'usticas especificadas para $f_i=0$,
\begin{equation}
p_{,i}+\rho\frac{\partial v_i}{\partial t}=0 \;\;\;,\;\;\;
v_{k,k}+\kappa\frac{\partial p}{\partial t}=0 \;.
\end{equation}
Fazendo isto, obtemos dois resultados importantes
\begin{equation}
c=(\rho \kappa)^{-1/2} \;\;\;,\;\;\; V_i=\frac{P}{\rho c}N_i \;.
\end{equation}
A primeira equa\c{c}\~ao expressa velocidade da fase em termos dos
par\^ametros que descrevem o meio ac\'ustico. A segunda
equa\c{c}\~ao especifica a dire\c{c}\~ao e magnitude da quantidade
vetorial $V_i$.

A solu\c{c}\~ao da onda plana para o caso ac\'ustico tem ent\~ao a
forma
\begin{equation}
p(x_m,t)=PF(t-T(x_m)) \;\;\;,\;\;\; v_i(x_m,t)=\frac{P}{\rho c}N_i
F(t-T(x_m)) \;,
\end{equation}
onde
\begin{equation}
T(x_m)=\frac{x_i N_i}{c} \;\;\;,\;\;\; c=(\rho \kappa)^{-1/2} \;.
\end{equation}

\section{Ondas planas el\'asticas transientes}

\subsection{Meio anisotr\'opico homog\^eneo}

Vamos inserir a tentativa de solu\c{c}\~ao
\begin{equation}
u_i(x_m,t)=U_iF(t-T(x_m)) \;\;\;,\;\;\; T(x_m)=\frac{x_i N_i}{c}
\;,
\end{equation}
na equa\c{c}\~ao elastodin\^amica para um meio anisotr\'opico
homog\^eneo com $f_i=0$,
\begin{equation}
a_{ijkl}u_{k,lj}=u_{i,tt} \;,
\end{equation}
onde $a_{ijkl}=c_{ijkl}/\rho$ (os par\^ametros $a_{ijkl}$ s\~ao
chamados de {\it par\^ametros el\'asticos de densidade
normalizada}, e a dimens\~ao deles \'e $(m/s)^2$). Logo, obtemos
\begin{equation}
a_{ijkl}U_k p_l p_j - U_i=0 \;.
\end{equation}
Usando que $p_i=N_i/c$, podemos reescrever esta equa\c{c}\~ao como
a chamada {\it equa\c{c}\~ao de Christoffel}:
\begin{equation}
(\Gamma_{ik}-c^2\delta_{ik})U_k=0 \;,
\end{equation}
onde $\Gamma_{ik}=a_{ijkl}N_lN_j$ \'e a {\it matriz de
Christoffel}. Podemos ver que a equa\c{c}\~ao de Christoffel tem a
forma da equa\c{c}\~ao que resolve o problema de autovalores para
a matriz $\Gamma_{ik}$, ou seja, encontrar os autovalores de
$\Gamma_{ik}$ que s\~ao iguais a $c^2$ e autovetores $g_j$ ($g_j
g_j =1$) proporcionais ao vetor constante $U_j$ ($U_j=Ag_j$).

Vejamos algumas propriedades importandes da matriz de Christoffel
$\Gamma_{ik}$. De sua defini\c{c}\~ao, temos que ela \'e
sim\'etrica. Esta matriz tamb\'em \'e definida positiva. Para
provar isto, temos que mostrar que para qualquer vetor $d_i$, vale
$\Gamma_{ik}d_id_k>0$. Observemos que
\begin{equation}
\Gamma_{ik}d_i d_k = a_{ijkl}N_jN_ld_id_k = a_{ijkl}b_{ij}b_{kl}
\;, \label{Christ_DP}
\end{equation}
onde $b_{ij}=d_iN_j$. Mostramos anteriormente, que a energia de
deslocamento $W$ \'e sempre positiva,
\begin{equation}
\frac{1}{2}c_{ijkl}e_{ij}e_{kl}>0 \;,
\end{equation}
o que implica automaticamente que
\begin{equation}
\frac{1}{2}a_{ijkl}e_{ij}e_{kl}>0 \;.
\end{equation}
Esta rela\c{c}\~ao \'e satisfeita para um tensor sim\'etrico
arbitr\'ario de deslocamento pequeno. Portanto, escrevemos
\begin{equation}
b_{ij}=\frac{1}{2}(b_{ij}+b_{ji})+\frac{1}{2}(b_{ij}-b_{ji})=b_{ij}^S+b_{ij}^A
\;.
\end{equation}
Assim, a express\~ao (\ref{Christ_DP}) pode ser reescrita como
\begin{equation}
a_{ijkl}(b_{ij}^S+b_{ij}^A)(b_{kl}^S+b_{kl}^A)=a_{ijkl}
b_{ij}^Sb_{kl}^S\;.
\end{equation}
Todos os termos contendo $b_{ij}^A$ s\~ao nulos devido \`a sua
anti-simetria e \`a simetria de $a_{ijkl}$. A matriz de
Christoffel \'e portanto definida positiva.

Das propriedades citadas, essa matriz tem tr\^es autovalores
positivos e reais. Desde que os autovalores $c^2$ s\~ao positivos,
temos tr\^es pares de valores reais $\pm c$, os quais correspondem
a tr\^es pares de ondas propagando num meio anisotr\'opico. Cada
par consiste em duas ondas propagando com a mesma velocidade de
fase $c$, por\'em em sentidos opostos. Por enquanto, vamos
considerar apenas o caso $c$ positivo. Portanto, num meio
homog\^eneo anisotr\'opico, na dire\c{c}\~ao especificada pela
normal de fase $N_i$, tr\^es ondas geralmente independentes podem
propagar. Suas velocidades de fase podem ser encontradas
resolvendo
\begin{equation}
\mathrm{det}(\Gamma_{ik}-c^2\delta_{ik})=0 \;.
\end{equation}

Como vimos anteriormente, se todos os tr\^es autovalores s\~ao
diferentes, os autovetores correspondentes podem ser determinados
de maneira \'unica. Outro resultado obtido, foi que eles s\~ao
mutuamente perpendiculares se a matriz correspondente for real e
sim\'etrica. Isto quer dizer que as tr\^es diferentes ondas
propagando no meio anisotr\'opico diferem n\~ao apenas na
velocidade de fase, mas tamb\'em em suas {\it polariza\c{c}\~oes}
(orienta\c{c}\~ao dos autovetores), que especificam as
dire\c{c}\~oes do vetor de movimento da part\'{\i}cula $u_i$.
Part\'{\i}culas se movem ao longo de linhas especificadas pelos
autovetores. Este tipo de polariza\c{c}\~ao \'e chamado portanto
de {\it polariza\c{c}\~ao linear}. Se todas as tr\^es ondas
propagam na mesma dire\c{c}\~ao (suas frentes de fase s\~ao
paralelas), ent\~ao os vetores de movimento da part\'{\i}cula
correspondendo a estas tr\^es ondas, s\~ao mutuamente
perpendiculares.

A situa\c{c}\~ao \'e diferente quando dois dos autovalores da
matriz $\Gamma_{ik}$ coincidem (caso degenerado). Ent\~ao os
autovetores correspondentes aos autovalores coincidentes n\~ao
podem ser determinados de maneira \'unica. Eles est\~ao situados
no plano perpendicular ao terceiro autovetor. Quaisquer dois
vetores mutuamente perpendiculares neste plano podem ser
escolhidos como autovetores. As dire\c{c}\~oes $N_i$, para a qual
dois autovalores de $\Gamma_{ik}$ coincidem s\~ao chamadas {\it
singulares}. No meio isotr\'opico, isto ocorre para qualquer
dire\c{c}\~ao.

A solu\c{c}\~ao da onda plana da equa\c{c}\~ao elastodin\^amica
para o meio anisotr\'opico homog\^eneo tem a forma
\begin{equation}
u_i^{(l)}(x_m,t)=A^{(l)}G_i^{(l)}F(t-T^{(l)}(x_m)) \;\;\;,\;\;\;
T^{(l)}=\frac{x_i N_i}{c^{(l)}} \;,
\end{equation}
onde $l=1,2,3$, $A^{(l)}$ \'e uma constante arbitr\'aria,
$c^{(l)}$ \'e a velocidade de fase e $G_i^{(l)}$ \'e o vetor de
movimento da part\'{\i}cula normalizado correspondente a uma das
tr\^es ondas que propagam no meio anisotr\'opico homog\^eneo.

\subsection{Meio isotr\'opico homog\^eneo}

Podemos procurar os coeficientes da tentativa de solu\c{c}\~ao da
onda plana inserindo estes na equa\c{c}\~ao elastodin\^amica para
o meio isotr\'opico homog\^eneo. Assim, vamos inserir a tentativa
de solu\c{c}\~ao
\begin{equation}
u_i(x_m,t)=U_iF(t-T(x_m)) \;\;\;,\;\;\; T(x_m)=\frac{x_i N_i}{c}
\;,
\end{equation}
na equa\c{c}\~ao elastodin\^amica para o meio isotr\'opico
homog\^eneo com $f_i=0$:
\begin{equation}
(\lambda+\mu)u_{k,ki}+\mu u_{i,kk} = \rho u_{i,tt} \;,
\end{equation}
de onde obtemos
\begin{equation}
(\lambda+\mu)U_kp_kp_i+\mu U_ip_kp_k-\rho U_i=0 \;.
\end{equation}
Inserindo $p_i=N_i/c$ nesta equa\c{c}\~ao, podemos escrever
\begin{equation}
\left[\frac{\lambda+\mu}{\rho}N_kN_i+\left(\frac{\mu}{\rho}-c^2
\right)\delta_{ik} \right] U_k=0 \;.
\end{equation}

Esta \'e a equa\c{c}\~ao de Christoffel para o meio isotr\'opico
homog\^eneo. An\'alogamente ao caso anterior, temos que resolver o
determinante
\begin{equation} \left |
\begin{array}{ccc}
\frac{\lambda+\mu}{\rho}N_1^2+\left(\frac{\mu}{\rho}-c^2 \right) &
\frac{\lambda+\mu}{\rho}N_1N_2 &
\frac{\lambda+\mu}{\rho}N_1N_3\\
\frac{\lambda+\mu}{\rho}N_1N_2 &
\frac{\lambda+\mu}{\rho}N_2^2+\left(\frac{\mu}{\rho}-c^2 \right) &
\frac{\lambda+\mu}{\rho}N_2N_3\\
\frac{\lambda+\mu}{\rho}N_1N_3 & \frac{\lambda+\mu}{\rho}N_2N_3  &
\frac{\lambda+\mu}{\rho}N_3^2+\left(\frac{\mu}{\rho}-c^2 \right)
\end{array} \right| =0 \;,
\label{M}
\end{equation}
que \'e uma equa\c{c}\~ao c\'ubica em $c^2$. Para determinar as
tr\^es ra\'{\i}zes, podemos desenvolver o determinante para $N_i$
escolhido arbitrariamente. Como o meio investigado \'e
isotr\'opico, a solu\c{c}\~ao obtida para um $N_i$ escolhido vale
para qualquer outro $N_i$ especificado. Vamos escolher a normal de
fase como $\vec{N}=(1,0,0)$. O determinante ent\~ao se reduz a
\begin{equation}
\left(\frac{\lambda+2\mu}{\rho}-c^2 \right)
\left(\frac{\mu}{\rho}-c^2 \right)^2 = 0 \;.
\end{equation}

Desta equa\c{c}\~ao, obtemos que dois dos tr\^es autovalores $c^2$
s\~ao iguais. Escolhendo as dire\c{c}\~oes positivas novamente,
temos
\begin{equation}
c^{(1)}=c^{(2)}=\sqrt{\frac{\mu}{\rho}} \;\;\;,\;\;\; c^{(3)}=
\sqrt{\frac{\lambda+2\mu}{\rho}}\;.
\end{equation}
Ent\~ao, no meio isotr\'opico, apenas duas ondas diferentes podem
propagar: a mais r\'apida com velocidade de fase $\alpha=c^{(3)}$,
e a outra com velocidade de fase $\beta=c^{(1)}$. Como visto
anteriormente, elas s\~ao chamadas de onda P e onda S,
respectivamente.

Vamos agora determinar a polariza\c{c}\~ao de ambas as ondas, isto
\'e, vamos achar os autovetores correspondentes aos autovalores
$\alpha^2$ e $\beta^2$. Para isso, vamos multiplicar a
equa\c{c}\~ao de Christoffel pelo vetor $U_i$. Logo,
\begin{equation}
\frac{\lambda+\mu}{\rho}(N_kU_k)(N_iU_i)+\left(\frac{\mu}{\rho}-c^2
\right)(U_iU_i)=0 \;.
\end{equation}
Vamos investigar primeiramente a onda S. Neste caso os autovalores
s\~ao coincidentes e assim, os autovetores n\~ao s\~ao
determinados de maneira \'unica. Para $c^2 = \beta^2$, obtemos
\begin{equation}
\frac{\lambda+\mu}{\rho}(N_kU_k)^2=0 \;.
\end{equation}
Como $(\lambda+\mu)/\rho>0$, ent\~ao $N_kU_k=0$, o que nos diz que
a polariza\c{c}\~ao da onda S \'e perpendicular \`a dire\c{c}\~ao
de propaga\c{c}\~ao especificada pela normal de fase $N_i$. Assim,
o vetor $U_k$ est\'a situado no plano da frente de fase. Quaisquer
dois vetores unit\'arios mutuamente perpendiculares que estejam
neste plano, podem ser escolhidos como autovetores $g_i^{(1)}$ e
$g_i^{(2)}$.

Desta observa\c{c}\~ao, podemos concluir que o autovetor
$g_i^{(3)}$ corresponde \`a propaga\c{c}\~ao da onda P na
dire\c{c}\~ao $N_i$, e sendo assim ele \'e paralelo a $N_i$. Para
confirmarmos isto, podemos inserir $c^2=\alpha^2$ na equa\c{c}\~ao
de Christoffel multiplicada por $U_i$. Logo,
\begin{equation}
\frac{\lambda+\mu}{\rho}[(N_kU_k)^2-(U_kU_k)]=0 \;.
\end{equation}
Como $(\lambda+\mu)/\rho>0$, ent\~ao $(N_kU_k)^2=U_kU_k$, o que
\'e verdade apenas para $U_k$ paralelo a $N_k$.

A solu\c{c}\~ao da onda plana P para a equa\c{c}\~ao
elastodin\^amica para o meio isotr\'opico homog\^eneo pode ser
escrita como
\begin{equation}
u_i(x_m,t)=AN_iF\left(t-\frac{N_ix_i}{\alpha} \right) \;,
\end{equation}
onde $A$ \'e uma constante arbitr\'aria, $N_i$ \'e a normal de
fase e $\alpha$ \'e a velocidade de fase da onda P. A
solu\c{c}\~ao da onda plana S tem a forma
\begin{equation}
u_i(x_m,t)=(Bg_i^{(1)}+Cg_i^{(2)})F\left(t-\frac{N_ix_i}{\beta}
\right) \;,
\end{equation}
onde $B$ e $C$ s\~ao constantes arbitr\'arias e $\beta$ \'e a
velocidade de fase da onda S.

Destas duas solu\c{c}\~oes, podemos ver que part\'{\i}culas da
onda P sempre se movem ao longo de uma linha paralela a $N_i$,
isto \'e, ao longo de uma linha perpendicular \`a frente de fase.
A polariza\c{c}\~ao da onda P \'e ent\~ao linear. J\'a as
part\'{\i}culas da onda S, sempre se movem num plano perpendicular
a $N_i$. Vamos considerar o sinal anal\'{\i}tico $F(\xi)$ como
sendo $\exp{(-i\omega \xi)}$. Ent\~ao podemos reescrever a
solu\c{c}\~ao da onda plana S como
\begin{equation}
u_i(x_m,t)=(Bg_i^{(1)}+Cg_i^{(2)})\exp\left[-i\omega
\left(t-\frac{N_ix_i}{\beta} \right)\right] \;.
\end{equation}
Vamos olhar o caminho da part\'{\i}cula espeficicado pelo vetor
$Re(u_i)$. Denotando as partes reais das componentes do vetor
$u_i$ por $u_1$ e $u_2$ (correspondentes aos vetores $g_i^{(1)}$ e
$g_i^{(2)}$), obtemos
\begin{equation}
u_1(x_m,t)=|B| \cos [\omega(t-T)-\varphi_B] \;,
\end{equation}
\begin{equation}
u_2(x_m,t)=|C| \cos [\omega(t-T)-\varphi_c] \;,
\end{equation}
onde $T=(N_ix_i)/\beta$ e usamos que $B=|B|\exp{(i \varphi_B)}$
$C=|C|\exp{(i \varphi_C)}$. Para simplificar, vamos escrever
$\varphi=\omega(t-T)$ e $\Delta \varphi=\varphi_B-\varphi_C$.
Logo,
\begin{equation}
u_1=|B| \cos (\varphi-\varphi_B) \;\;\;,\;\;\;u_2=|C| \cos
(\varphi-\varphi_B+\Delta\varphi)\;.
\end{equation}
Podemos escrever que
\begin{equation}
\frac{u_2}{|C|}=\cos(\varphi-\varphi_B)\cos\Delta\varphi-\sin(\varphi-\varphi_B)\sin\Delta\varphi\;.
\end{equation}
Multiplicando $u_1/|B|$ por $\cos\Delta\varphi$ e subtraindo este
da \'ultima equa\c{c}\~ao, obtemos
\begin{equation}
\frac{u_2}{|C|}-\frac{u_1}{|B|}\cos\Delta\varphi=-\sin(\varphi-\varphi_B)\sin\Delta\varphi
\;.
\end{equation}
Agora, da equa\c{c}\~ao para $u_1$ obtemos
\begin{equation}
\sin^2(\varphi-\varphi_B)=1-\frac{u_1^2}{|B|^2}\;.
\end{equation}
Substituindo este resultado no quadrado da equa\c{c}\~ao anterior,
escrevemos
\begin{equation}
\left(\frac{u_2}{|C|}-\frac{u_1}{|B|}\cos\Delta\varphi\right)^2=\left[1-\frac{u_1^2}{|B|^2}\right]\sin^2\Delta\varphi
\;.
\end{equation}
Ap\'os arranjar os termos, podemos escrever
\begin{equation}
\left(\frac{u_2}{|C|}\right)^2+\left(\frac{u_1}{|B|}\right)^2-2\frac{u_1u_2}{|B||C|}\cos\Delta\varphi=\sin^2\Delta\varphi
\;.
\end{equation}
Esta \'e a equa\c{c}\~ao de uma elipse. Isto quer dizer que o
vetor de deslocamento para $\Delta\varphi \neq k\pi$, tra\c{c}a
uma elipse no plano da frente de fase. Por isso, este tipo de
polariza\c{c}\~ao da onda S \'e chamada {\it polariza\c{c}\~ao
el\'{\i}ptica}. Para $\Delta \varphi = \pm k\pi$ ($k$ inteiro),
esta equa\c{c}\~ao se reduz a
\begin{equation}
u_2 = \pm \frac{|C|}{|B|}u_1 \;,
\end{equation}
que \'e a equa\c{c}\~ao de uma reta. Assim, em situa\c{c}\~oes
especiais, como quando B e C s\~ao reais
($\varphi_B=\varphi_C=0$), a onda S pode ser {\it polarizada
linearmente} no plano da frente de fase. Para o caso em que
$|B|=|C|$ com $\Delta\varphi \neq \pm k\pi$, temos uma {\it
polariza\c{c}\~ao circular}.
