%\documentclass{article}
%\usepackage{amsmath}
%\usepackage{amssymb}
%\usepackage[brazil]{babel}
%\begin{document}

\section{Energia de Deforma\c{c}\~ao}


For\c{c}as externas atuando sobre um corpo deformam o mesmo. Levando em conta este fen\^oneno a energia interna
e a temperatura do corpo podem mudar. No que se segue consideramos processos adiab\'aticos, isto \'e, n\~ao h\'a troca
 de calor com o meio. O fen\^omeno de onda pode ser considerado como um processo adiab\'atico desde que as oscila\c{c}\~oes
sejam t\~ao r\'apidas que n\~ao h\'a troca de calor. A primeira lei da termodin\^amica diz
\[  dU+dK=dA. \]

Essa equa\c{c}\~ao no diz que as energias interna $U$ e cin\'etica $K$ s\~ao balanceadas pelo trabalho $A$ das for\c{c}as no volume e na superf\'\i cie
agindo no volume. Pela equa\c{c}\~ao acima deduzimos que a energia interna prov\'em da energia potencial, ent\~ao introduzimos
\[ U=\int \! \!\int\limits_{V} \! \!\int W dV,\]
onde $W$ \'e chamada densidade de energia de deforma\c{c}\~ao.

\subsection{Trabalho}

Associando as equa\c{c}\~oes j\'a existentes obtemos,
\[ \frac{dA}{dt}=\frac{d}{dt}\int \! \!\int\limits_{V} \! \!\int W dV+\frac{dK}{dt}=\int \! \!\int\limits_{V} \! \!\int \frac{\partial W}{\partial t} dV+\frac{dK}{dt}.\]

Calculemos agora  o trabalho feito pelas for\c{c}as de superf\'\i
cie e volume. O deslocamento de um ponto do volume $V$ no
intervalo de tempo $(t,t+dt)$ \'e $\frac{\partial u_{i}}{\partial
t}dt$. Portanto o trabalho das  for\c{c}as de volume $f_{i}$ em um
dado intervalo de tempo e um volume, ent\~ao $f_{i}\frac{\partial
u_{i}}{\partial t}dt$. Da mesma forma teremos para o trabalho das
for\c{c}as de  superf\'\i cie $T_{i}(\vec{n})\frac{\partial
u_{i}}{\partial t}dt$. Portanto para $\frac{dA}{dt}$ teremos,
\[\frac{dA}{dt}=\int \! \!\int\limits_{V} \! \!\int f_{i}\frac{\partial u_{i}}{\partial t}dV+\int\limits_{S} \! \!\int
\tau_{ji}\mbox{n}_{j}\frac{\partial u_{i}}{\partial t}dS,\] onde
$\mbox{n}_{j}$ \'e o vetor unit\'ario externo. Aplicando a lei de
Gauss, e utilizando a equa\c{c}\~ao do movimento $f_{i}+\tau_{ji,i}=\rho\frac{\partial^{2}u_{i}}{\partial t^{2}}$ temos,
\[\int\limits_{S} \! \!\int \tau_{ji}\mbox{n}_{j}\frac{\partial u_{i}}{\partial t}dS=
\int \! \!\int\limits_{V} \! \!\int \left(\tau_{ji}\frac{\partial
u_{i}}{\partial t}\right)_{,j}dV= \int \! \!\int\limits_{V} \!
\!\int \left(\tau_{ji,j}\frac{\partial u_{i}}{\partial t}+
\frac{\partial u_{i,j}}{\partial t}\tau_{ji}\right)dV=\]
\[=-\int \! \!\int\limits_{V} \! \!\int f_{j}\frac{\partial u_{j}}{\partial t}dV+
\int \! \!\int\limits_{V} \! \!\int \rho\frac{\partial^{2}
u_{j}}{\partial t^{2}}\frac{\partial u_{j}}{\partial t}dV+ \int \!
\!\int\limits_{V} \! \!\int \tau_{ij}\frac{\partial}{\partial
t}e_{ij}dV.\] Para encontrarmos o \'ultimo termo na terceira igualdade utilizamos o fato de que rot$\vec{u}=0$, ou 
seja n\~ao h\'a rota\c{c}\~ao.

Logo,
\[\frac{dA}{dt}=\int \! \!\int\limits_{V} \! \!\int \rho\frac{\partial^{2} u_{j}}{\partial t^{2}}\frac{\partial u_{j}}{\partial t}dV+
\int \! \!\int\limits_{V} \! \!\int
\tau_{ij}\frac{\partial}{\partial t}e_{ij}dV.\]

\subsection{Energia Cin\'etica}

A energia cin\'etica do volume \'e dada por
\[K=\frac{1}{2}\int \! \!\int\limits_{V} \! \!\int \rho\frac{\partial u_{i}}{\partial t}
\frac{\partial u_{i}}{\partial t}dV,\]
de onde conclu\'\i mos que
\[\frac{dK}{dt}=\int \! \!\int\limits_{V} \! \!\int \rho\frac{\partial^{2} u_{i}}{\partial t^{2}}
\frac{\partial u_{i}}{\partial t}dV.\]

Agora a primeira lei da termodin\^amica pode ser escrita da seguinte forma
\[\int \! \!\int\limits_{V} \! \!\int \frac{\partial W}{\partial t} dV-
\int \! \!\int\limits_{V} \! \!\int
\tau_{ji}\frac{\partial}{\partial t}e_{ji}dV=0.\]

Como o volume \'e arbitr\'ario ent\~ao para equa\c{c}\~ao acima ter solu\c{c}\~ao, devemos ter
\[\frac{\partial W}{\partial t} =\tau_{ij}\frac{\partial}{\partial t}e_{ij}.\]

Da equa\c{c}\~ao acima e do fato que $W$ \'e uma fun\c{c}\~ao de $e_{ij}$ conclu\'\i mos que
\[ \tau_{ij}=\frac{\partial W}{\partial e_{ij}}.\]

Como $\tau_{ij}$ \'e uma fun\c{c}\~ao linear de $e_{kl}$, teremos que a derivada da enrgia de deforma\c{c}\~ao
deve ser linear com $e_{kl}$, logo $W$ \'e uma fun\c{c}\~ao quadr\'atica de $e_{kl}$. Definiremos
o estado sem deforma\c{c}\~ao, isto \'e, $e_{kl}=0$, a energia de deforma\c{c}\~ao zero, $W=0$. Ent\~ao
$W$ \'e uma fun\c{c}\~ao quadr\'atica homog\^enea de $e_{kl}$. De acordo coma teorema de Euler
para fun\c{c}\~oes homog\^eneas
\[\frac{\partial W}{\partial e_{kl}}e_{kl}=2W.\]

A partir disto temos,
\[W=\frac{1}{2}\frac{\partial W}{\partial e_{ij}}e_{ij}=\frac{1}{2}\tau_{ij}e_{ij}=
\frac{1}{2}c_{ijkl}e_{kl}e_{ij}.\]

Esta \'e a equa\c{c}\~ao da densidade de energia de deforma\c{c}\~ao.

\section{Fluxo de Energia}

Agora consideramos novamente um volume $V$ arbitr\'ario. Nesta
se\c{c}\~ao estamos interessados em investigar como \'e mantido o
fluxo de energia  em um volume $V$ deformado. A energia el\'astica $\epsilon$ (a soma das energias
cin\'etica e de deforma\c{c}\~ao) no volume $V$,e sendo $E=(
\rho\frac{\partial u_{i}}{\partial t}\frac{\partial u_{i}}{\partial t}+\tau_{ij}e_{ij})$, \'e
\[\epsilon=\int \! \!\int\limits_{V} \! \!\int \mbox{E}dV=\frac{1}{2}\int \! \!\int\limits_{V} \! \!\int\left(
\rho\frac{\partial u_{i}}{\partial t}\frac{\partial u_{i}}{\partial t}+\tau_{ij}e_{ij}\right)dV.\]

A derivada de $\epsilon$ \'e
\[\frac{d\epsilon}{dt}=\int \! \!\int\limits_{V} \! \!\int \frac{\partial \mbox{E}}{\partial t}dV=
\int \! \!\int\limits_{V} \! \!\int \left[\rho\frac{\partial^{2}
u_{i}}{\partial t^{2}} \frac{\partial u_{i}}{\partial t}+\tau_{ij}
\frac{\partial}{\partial t}\left(\frac{\partial u_{i}}{\partial
x_{j}}\right)\right]dV.\]
Note o segundo termo sai da defini\c{c}\~ao de $\tau_{ij}$, quando aplicamos a regra do produto para derivadas.

A segunda fun\c{c}\~ao do integrando da equa\c{c}\~ao acima pode ser reformulada da seguinte
maneira:
\[\int \! \!\int\limits_{V} \! \!\int\tau_{ij}
\frac{\partial}{\partial t}\left(\frac{\partial u_{i}}{\partial
x_{j}}\right)dV= \int \! \!\int\limits_{V} \! \!\int
\frac{\partial}{\partial x_{j}}\left(\tau_{ij}\frac{\partial
u_{i}}{\partial t}\right)dV- \int \! \!\int\limits_{V} \! \!\int
\frac{\partial u_{i}}{\partial t} \tau_{ij,j}dV=\]
\[=-\int \! \!\int\limits_{V} \! \!\int \frac{\partial S_{j}}{\partial x_{j}}-
\int \! \!\int\limits_{V} \! \!\int\frac{\partial u_{i}}{\partial
t}\left(\rho\frac{\partial^{2} u_{i}}{\partial
t^{2}}-f_{i}\right)dV,\]
onde $S_{j}=-\tau_{ij}\frac{\partial
u_{i}}{\partial t}$ e $\tau_{ij,j}$ \'e dada pela equa\c{c}\~ao din\^amica $\tau_{ij,j}=\rho
\frac{\partial^{2} u_{i}}{\partial t^{2}}-f_{i}$.

Substituindo esta express\~ao para $d\epsilon/dt$, teremos
\[\frac{d\epsilon}{dt}+\int \! \!\int\limits_{V} \! \!\int \mbox{div} \vec{S}dV=\int \! \!\int\limits_{V} \! \!\int
\frac{\partial u_{i}}{\partial t}f_{i}dV. \]

Para dar uma interpreta\c{c}\~ao f\'\i sica de
$S_{j}=-\tau_{ij}\frac{\partial u_{i}}{\partial t}$ tomaremos
$f_{i}=0$. Ent\~ao a equa\c{c}\~ao acima pode ser reescrita da
seguinte forma
\[\frac{d\epsilon}{dt}+\int \! \!\int\limits_{V} \! \!\int \mbox{div} \vec{S}dV=0\Leftrightarrow
\frac{d\epsilon}{dt}+\int\limits_{\Sigma} \! \!\int
S_{j}\mbox{n}_{j}d\Sigma=0,\] onde $\Sigma$ denota a superf\'\i
cie limitando o volume. As equa\c{c}\~oes acima dizem que a
derivada da energia $\epsilon$ no volume $V$ \'e balanceada pelo
fluxo do vetor $\vec{S}$ na superf\'\i cie $\Sigma$. Portanto
chamaremos o vetor $\vec{S}$ de fluxo de energia el\'astica. A
dire\c{c}\~ao do vetor $\vec{S}$ especif\'\i ca a dire\c{c}\~ao do
fluxo de energia em um dado ponto, e a seu m\'odulo corresponde a
quantidade de energia que passa em um intervalo de tempo atrav\'es
da superf\'\i cie perpendicular a $\vec{S}$.

Podemos escrever as equa\c{c}\~oes acima em forma de uma
equa\c{c}\~ao de conserva\c{c}\~ao. No caso $f_{i}=0$ temos,
\[\frac{\partial \mbox{E}}{\partial t}+\frac{\partial S_{j}}{\partial x_{j}}=0,\]
e no caso de $f_{i} \neq 0$ temos,
\[\frac{\partial \mbox{E}}{\partial t}+\frac{\partial S_{j}}{\partial x_{j}}=
\frac{\partial u_{i}}{\partial t}f_{i}.\]









%\begin{thebibliography}{bib}
%
%\bibitem{pse}{I. P\v{s}en\v{c}\'\i k - \emph{Introduction to seismic
%methods}, Lecture Notes, PPPG/UFBa, 1994.}
%
%\end{thebibliography}
%\end{document}
