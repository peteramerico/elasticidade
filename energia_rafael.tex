\section{Energia de Deforma\c{c}\~ao}


For\c{c}as externas atuando sobre um corpo deformam o mesmo.
Levando em conta este fen\^oneno a energia interna
e a temperatura do corpo podem mudar. No que se segue,
consideramos processos adiab\'aticos, isto \'e, onde n\~ao h\'a troca
de calor com o meio. Processos de onda podem ser considerados
como adiab\'aticos pois as oscila\c{c}\~oes
ocorrem t\~ao r\'apido que n\~ao h\'a tempo para troca de calor.
Nessa situa\cao, a primeira lei da termodin\^amica pode ser escrita como
\begin{equation}
dU+dK=dA \;.
\end{equation}

Essa equa\c{c}\~ao diz que as mudan\cas\ da energia interna $U$ e da energia cin\'etica $K$ 
num volume $V$ do corpo deformado em quest\~ao
devem ser balanceadas pelo trabalho $A$ das for\c{c}as de volume e de superf\'\i cie
agindo no volume. Pela forma da equa\c{c}\~ao acima
deduzimos que a energia interna $U$ \'e
relacionada a energia potencial. Introduzimos uma quantidade $W$ tal que
\begin{equation}
dU=\int \! \!\int\limits_{V} \! \!\int W dV \;,
\end{equation}
onde $W$ \'e chamada densidade de energia de deforma\c{c}\~ao.


\subsection{Trabalho}

Associando as quantidades da primeira lei da termodin\^amica a uma unidade de tempo, temos
\begin{equation}
\frac{dA}{dt}=\frac{d}{dt}\int \! \!\int\limits_{V} \! \!\int W dV+\frac{dK}{dt}=
\int \! \!\int\limits_{V} \! \!\int \frac{\partial W}{\partial t} dV+\frac{dK}{dt} \;.
\end{equation}

Calculemos agora  o trabalho feito pelas for\c{c}as de superf\'\i
cie e volume. O deslocamento de um ponto do volume $V$ no
intervalo de tempo $(t,t+dt)$ \'e $\frac{\partial u_{i}}{\partial
t}dt$. Portanto, o trabalho das  for\c{c}as de volume $f_{i}$ nesse
intervalo de tempo \'e $f_{i}\frac{\partial
u_{i}}{\partial t}dt$. Da mesma forma, para o trabalho das
for\c{c}as de  superf\'\i cie: $T_{i}(\vec{n})\frac{\partial
u_{i}}{\partial t}dt$. Ent\~a, para $\frac{dA}{dt}$ no volume $V$ temos
\begin{equation}
\frac{dA}{dt}=\int \! \!\int\limits_{V} \! \!\int f_{i}\frac{\partial u_{i}}{\partial t}dV+\int\limits_{S} \! \!\int
\tau_{ji}\mbox{n}_{j}\frac{\partial u_{i}}{\partial t}dS \;,
\end{equation}
onde
$\mbox{n}_{j}$ \'e o vetor unit\'ario externo. Aplicando a lei de
Gauss na integral de superf\'icie, e utilizando a equa\c{c}\~ao do movimento \ref{eq_dinam} temos,
\begin{equation}
\begin{array}{rcl}
\int\limits_{S} \! \!\int \tau_{ji}\mbox{n}_{j}\frac{\partial u_{i}}{\partial t}dS= 
&& \int \! \!\int\limits_{V} \! \!\int \left(\tau_{ji}\frac{\partial
u_{i}}{\partial t}\right)_{,j}dV= \int \! \!\int\limits_{V} \!
\!\int \left(\tau_{ji,j}\frac{\partial u_{i}}{\partial t}+
\frac{\partial u_{i,j}}{\partial t}\tau_{ji}\right)dV \\
&& =-\int \! \!\int\limits_{V} \! \!\int f_{j}\frac{\partial u_{j}}{\partial t}dV+
\int \! \!\int\limits_{V} \! \!\int \rho\frac{\partial^{2}
u_{j}}{\partial t^{2}}\frac{\partial u_{j}}{\partial t}dV+ \int \!
\!\int\limits_{V} \! \!\int \tau_{ij}\frac{\partial}{\partial
t}e_{ij}dV \;.
\end{array}
\end{equation}
%Para encontrarmos o \'ultimo termo na terceira igualdade utilizamos o fato de que rot$\vec{u}=0$, ou 
%seja n\~ao h\'a rota\c{c}\~ao.

Logo,
\begin{equation}
\frac{dA}{dt}=\int \! \!\int\limits_{V} \! \!\int \rho\frac{\partial^{2} u_{j}}{\partial t^{2}}\frac{\partial u_{j}}{\partial t}dV+
\int \! \!\int\limits_{V} \! \!\int
\tau_{ij}\frac{\partial}{\partial t}e_{ij}dV \;.
\end{equation}

\subsection{Energia Cin\'etica}

A energia cin\'etica do volume \'e dada por
\begin{equation}
K=\frac{1}{2}\int \! \!\int\limits_{V} \! \!\int \rho\frac{\partial u_{i}}{\partial t}
\frac{\partial u_{i}}{\partial t}dV \;,
\end{equation}
e sua varia\c{c}\~ao temporal (assumindo que $\rho$ n\~ao varia com o tempo)
\begin{equation}
\frac{dK}{dt}=\int \! \!\int\limits_{V} \! \!\int \rho\frac{\partial^{2} u_{i}}{\partial t^{2}}
\frac{\partial u_{i}}{\partial t}dV \;.
\end{equation}
Agora a primeira lei da termodin\^amica pode ser escrita da seguinte forma
\begin{equation}
\int \! \!\int\limits_{V} \! \!\int \frac{\partial W}{\partial t} dV-
\int \! \!\int\limits_{V} \! \!\int
\tau_{ji}\frac{\partial}{\partial t}e_{ji}dV=0 \;.
\end{equation}
Como o volume \'e arbitr\'ario, ent\~ao para equa\c{c}\~ao acima ter solu\c{c}\~ao, devemos ter
\begin{equation}
\frac{\partial W}{\partial t} =\tau_{ij}\frac{\partial}{\partial t}e_{ij} \;.
\end{equation}
Da equa\c{c}\~ao acima e do fato que $W$ \'e uma fun\c{c}\~ao de $e_{ij}$ conclu\'\i mos que
\begin{equation}
\tau_{ij}=\frac{\partial W}{\partial e_{ij}} \;.
\end{equation}
Como $\tau_{ij}$ \'e uma fun\c{c}\~ao linear de $e_{kl}$,
teremos que a derivada da enrgia de deforma\c{c}\~ao
deve ser linear com $e_{kl}$, logo $W$ \'e uma fun\c{c}\~ao
quadr\'atica de $e_{kl}$. Definiremos
que no estado sem deforma\c{c}\~ao, isto \'e, quando
$e_{kl}=0$, a energia de deforma\c{c}\~ao \'e zero, $W=0$. Ent\~ao
$W$ \'e uma fun\c{c}\~ao quadr\'atica homog\^enea de $e_{kl}$. De acordo com o teorema de Euler
para fun\c{c}\~oes homog\^eneas
\begin{equation}
\frac{\partial W}{\partial e_{kl}}e_{kl}=2W \;.
\end{equation}
A partir disto temos,
\begin{equation}
W=\frac{1}{2}\frac{\partial W}{\partial e_{ij}}e_{ij}=\frac{1}{2}\tau_{ij}e_{ij}=
\frac{1}{2}c_{ijkl}e_{kl}e_{ij} \;.
\end{equation}
Esta \'e a equa\c{c}\~ao da densidade de energia de deforma\c{c}\~ao.

\section{Fluxo de Energia}

Agora consideramos novamente um volume $V$ arbitr\'ario. Nesta
se\c{c}\~ao estamos interessados em investigar como \'e mantido o
fluxo de energia  em um volume $V$ deformado. A energia el\'astica $\epsilon$ (a soma das energias
cin\'etica e de deforma\c{c}\~ao) no volume $V$,e sendo $E=(
\rho\frac{\partial u_{i}}{\partial t}\frac{\partial u_{i}}{\partial t}+\tau_{ij}e_{ij})$,
\'e
\begin{equation}
\epsilon=\int \! \!\int\limits_{V} \! \!\int \mbox{E}dV=\frac{1}{2}\int \! \!\int\limits_{V} \! \!\int\left(
\rho\frac{\partial u_{i}}{\partial t}\frac{\partial u_{i}}{\partial t}+\tau_{ij}e_{ij}\right)dV \;.
\end{equation}

A derivada de $\epsilon$ \'e
\begin{equation}
\frac{d\epsilon}{dt}=\int \! \!\int\limits_{V} \! \!\int \frac{\partial \mbox{E}}{\partial t}dV=
\int \! \!\int\limits_{V} \! \!\int \left[\rho\frac{\partial^{2}
u_{i}}{\partial t^{2}} \frac{\partial u_{i}}{\partial t}+\tau_{ij}
\frac{\partial}{\partial t}\left(\frac{\partial u_{i}}{\partial
x_{j}}\right)\right]dV \;.
\end{equation}
Note o segundo termo sai da defini\c{c}\~ao de $\tau_{ij}$, quando aplicamos a regra do produto para derivadas.

A segunda fun\c{c}\~ao do integrando da equa\c{c}\~ao acima pode ser reformulada da seguinte
maneira:
\begin{equation}
\begin{array}{rcl}
\int \! \!\int\limits_{V} \! \!\int\tau_{ij}
\frac{\partial}{\partial t}\left(\frac{\partial u_{i}}{\partial
x_{j}}\right)dV=
&& \int \! \!\int\limits_{V} \! \!\int
\frac{\partial}{\partial x_{j}}\left(\tau_{ij}\frac{\partial
u_{i}}{\partial t}\right)dV- \int \! \!\int\limits_{V} \! \!\int
\frac{\partial u_{i}}{\partial t} \tau_{ij,j}dV \\
&& =-\int \! \!\int\limits_{V} \! \!\int \frac{\partial S_{j}}{\partial x_{j}}-
\int \! \!\int\limits_{V} \! \!\int\frac{\partial u_{i}}{\partial
t}\left(\rho\frac{\partial^{2} u_{i}}{\partial
t^{2}}-f_{i}\right)dV \;,
\end{array}
\end{equation}
onde $S_{j}=-\tau_{ij}\frac{\partial
u_{i}}{\partial t}$ e $\tau_{ij,j}$ \'e dada pela equa\c{c}\~ao din\^amica $\tau_{ij,j}=\rho
\frac{\partial^{2} u_{i}}{\partial t^{2}}-f_{i}$.

Substituindo esta express\~ao para $d\epsilon/dt$, teremos
\begin{equation}
\frac{d\epsilon}{dt}+\int \! \!\int\limits_{V} \! \!\int \mbox{div} \vec{S}dV=\int \! \!\int\limits_{V} \! \!\int
\frac{\partial u_{i}}{\partial t}f_{i}dV \;.
\end{equation}

Para dar uma interpreta\c{c}\~ao f\'\i sica de
$S_{j}=-\tau_{ij}\frac{\partial u_{i}}{\partial t}$ tomaremos
$f_{i}=0$. Ent\~ao a equa\c{c}\~ao acima pode ser reescrita da
seguinte forma
\begin{equation}
\frac{d\epsilon}{dt}+\int \! \!\int\limits_{V} \! \!\int \mbox{div} \vec{S}dV=0\Leftrightarrow
\frac{d\epsilon}{dt}+\int\limits_{\Sigma} \! \!\int
S_{j}\mbox{n}_{j}d\Sigma=0 \;,
\end{equation}
onde $\Sigma$ denota a superf\'\i
cie limitando o volume. As equa\c{c}\~oes acima dizem que a
derivada da energia $\epsilon$ no volume $V$ \'e balanceada pelo
fluxo do vetor $\vec{S}$ na superf\'\i cie $\Sigma$. Portanto
chamaremos o vetor $\vec{S}$ de fluxo de energia el\'astica. A
dire\c{c}\~ao do vetor $\vec{S}$ especif\'\i ca a dire\c{c}\~ao do
fluxo de energia em um dado ponto, e a seu m\'odulo corresponde a
quantidade de energia que passa em um intervalo de tempo atrav\'es
da superf\'\i cie perpendicular a $\vec{S}$.

Podemos escrever as equa\c{c}\~oes acima em forma de uma
equa\c{c}\~ao de conserva\c{c}\~ao. No caso $f_{i}=0$ temos,
\begin{equation}
\frac{\partial \mbox{E}}{\partial t}+\frac{\partial S_{j}}{\partial x_{j}}=0 \;,
\end{equation}
e no caso de $f_{i} \neq 0$ temos,
\begin{equation}
\frac{\partial \mbox{E}}{\partial t}+\frac{\partial S_{j}}{\partial x_{j}}=
\frac{\partial u_{i}}{\partial t}f_{i} \;.
\end{equation}









%\begin{thebibliography}{bib}
%
%\bibitem{pse}{I. P\v{s}en\v{c}\'\i k - \emph{Introduction to seismic
%methods}, Lecture Notes, PPPG/UFBa, 1994.}
%
%\end{thebibliography}
%\end{document}
