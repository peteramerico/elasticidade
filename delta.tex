
\section{A fun\cao\ delta de Dirac}

Uma fun\cao\ extremamente \'util para a descri\cao\ de solu\coes\ de
equa\coes\ diferenciais \'e a fun\cao\ delta de Dirac. Ela \'e definida
pelas seguintes propriedades.
\begin{enumerate}
\item \hspace*{1cm} $\displaystyle \delta(t) = 0 \; \forall \; t \neq 0 $,
\item \hspace*{1cm} $\displaystyle \intii \delta(t) dt = 1 $.
\end{enumerate}
Obviamente, estas duas propriedades s\ao\ mutuamente exclusivas para
fun\coes\ comuns, uma vez que o valor da integral sobre uma fun\cao\ n\ao\ altera se o
valor da fun\cao\ e alterada em um \'unico ponto. Portanto, uma fun\cao\
que satisfaz a condi\cao\ 1. nunca pode satisfazer a condi\cao\ 2. e
vice versa. Por reunir as duas propriedades, a ``fun\cao\ delta'',
introduzida por Paul Dirac, foi a primeira de uma nova classe de
``fun\coes'', mais tarde chamadas de ``fun\coes\ generalizadas'' ou
``distribui\coes''. Apesar disso, \'e de praxe comum na literatura
o uso da palavra ``fun\cao'' quando se refere \`a $\delta(t)$. Seguimos
aqui este costume.

Para que a fun\cao\ delta possa satisfazer duas propriedades ao mesmo
tempo, \'e necess\'ario que ela esteja infinita na origem. Por\'em, este
"infinito" n\ao\ \'e um infinito comum. Observamos que normalmente,
esperariamos que $2\delta(t)$ seria igual a $\delta(t)$, uma vez que
duas vezes infinito ainda \'e infinito. Por\'em, pela condi\cao\ 2.,
vemos que a fun\cao\ delta \'e normalizada que implica que
$2\delta(t)\neq\delta(t)$.

\subsection{Defini\cao\ da fun\cao\ delta}

A fun\cao\ delta \'e utilizada para descrever excita\coes\ pontuais no
espa\co\ ou no tempo. Para introduzirmos excita\coes\ pontuais,
consideramos uma fun\cao\ auxiliar, $d(t)$, que seja n\ao\ nula somente
em um intervalo pequeno $(-\eps,\eps)$, onde $\eps > 0$.
Observamos que 
\begin{equation}
\intii d(t) \, dt = 
\int_{-\eps}^{\eps} d(t) \, dt 
\label{eq:idt}
\end{equation}
Para que esta fun\cao\ seja normalizada, queremos que a integral acima
forneca o valor um. Uma possibilidade de definir a fun\cao\ $d(t)$ \'e
\begin{equation}
d(t) = \left\{ \begin{array}{ccc}
\displaystyle \frac{1}{2\eps} & \forall & -\eps < t <
\eps \\
0 && \mbox{sen\ao}.
\end{array}
\right.
\label{eq:dt}
\end{equation}
Notamos que esta fun\cao\ descreve um pequeno ret\^angulo.
Com esta defini\cao, a integral acima \'e
\begin{equation}
\intii d(t) \, dt = 
\int_{-\eps}^{\eps} \frac{1}{2\eps} \, dt =
\left.\frac{t}{2\eps}\right|_{-\eps}^{\eps} = 1,
\label{eq:idtu}
\end{equation}
independentemente do valor de $\eps$. Portanto,
\begin{equation}
\intii \lim_{\eps\to0} d(t) \, dt = 
\lim_{\eps\to0} \intii d(t) \, dt = 1.
\label{eq:idtl}
\end{equation}
Por outro lado, para todo $t\neq0$, temos que $\displaystyle
\lim_{\eps\to0} d(t) = 0$. Desta forma, observamos que a fun\cao\
\begin{equation}
\delta(t) = \lim_{\eps\to0} \, d(t)
\label{eq:defdel}
\end{equation}
satisfaz as duas condi\coes\ da fun\cao\ delta.

Devemos notar que esta defini\cao\ da fun\cao\ delta n\ao\ \'e \'unica.
Todas as poss\iveis\ defini\coes\ da fun\cao\ delta usam um procedimento
correspondente ao de cima. O que muda \'e a fun\cao\ que faz o papel da
$d(t)$. Outras possibilidades incluem
\begin{equation}
d(t) = \left\{ \begin{array}{ccc}
\displaystyle \frac{\eps+t}{\eps^2} & \forall & -\eps < t < 0 \\
\displaystyle \frac{\eps-t}{\eps^2} & \forall & 0 \leq t < \eps \\
0 && \mbox{sen\ao}.
\end{array} \right.
\label{eq:dta}
\end{equation}
Esta fun\cao\ representa um pequeno tri\^angulo. Tamb\'em podem ser
utilizadas a fun\cao\ em forma de sino ({\it bell-shaped function})
\begin{equation}
d(t) = \frac{1}{\pi} \frac{\eps}{\eps^2+t^2}
\label{eq:dtb}
\end{equation}
e a Gaussiana
\begin{equation}
d(t) = \frac{1}{\sqrt{\pi\eps}} \, e^{-t^2/\eps} \; .
\label{eq:dtc}
\end{equation}
Para as duas \'ultimas, o desenvolvimento acima precisa ser modificado
ligeiramente, uma vez que estas s\ao\ diferentes de zero no eixo $t$
inteiro para todo $\eps\neq0$ e s\'o tendem a zero para todo $t\neq0$
quando $\eps$ tende a zero. Mesmo assim, ambas t\^em \'area unit\'aria
independentemente do valor de $\eps$ e, portanto, possuem a propriedade
\refi{eq:idtl}. Desta forma, tendem \`a fun\cao\ delta quando $\eps$
tende a zero.

\subsection{Propriedades da fun\cao\ delta}

Utilizando a fun\cao\ $d(t)$ definida na equa\cao\ \refi{eq:dt}, bem
como seu limite quando $\eps$ tende a zero, podemos discutir as
propriedades da fun\cao\ delta. Primeiramente, observamos que a fun\cao\
delta \'e uma fun\cao\ par, que segue da condi\cao\ 1., junto com a
observa\cao\ que
\begin{equation}
\intii \delta(-t) \, dt =  
- \intii \delta(\tau) \, d\tau =  
\intii \delta(\tau) \, d\tau \; .
\label{eq:delpar}
\end{equation}

Para uma fun\cao\ cont\inua\ e limitada $f(t)$, estudamos a integral
\begin{eqnarray}
\intii \delta(t-\tau)f(t) \, dt
& = & \intii \lim_{\eps\to0} d(t-\tau) f(t) \, dt
\nonumber \\ & = &
\lim_{\eps\to0} \intii d(t-\tau) f(t) \, dt
%\nonumber \\ & = &
=
\lim_{\eps\to0} \int_{\tau-\eps}^{\tau+\eps}
\frac{f(t)}{2\eps} \, dt.
\end{eqnarray}
Pelo teorema do valor m\'edio da integra\cao,
\begin{equation}
\int_a^b f(t) \, dt = f(\xi) (b-a) \qquad (a\leq\xi\leq b) \; ,
\label{eq:tvm}
\end{equation}
concluimos que
\begin{equation}
\int_{\tau-\eps}^{\tau+\eps} \frac{f(t)}{2\eps} \, dt
= \frac{f(\xi)}{2\eps} \, 2\eps = f(\xi) 
\qquad (\tau-\eps \leq \xi \leq \tau+\eps) \; ,
\label{eq:res}
\end{equation}
e, portanto, no limite quando $\eps$ tende a zero,
\begin{equation}
\intii \delta(t-\tau)f(t) \, dt = f(\tau) \; .
\label{eq:samp}
\end{equation}

Em conseq\"u\^encia das propriedades \refi{eq:delpar} e \refi{eq:samp},
a fun\cao\ delta \'e a fun\cao\ unit\'aria da convolu\cao,
i.e.,
\begin{equation}
f(t) * \delta(t) = \intii f(\tau) \delta(t-\tau) \, d\tau = 
\intii f(\tau) \delta(\tau-t) \, d\tau = f(t) \; .
\label{eq:unconv}
\end{equation}

Outra propriedade da fun\cao\ delta em conseq\"u\^encia da propriedade
\refi{eq:samp} \'e o fato da transformada de Fourier dela ser facil de
calcular, fornecendo
\begin{equation}
\hat{\delta}(\omega) = \TF{\delta(t)}
= \intii \delta(t) e^{-i\omega t} dt
= e^{-i\omega 0} = 1 \; .
\label{eq:ftdel}
\end{equation}
Pela transformada de Fourier inversa, podemos escrever
\begin{equation}
\delta(t) = \frac{1}{2\pi} \intii \hat{\delta}(\omega) e^{i\omega t} d\omega 
= \frac{1}{2\pi} \intii e^{i\omega t} d\omega \; .
\label{eq:ftidel}
\end{equation}

Mediante o par de transformadas de Fourier, mostra-se tamb\'em que
\begin{equation}
\TF{\delta(ct)} = \intii \delta(ct) e^{-i\omega t} dt
= \frac{1}{c} \intii \delta(\tau) e^{-i\frac{\omega}{c}\tau} d\tau =
\frac{1}{c} = \frac{1}{c} \hat{\delta}(\omega) \; ,
\label{eq:scadelf}
\end{equation}
e, portanto, 
\begin{equation}
\delta(ct) = \frac{1}{c} \delta(t) \; .
\label{eq:scadel}
\end{equation}
Esta propriedade \'e fundamental quando queremos analisar a unidade de
uma express\ao\ que envolve a fun\cao\ delta. Da equa\cao\
\refi{eq:scadel} segue, por exemplo, que, quando $t$ simboliza o tempo em
segundos, $\delta(t)$ tem a unidade de Hz = s$^{-1}$.

\subsection{Fun\cao\ delta multidimensional}

Para descrever a\coes\ pontuais em mais do que uma dimens\ao, pode-se
usar uma fun\cao\ delta multidimensional. Por exemplo, se $\vec x$
denota um vetor tridimensional com componentes $x$, $y$, e $z$, a
fun\cao\ delta tridimensional pode ser definida por
\begin{equation}
\delta(\vec x) = \delta(x)\delta(y)\delta(z) \; .
\label{eq:delmul}
\end{equation}
Uma outra possibilidade \'e a defini\cao\ mediante uma fun\cao\
$d(\vec{x})$, cujo limite tenda \`a fun\cao\ delta tridimensional, i.e.,
\begin{equation}
\delta(\vec x) = \lim_{\eps\to0} d(\vec x) \; .
\label{eq:del3d}
\end{equation}
Um exemplo para uma fun\cao\ assim \'e
\begin{equation}
d(\vec x) = \left\{ \begin{array}{ccc}
\displaystyle \frac{3}{4\pi\eps^3} & \forall & -\eps < r < \eps \\
0 && \mbox{sen\ao},
\end{array}
\right.
\label{eq:dt3d}
\end{equation}
onde $r=\sqrt{x^2+y^2+z^2}$. Esta fun\cao\ $d(\vec{x})$ \'e diferente de
zero em uma pequena esfera em volta da origem. Outras possibilidades
correspondentes \`as diferentes fun\coes\ $d(t)$ acima tamb\'em podem
ser utilizados, por exemplo representando pequenos cubos,
paralelep\ipedos, cilindros, Gaussianas ou sinos tridimensionais. Para
todas elas, \'e necess\'ario que $\displaystyle \lim_{\eps\to0} d(\vec
x) = 0$ para $r\neq0$ e que
\begin{equation}
\intii \intii \intii d(\vec x) \, dxdydz = 1,
\label{eq:volum}
\end{equation}
independentemente do valor de $\eps$. N\ao\ \'e dif\icil\ enxergar que
estas condi\coes\ s\ao\ satisfeitas pela fun\cao\ $d(\vec{x})$ da
equa\cao\ \refi{eq:dt3d}.

De forma eq\"uivalente \`a fun\cao\ delta unidimensional, a fun\cao\ delta
multidimensional possui a propriedade 
\begin{equation}
\intii \intii \intii d(\vec x-\vec x_0) f(\vec x) \, dxdydz = f(\vec x_0),
\label{eq:samp3d}
\end{equation}
com as correspondentes conseq\"u\^encias.
